\chapter*{Introduction} \pagenumbering{arabic}
\addcontentsline{toc}{chapter}{Introduction}
This thesis is part of a long-standing project between UCLouvain and IRCAM (\textit{Institut de Recherche et Coordination Acoustique/Musique})\parencite{IRCAM}. It is one more step towards a complete tool capable of creating music from a musical style to assist composers without programming skills by automating repetitive, arduous, or time-consuming tasks. On the technical level, it is a question of formalizing musical rules in discrete mathematics in order to represent them in the form of constraints. The tool uses constraint programming (CP) to find solutions satisfying previously chosen musical rules. It is a composition assistant and not an independent composer. Ideally, the tool serves as a first draft for the composer so that he can adapt the solution or seek a new solution that better meets his needs.\\

Currently, this tool is achieved through the use of Gecode\parencite{Gecode} for the CP part and OpenMusic\parencite{OpenMusic} (OM) for the user interface. Gecode is an open-source C++ toolkit for developing constraint-based systems and applications. While OM is an open-source Lisp graphical programming environment for music composition. The link between Gecode and OM is done via the wrapper GiL\parencite{GiL}.\\

But why constraint programming? It is an innovative approach in the field of music computing. Today, there are several machine learning (ML) models capable of reproducing certain styles of music more or less faithfully (e.g. MuseNet by OpenAI\parencite{MuseNet} or Music Transformer by Magenta\parencite{MusicTrans}). Unlike machine learning, constraint programming allows full transparency on how solutions are generated and full control over the musical rules that apply\footnote{Opacity in ML is a recurring problem that is still one of the main challenges today. For example, \textcite{ferreira2021learning} explain that "it is very hard to control such models in order to guide the compositions towards a desired goal".}. In a way, this paradigm allows the model to "understand" the music it generates. This is practical to leave to the composer the possibility of making the music he truly wants. CP is a very powerful paradigm still under-exploited in the field of computer music allowing to generate musically correct solutions in a few seconds, or even less.\\

Before embarking on an overly complex formalization of music, it is necessary to prove that this kind of tool is feasible with a musical style that is not too complex and fairly strict. It is for this reason that this thesis focuses on the formalization of the rules of Johann Joseph Fux's \citetitle{IMSLPlatin}\parencite{IMSLPlatin} on two-voice counterpoint. Counterpoint is a musical style, mostly developed during the Baroque era, which consists of having several melodies played at the same time\parencite{CpSachs}. These melodies are harmonically interdependent but melodically independent\parencite{CpLaitz}. They are all built from a \cfcomma i.e. the "given song" which determines the context of the musical piece. It is a horizontal construction of music which is partly opposed to the common harmonic vertical approach.\\

The \citetitle{IMSLPlatin} is a good book to start this project for several reasons. First, this work is recognized as a pillar of the theory of contrapuntal composition. These rules are therefore still applied by contemporary composers. Second, Fux's work is separated into several chapters according to the species of counterpoint. Without going into details, this allows to formalize each type of counterpoint iteratively without getting confused.\\

To summarize, the purpose of this thesis is, on the one hand, to create a tool capable of generating a counterpoint based on a \cfcomma on the other hand, to determine the advantages and disadvantages of CP for composition computer-aided.\\

This work will be divided into several parts: it will first briefly present the work taking place in the vast field of computer-assisted composition. Then, the musical and technical knowledge for a good understanding of the work will be established. Of course, most of this paper consists in formalizing in the form of mathematical constraints the different rules that can be extracted from Fux's work. For this, a chapter will be devoted to the system of variables on which all the constraints are created. There will follow a chapter by species detailing each rule in natural language and then in mathematical constraints. Before concluding, the evaluation of all species will be done to determine if the CP approach was realistic or not. Finally, criticisms and suggestions will be given in order to continue this big project as well as possible.