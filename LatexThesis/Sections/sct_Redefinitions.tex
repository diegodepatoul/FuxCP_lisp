\chapter{Defining some concepts and redefining the variables} 
\section{Voices, parts and strata}\label{section:parts-and-strata}
Before we start this section, we need to look at some vocabulary to make sure we understand what we are discussing. The most important definition (and distinction) we will introduce is the definition of the terms \textit{part} and \textit{stratum}. The need for these definitions arises from the increasing complexity of the rules of counterpoint when it is generalised to three voices. Indeed, the rules are no longer (as we shall see later) concerned solely with counterpoint and its cantus firmus, but also with new concepts, such as that referred to by Fux as 'the lowest voice'. As the term 'voice' is too generic (it is used in Fux's text to describe notions as different as counterpoint, cantus firmus, voice range and the so called 'lowest voice'), we need to create a precise vocabulary that is different from the word 'voice' to talk about these new concepts. 

With this in mind, let's explain what 'parts' and 'strata' are. Each of these two concepts is a type of voice, which means that in a composition with $n$ voices, there will also be $n$ parts and $n$ strata.

\subsubsection{Parts}
The parts correspond to what a given individual sing. They correspond to a staff (each staff corresponds to one part). The term 'part' is the same as that used by Fux in his work. The three parts in a three voices composition are: the \cf, the first \cp and the second \cp. Fux differentiates them by calling them by the name of their voice range, that is: "bass", "tenor", "alto" or "soprano".

\subsubsection{Strata}
As for the strata, they are defined like this: a stratum delineates discrete layers or levels of pitches at any given moment in the composition. It denotes a vertical alignment of simultaneous notes and organizes them into distinct strata. By definition, the lowest stratum encompasses the lowest sounding notes, the highest stratum comprises the highest sounding notes, and intermediary strata represent pitch levels in between.
This concept is very helpful in identifying and categorising the vertical placement of pitches, creating distinct categories of sound within the overall texture of the counterpoint composition. It provides a way of analysing and understanding the distribution of pitches across different parts, allowing more complex rules to be established: for example, it would now be possible to establish a rule between the notes of the cantus firmus and the highest sounding notes (no matter which part they come from). The full potential of strata lies in harmonic rules, but as we shall see, some melodic rules are also related to it.

\begin{minipage}{0.6\textwidth}
    The term stratum was chosen in this context for its visual impact. In geology, a stratum "is a rock layer with a lithology (texture, colour, grain size, composition, fossils, etc.) different from the adjacent ones", see figure \ref{fig:geological-strata}.
    \end{minipage}
    \hfill
    \begin{minipage}{0.3\textwidth}
      \centering
      \includegraphics[width=\textwidth]{Images/rainbow-sediment.jpg}
      \captionof{figure}{Geological strata, for the illustration}
      \label{fig:geological-strata}
\end{minipage}
\vspace{.5cm}

When Fux speaks about the lowest stratum, he often uses the word 'bass'. It was deliberately chosen to speak about the 'lowest stratum' instead of the 'bass' (like Fux does), because 'bass' is also the name of a range of voices (on a par with soprano and alto, for example), and there is already enough complexity in all the terminology to add even further ambiguity. 

\paragraph{}
These new terms (parts and strata) are used where the distinction between the concepts is important. Whenever this distinction is not relevant, the more general term 'voice' is used to reduce the complexity of reading. In this case, the 'voice' could refer to both a stratum and a part.

Since a picture is worth a thousand words, Figure \ref{fig:lowest} illustrates the difference between parts (the blue lines) and strata (the red and orange lines). The lowest stratum is shown in its own colour (red) because it is the most meaningfull stratum, and it will be particularly important later on.

\begin{figure}[h]
  \centering
  \includegraphics[width=1\textwidth]{Images/strata_example.png}
  \caption{Parts and strata in a three voice composition}
  \label{fig:lowest}
\end{figure}


Here is also the mathematical representation for the notes of the lowest stratum (written $N(a)$, see section \ref{section:changes induced} for the notations):
% todo a, b, c, et A
\begin{equation}
    \forall i \in [0, 3] \quad \forall j \in [0, m-1): N(a)[i,j] = \text{max} (N(cf)[i,j], N(cp_1)[i,j], N(cp_2)[i,j])
\end{equation}

Of the first upper stratum, or medium stratum (written $N(b)$, see section \ref{section:changes induced} for the notations):
\begin{equation}
    \forall i \in [0, 3] \quad \forall j \in [0, m-1): N(a)[i,j] = \text{med} (N(cf)[i,j], N(cp_1)[i,j], N(cp_2)[i,j])
\end{equation}

And of the second upper stratum, or uppest stratum (written $N(c)$, see section \ref{section:changes induced} for the notations):
\begin{equation}
    \forall i \in [0, 3] \quad \forall j \in [0, m-1): N(a)[i,j] = \text{min} (N(cf)[i,j], N(cp_1)[i,j], N(cp_2)[i,j])
\end{equation}

\subsubsection{One part per stratum and one stratum per part} \label{subsubsection:one-part-per-stratum}
It is important to note that, for each musical measure, there is a bijection between the individual parts and the corresponding strata. This means that, for any given measure, each stratum uniquely corresponds to a single part, and vice versa. Put differently, if two parts within a measure share the same pitch, they do not constitute the same stratum. Instead, one part corresponds with one stratum, and the other one to a separate stratum.

To illustrate this, consider a scenario in a two-voice composition (see figure \ref{fig:one-voice-max-can-be-a}), where part 'cf' and part 'cp1' in measure X both have a pitch value of 67 (representing a G). Despite having identical pitches at the same moment, one part is categorised as the lowest stratum, while the other is designated as the uppermost stratum. This distinction becomes crucial for subsequent analysis, especially when calculating aspects like motions.

To know which part gets to be the lowest stratum in such situations, an arbitrary hierarchical rule is implemented. If the ambivalence is between the \cfs and another part, the \cfs is always prioritised and assigned the role of the lowest stratum, over any other part. In the case of a ambivalence between the first \cp and the second \cp, the first \cp is given the status of the lowest stratum. 

\begin{figure}[h]
    \centering
    \includegraphics[width=.5\textwidth]{Images/one-voice-max-can-be-a.png}
    \caption{Establishing which part corresponds to the lowest stratum}
    \label{fig:one-voice-max-can-be-a}
  \end{figure}

\section{Exploring the interaction of the parts with the lowest stratum}

One of the major differences between the composition of two voices (i.e., one \cfs and one counterpoint) and the generalisation to three voices (i.e., one \cfs and two counterpoints) is that the rules no longer necessarily apply between the counterpoints and the \cf. If we go back to the rules for two voices, we see that each of them applied between the single counterpoint and the \cf. For example, when it was stated that each interval must be consonant, this referred to the interval between the counterpoint and the \cf.
On the other hand, in his second part (where he describes the rules for composing in three voices), Fux explains that the rules are not necessarily to be observed between each of the counterpoints and the \cf, but rather between "each of the voices and the lowest voice" (i.e. the lowest stratum). Again, if we take the example of the need for consonance between the voices, consonance will be required in the intervals between the notes of any voice and those of the lowest voice (whether or not the latter is the \cf).
Fux approaches the concept of lowest stratum without ever stating it clearly, mentioning for example that the lowest voice can change (sometimes the bass is the lowest voice, sometimes the tenor, ...), and that at any given moment the lowest voice should be considered. In other words, Fux says that the rules apply between the parts and the lowest stratum.

\paragraph{}
One might be tempted to conclude that three-part composition breaks completely with two-part composition, but that would be too hasty a conclusion. Indeed, on closer inspection, the way the rules worked in two-part composition (from counterpoint to \textit{cantus firmus}) is just one particular case of this new vision of things. In two-part composition, too, the rules apply between the parts and the lowest stratum. But of course, since there were only two voices, the lowest stratum was either counterpoint or cantus firmus. This means that when links were established between the upper part and the lowest stratum, links were also established between the counterpoint and the cantus firmus. Considering the rules as being established between the counterpoint and the \cfs was just a simplification of reality, although it was perfectly correct. We were therefore considering a convenient particular case, and not the general case. Please note that when we talk about "applying constraints from voice A to voice B", it is clear that the constraints are bidirectional and that they also apply from voice B to voice A. What is shown here is rather the philosophy behind the application of these constraints, and the reasons why they were imposed.

The particular case happening when composing with two parts is illustrated in figures \ref{fig:cp2cf-2v} and \ref{fig:p2l-2v}. As we can see on those pseudo-compositions, it does not change anything to apply the constraints from the counterpoints to the \cfs or from the parts to the lowest stratum.

\vspace{.5cm}
\begin{minipage}{0.46\textwidth}
    \centering
    \includegraphics[width=\textwidth]{Images/cp2cf-2v.png}
    \captionof{figure}{Applying the constraints from the ctp. to the \cf}
    \label{fig:cp2cf-2v}
    \end{minipage}
    \hfill
    \begin{minipage}{0.46\textwidth}
      \centering
      \includegraphics[width=\textwidth]{Images/p2l-2v.png}
      \captionof{figure}{Applying the constraints from the parts to the lowest stratum}
      \label{fig:p2l-2v}
\end{minipage}
\vspace{.5cm}

However, when it comes to generalising the composition of counterpoint for three voices, it is no longer possible to follow the same simplification. We are now forced to set our rules between the parts and the lowest stratum, and no longer between the counterpoints and the \cf. On figures \ref{fig:cp2cf-3v} and \ref{fig:p2l-3v}, it is made clear that establishing the rules from the counterpoints to the \cfs is really different than applying them from the various parts to the lowest stratum. In those figures, the parts don't cross and therefore match perfectly with the strata, hence the constraints are always applied to the same counterpoint. This was made for the sake of intellegibility of the graphs, but it is obviously possible that the parts cross and that the "target" of the constraints is not always the same counterpoint.

\vspace{.5cm}
\begin{minipage}{0.46\textwidth}
    \centering
    \includegraphics[width=\textwidth]{Images/cp2cf-3v.png}
    \captionof{figure}{Applying the constraints from the ctp. to the \cf}
    \label{fig:cp2cf-3v}
    \end{minipage}
    \hfill
    \begin{minipage}{0.46\textwidth}
      \centering
      \includegraphics[width=\textwidth]{Images/p2l-3v.png}
      \captionof{figure}{Applying the constraints from the parts to the lowest stratum}
      \label{fig:p2l-3v}
\end{minipage}
\vspace{.5cm}

It is, of course, possible for the \cfs to be equal to the lowest stratum all along, in which case nothing changes from the perspective we had when composing for two voices. In this particular case, by applying the rules with respect to the \cf, we would find ourselves de facto applying the rules with respect to the lowest stratum (and we would be back to the situation described above, see figures \ref{fig:cp2cf-3v} and \ref{fig:p2l-3v}, only that there is now one more part). It is when the \cfs pitches are higher up than those of the counterpoints that considering the lowest stratum consideration becomes necessary.

\paragraph{}
A very important detail, and perhaps the biggest change brought about by this paradigm shift, is that the \cfs now becomes a counterpoint like any other, with the difference that its notes are already fixed. This means that we will treat the \cfs as a first species counterpoint (since it is always made up of whole notes), and that we will have to calculate its intervals, movements and costs, as we have already done with 'normal' counterpoints.

A second point to bear in mind, and not the least, is that all this does not mean that \textit{all} the rules are established between the parts and the lowest stratum. Certain rules will continue to apply between the different parts, regardless of whether they are high, low or intermediate.

\section{Changes induced in the variables} \label{section:changes induced}


Many changes have been induced as a result of the three part generalisation. In two part composition, it was obvious that the harmonic intervals array was describing the intervals between the \cfs and the only counterpoint, it was obvious that the motions were those of the only counterpoint, and so it goes for all the variables. When writing a three part composition, we are dealing with many possibilities when speaking about intervals or motions. Intervals between which voices? Motions of which counterpoint? To tackle this, each variable is now related to a voice.

This relationship between a variable and a voice is expressed as function. $X(v)$ represents the variable $X$ of the voice $v$. The arguments of the function can either be:
\begin{itemize}
    \item $cf$ - for linking the variable to the \cf.
    \item $cp_1$ - for linking the variable to the second \cp.
    \item $cp_2$ - for linking the variable to the third \cp.
    \item $a$ - for linking the variable to the lowest stratum.
    \item $b$ - for linking the variable to the intermediate stratum.
    \item $c$ - for linking the variable to the uppermost stratum.
  \end{itemize}

\noindent For example, $X(cf)$ refers to the variable $X$ of the \cf.

\paragraph{}
When a variable is not explicitly linked to a voice, it is implied that the relation expressed for it is true for all \textit{parts}. In other words, if the variable $X$ is written without any precision, it means $\forall v \in \{cf, cp_1, cp_2\}: X(v)$.

\paragraph{}
\noindent This change applies to all variables, namely:
\begin{itemize}
    \item \textbf{N}(v) - the notes (pitches) of voice v (this variable was written cp in T. Wafflard's thesis)
    \item \textbf{H}(v$_1$, v$_2$) - the harmonic intervals between voice v$_1$ and voice v$_2$. This variable is particular, as it needs to arguments to be meaningful.
    \item \textbf{M}(v) - the melodic intervals of voice v, 
    \item \textbf{P}(v) - the motions of voice v, 
    \item \textbf{IsCfB}(v) - the boolean array representing whether the cantus firmus is lower than a given voice,
    \item \textbf{IsCons}(v) - to the boolean array representing whether voice v is consonant or not.
\end{itemize}

\noindent It also applies to \textit{some} constants and arrays, namely:
\begin{itemize}
    \item \textbf{species}(p) - the species of part p,
    \item \textbf{n}(p) - the amount of notes in part p,
    \item \textbf{lb}(p) - the lower bound of the range of part p,
    \item \textbf{ub}(p) - the upper bound of the range of part p,
    \item $\mathcal{R}$(p) - the range of part p,
    \item \textbf{borrow}(p) - the borrowing scale of part p,
    \item $\mathcal{N}$(p) - the extended domain of part p,
    \item $\mathcal{B}$(p) - the set of beats in a measure according to the species of part p,
    \item b(p) - the number of beats in a measure according to the species of part p,
    \item d(p) - the duration of a note according to the species of part p,
\end{itemize}
Please note that the constants cannot be linked to a stratum, and only to a part. Indeed, it would have no sense to speak about the species or about the extended domain of a stratum.

The costs are also affected by the change, except for $\mathcal{C}$ (the cost factors) and $\tau$ (the total cost). The latter two remain global and are not duplicated.


To make sure that those notations are clear, here are some examples: the notation $N(a)$ corresponds to the variable representing the notes (pitches) of the lowest stratum, whereas $N(cf)$ are the notes of the \cf. The species of the second counterpoint is written $species(c)$. If only $N$ is written, then the equation in which $N$ is located holds true for any possible part (not necessary stratum). That is, the relationship $N < 60$ would mean: the pitches \textit{of all parts} must be lower than a middle C (whose representation is 60 is Open Music).

\subsection{Added constants}
Here are described some added constants, that are useful throughout the whole work.

\vspace{.5cm} \noindent \textbf{\#parts} \hspace{.2cm} \texttt{*N-PARTS}

This integer describes as many parts there are in a given problem. At this stage, it can either be equal to two (two-part composition) or to three (three-part composition). It is mainly used in the loops of the program as an end condition (\texttt{(dotimes (i *N-PARTS))})




\subsection{Added variables}
\vspace{.5cm} \noindent \textbf{A} \hspace{.cm} \texttt{is-lowest} \label{is-lowest}
% expliquer en EN que on peut pas lambda(cp1) ET lambda(cf)

A is an array of boolean variables with a size of $m$, where each variable indicates whether the corresponding part is the lowest stratum. In other words, $A(v)$ is true if v is the lowest stratum. The notation "A" was chosen as the uppercase of "a", which itself represents the lowest stratum. 
It is also worth to be noted that only one of the parts can be the lowest stratum at the time. This does not mean that two parts cannot equal the lowest stratum at the same time, it \textit{is} indeed possible that two parts blend in unison in the final chord, and that both pitches are the lowest sounding notes. It means that only one of those two is going to be considered to \textit{be} the lowest stratum (and the other one will be the intermediate stratum). This is needed in order for motions to work well. See \ref{subsubsection:one-part-per-stratum} for the details.

Here is the mathematical definition of the A array:
\begin{equation}
\begin{aligned}
\forall j \in [0, m-1)& \colon  \\
A(cf)[j] &= \,  
\begin{cases}
    \top & \text{if } N(cf)[0,j] = N(a)[0,j] \\
    \bot & \text{else }
\end{cases}\\
A(cp_1)[j] &= \,  
\begin{cases}
    \top & \text{if } (N(cp_1)[0,j] = N(a)[0,j]) \land \neg A(cf)[j] \\
    \bot & \text{else }
\end{cases}\\
A(cp_2)[j] &= \,  
\begin{cases}
    \top & \text{if } \neg A(cf) \land \neg A(cp_1)\\
    \bot & \text{else }
\end{cases}
\end{aligned}
\end{equation}

As can be seen in these equations, only the downbeat of each measure is taken into account when computing the A array. The reason for this is that it is the downbeat note that determines which chord will be \textit{the} chord of the measure, and the other beats are just fioritures. Another reason for this is also that it is only going to serve in contexts where the first note of the measure is relevant.

\paragraph{}
In practice, there is only an \texttt{is-not-bass} array in the code (which is then equal to $\neg A$), as it is almost always more useful to know if a part is \textit{not} the lowest stratum than knowing if it is the lowest one. 

\subsection{Modified constants} \label{subsection:modified_constants}
\noindent \textbf{species} \hspace{.2cm} \texttt{species} 

The species constant, which represents the species of a given part, can now take a value between zero and five (inclusive). The value zero is specific to the \cfs (which can be understood as a simplified first species counterpoint). This constant is more useful in the code than in the mathematical notations.
\begin{equation}
species(v) = 0 \iff v = cf
\end{equation}

\subsection{Modified variables} \label{subsection:modified_variables}
Given that the majority of the rules now apply between parts and the lowest stratum, the meaning of the variables has been modified to adapt to this reality. 

\vspace{.5cm} \noindent \textbf{N}(v) \hspace{.2cm} \texttt{notes} 

We changed the name of former \texttt{cp} array and renamed it to \texttt{N} (for notes), for the sake of clarity. As we have now three of those arrays (one for the first counterpoint, one for the second counterpoint, and even one for the \cf), it needed a less ambiguous name than the one it had before.
It still works exactly the same. It is an array of size $s_m$ representing the pitches of a given voice at a given beat in a measure.



\vspace{.5cm} \noindent \textbf{H}$_{(\text{abs})}$(v$_1$, v$_2$) \hspace{.2cm} \texttt{h-intervals}\hspace{.2cm} \texttt{h-intervals-abs}\hspace{.2cm} \texttt{h-intervals-to-cf}\hspace{.2cm}  ...
%todo must be reworked

The way in which this variable works remains much the same, i.e. it represents the interval between one voice and another. This definition has been extended to include intervals other than that between the counterpoint and the \cf. Thus, $H(v1,v2)[i,j]$ represents the intervals between the $i$th beat of voice $v_1$ and the \textit{first} beat of voice $v_2$. $v_1$ may be a part and $v_2$ may be a stratum, as you can calculate harmonic intervals between a part and a stratum. When no $v_2$ is precised, it is equal to $a$, by default. In other words, $H(v_1)$ represents the intervals between the voice $v_1$ and the lowest stratum: $H(v_1) := H(v_1,a)$.

Here is the generalisation explained above, matching to the current definition of the harmonic intervals array:
\begin{equation}
\begin{aligned}
    &\forall v_1, v_2 \in \{cf, cp_1, cp_2\}, \quad \forall i \in \mathcal{B}(v_1), \quad \forall j \in [0, m):\\
    &H_{abs}(v_1,v_2)[i, j] = \left|N(v_1)[i, j] - N(v_2)[0,j]\right|\\
    &H(v_1-v_2)[i, j] = H_{abs}[i, j]\ \text{mod}\ 12\\
    &\text{where } H_{abs}[i, j] \in [0, 127], H[i, j] \in [0, 11]
\end{aligned}
\end{equation}

\vspace{.5cm}
\noindent \textbf{M}$_{\text{brut}}$(s) \hspace*{.2cm} \texttt{m-intervals-brut}

The variable M(p) (i.e. when related to a part) keeps representing the melodic intervals of the voice v and its way of working remains intact. What is described here is its working when related to a stratum, and more specifically, to the lowest stratum. Since strata don't have melodic intervals \textit{per se} (they actually do have melodic intervals, but it doesn't really make sense to consider them), we need to redefine what we mean when speaking about their melodic intervals. Thus, the melodic intervals of the lowest stratum are defined like this: the melodic interval in measure $j$ of the lowest stratum is equal to the last melodic interval in measure $j$ of the part that is the lowest stratum in measure $j+1$. This complex definition is needed in order for the computation of the motions to work fine. The motions of the lowest stratum are an abstract notion that serves only in formulas and constraints and \textit{does not intend to represent any concrete motion really happening in the composition, nor does it correspond to the melodic intervals between the pitches of the lowest stratum}.

\begin{equation}
    \begin{aligned}
        &\forall j \in [0, m-2):\\
        &M_{brut}(a)[j] = \,  
        \begin{cases}
            M_{brut}(cf)[0][j] & \text{if } A(cf)[j+1]\\
            M_{brut}(cp_1)[\text{max}(\mathcal{B}(cp_1))][j] & \text{if } A(cp_1)[j+1]\\
            M_{brut}(cp_2)[\text{max}(\mathcal{B}(cp_1))][j] & \text{if } A(cp_2)[j+1]\\
        \end{cases}
    \end{aligned}
\end{equation}

\noindent It might be helpful to have a look at figures \ref{fig:stratum-m-intervals-1} and \ref{fig:stratum-m-intervals-2} to understand better how the melodic intervals arrays for the lowest stratum.

\vspace{.5cm}
\begin{minipage}{0.46\textwidth}
    \centering
    \includegraphics[width=\textwidth]{Images/stratum-m-intervals.png}
    \captionof{figure}{Understanding the melodic intervals of the lowest stratum with a first species counterpoint}
    \label{fig:stratum-m-intervals-1}
    \end{minipage}
    \hfill
    \begin{minipage}{0.46\textwidth}
      \centering
      \includegraphics[width=\textwidth]{Images/stratum-m-intervals2.png}
      \captionof{figure}{Understanding the melodic intervals of the lowest stratum with a second species counterpoint}
      \label{fig:stratum-m-intervals-2}
\end{minipage}
\vspace{.5cm}

\section{Exploring the interaction of the parts with the lowest stratum}



\vspace{.5cm}
\noindent \textbf{P}(v) \hspace*{.2cm} \texttt{motions}

The motions array was redefined to compute the motions of a voice v with respect to the lowest stratum, instead of computing them with respect to the \cf. 

Motions are now calculated according to the movements of the lowest stratum. This might be counterintuitive because strata have no  The problem here is that if a part is also the lowest stratum, we end up calculating the motion between a part and itself. This inevitably leads to direct motions being calculated, because a part is always moving in the same direction as itself. To solve this problem, the motions of a part are now equal to -1 when the part is also the lowest stratum (denoted A, see section \ref{is-lowest}). 
\begin{equation}
\begin{aligned}
&\forall v \in \{cf, cp_1, cp_2\}, \quad \forall x \in \{1, 2\}, \quad \forall i \in B, \quad \forall j \in [0, m - 1),\quad x := b - i\\
    motions(v)[i,j]& = \,  
    \begin{cases}
        0 & \land (M_{brut}^{x}(v)[i, j] > 0 > M(a)_{brut}[j]) \vee\, (M_{brut}^{x}(v)[i, j] < 0 < M(a)_{brut}[j]) \\
        1 & M_{brut}^{x}(v)[i, j] = 0  \oplus M(a)_{brut}[j]=0 \\
        2 & (M_{brut}^{x}(v)[i, j] > 0 \land M(a)_{brut}[j] > 0) \vee\, (M_{brut}^{x}(v)[i, j] < 0 \land M(a)_{brut}[j] <0)
    \end{cases} 
    \\
    P(v)[i,j]& = \,  
    \begin{cases}
        -1 & \text{if } A(v)[j] \\
        motions(v)[i,j] & \text{if } \neg A(v)[j]
    \end{cases}
\end{aligned}
\end{equation}
