% Already exists
% Machine Learning
% Basic informations (previous thesis, article of Damien and Gradus)
\chapter*{Related Work}
\addcontentsline{toc}{chapter}{Related Work}
Before getting to the heart of the matter, it is important to underline the existence of work related to the field of computer music. Indeed, there are already some works using approaches relatively similar to that of this paper. It will then quickly describe some examples of work in the field of ML to explain problems related to this approach that can be partially solved with constraint programming. Finally, the main references of this thesis will be given to get an idea of its writing environment.

\section*{About Counterpoint and Constraint Programming}

First, in 1984, \citeauthor{bill1984} wrote \citetitle{bill1984}\parencite{bill1984}. His work describes an expert system, i.e. a sequence of if-then statements representing the rules of \citetitle{IMSLPlatin}. There are several similarities with the present work. Where an expert system represents knowledge through if-then statements\parencite{Expertwiki}, the CP makes it possible to represent this knowledge in a more complete and mathematical way. Moreover, the inference engines of the 80s were not as smart as the algorithm used in the present work. Indeed, his searches are more linear in their choice by running more standard algorithms such as "best-first search".

While his work is a precursor of this thesis, it did not really influence this thesis but rather demonstrates certain issues still present today. For example, \citeauthor{bill1984} uses a penalty list system to represent preferences between several choices. As will be explained later, there is a relatively similar system for representing the cost of certain situations in this formalization. The results of his report were poor given the number of mediocre solutions that the expert system offered and the amount of rules put in place.

Secondly, in \citetitle{ovans1992}, \textcite{ovans1992} attempted the same approach by using the CP to represent the first species of counterpoint (comprising only whole notes). Their goal was that the user could interact with the graph representing the search space. It was therefore a tool where the direction of the search was partly directed by a human. It is an interesting approach but different from that of this paper which consists rather in making the user interact via the preference of certain musical notions. Moreover, the first species of counterpoint is not sufficient to form an opinion of the use of CP in musical creation. However, it is quite interesting to see that their conclusions have great similarities with those described at the end of this thesis.

The subject is not recent and several readings can be recommended for those wishing to learn more. For example, \textcite{pachet2001} describe musical harmony as constraints. Or, more recently, \textcite{sandred2021} establishes the different researches and constraints satisfaction problem solvers over time.

\section*{About Machine Learning}
In terms of machine learning, most of the research is more recent. As explained in the introduction, several music generation products/models already exist such as OpenAI's MuseNet and Magenta's Music Transformer. These are therefore technologies that are extremely different from the one presented in this paper. These models also aim to generate music but do not necessarily aim to be highly configurable tools by their users. These technologies can be useful in other contexts, which is beyond the scope of this thesis. However, the analysis of \citeauthor{briot2020} must be highlighted because it explains well the issues encountered with this technology:
\begin{quote}
    "[A] direct application of deep learning to generate content rapidly reaches limits as the generated content tends to mimic the training set without exhibiting true creativity. Moreover, deep learning architectures do not offer direct ways for controlling generation (e.g., imposing some tonality or other arbitrary constraints). Furthermore, deep learning architectures alone are autistic automata which generate music autonomously without human user interaction, far from the objective of interactively assisting musicians to compose and refine music."
    \textcite{briot2020}.
\end{quote}

Also, one of the researchers who worked on the Music Transformer\parencite{huang2018music} tried to generate counterpoints by convolution \parencite{huang2019counterpoint}. The results are not very convincing but acceptable. The problem, in this case, is that the model is trained from a fairly limited database which prevents the model from getting out of its comfort zone. All the approaches remain interesting knowing that these technologies are not enemies because they can collaborate to fill the weaknesses of one another.

\section*{About this Thesis}
Finally, this thesis is based on the work of previous years:

\begin{itemize}
    \item \textcite{GiLthesis} presented an interface for using Gecode functions in Lisp called GiL. He tested it with some rhythm-oriented constraints.
    \item \textcite{Melothesis} explored the use of constraint programming in OpenMusic using GiL. He made a tool capable of producing songs with basic harmonic and melodic constraints.
    \item \textcite{Melo2thesis} created a tool capable of combining the strengths of the first two implementations while continuing to develop support for GiL.
\end{itemize}

This thesis is the first to be a "complete" representation of a particular musical style. As explained above, Fux's \citetitle{IMSLPlatin} was chosen as the basis for the work. The original text dates from \citeyear{IMSLPlatin} and is written in Latin. Nobody on the research team speaks this dead language so it's obvious that translations were used instead. Two of them have been particularly used: the first is that of \textcite{GaPFr}, a French translation dating from \citeyear{GaPFr}. It is from this work that the rules have been taken. The second is \textcite{GaPEng}'s English translation dating from \citeyear{GaPEng}. This one is interesting because it includes footnotes to better understand certain ambiguous rules. It is also from this work that most of the quotations are taken.