% \documentclass[12pt]{report}
\documentclass[11pt,table,xcdraw]{report}

% Fonts
% Greek font
% \usepackage[T1]{fontenc} %% LGR encoding is needed for loading the package gfsneohellenic
% \usepackage[default]{gfsneohellenic}

% Sans serif font
% \usepackage[T1]{fontenc}
% \usepackage{sansmathfonts}
% \renewcommand*\familydefault{\sfdefault} %% Only if the base font of the document is to be sans serif
% \usepackage{lmodern}
% \usepackage{fix-cm}

% TeX Gyre Pagella
\usepackage[T1]{fontenc}
\usepackage{tgpagella}
\usepackage[scaled]{beramono}
% \usepackage{palatino}
% \renewcommand*\familydefault{\sfdefault} %% Only if the base font of the document is to be sans serif
\usepackage{fix-cm}

% Reduce space before chapter title
\usepackage{titlesec}
\titleformat{\chapter}[display]
{\normalfont\huge\bfseries}{\chaptertitlename\ \thechapter}{20pt}{\Huge}
\titlespacing*{\chapter}{0pt}{0pt}{20pt}

% Packages
\usepackage{graphicx}
\usepackage{float}
\usepackage{mathalpha}
\usepackage{amsmath}
\usepackage{amsfonts}
\usepackage{amssymb}
\usepackage{enumitem}
\usepackage{hyperref}
\usepackage{multicol}
\usepackage[super]{nth}
\usepackage{subcaption}
\usepackage[export]{adjustbox}
\usepackage{pdfpages}

% TABLES
\usepackage{xcolor}
\usepackage[normalem]{ulem}
\useunder{\uline}{\ul}{}
\definecolor{llgray}{gray}{0.98}

% Listing in Lisp
\usepackage{listings}
\lstset{
    numbers=left,
    numberstyle=\tiny,
    numbersep=9pt,
    language=Lisp,
    % stringstyle=\ttfamily\small,
    stringstyle=\ttfamily\footnotesize,
    basicstyle=\ttfamily\footnotesize,
    % basicstyle=\ttfamily\scriptsize,
    showstringspaces=false,
    breaklines=true,
    frame=single,
    keywordstyle=\color{purple},
    commentstyle=\color{darkgray},
    backgroundcolor=\color{llgray},
    morekeywords = {gil, om, for, if, setq}
}
% Captions package
\usepackage[hypcap=false]{caption}
\DeclareCaptionType{trans}[Transcription][List of transcriptions]
% Bibliography package
\usepackage[
    backend=biber,
    sorting=none
    % style=authoryear-icomp
]{biblatex}

% Others
\newenvironment{Figure}
  {\par\medskip\noindent\minipage{\linewidth}}
  {\endminipage\par\medskip}

% External files
% \usepackage{xr}
% \externaldocument{Sections/sct_1SP}

% New commands
% TODO counter
\newcounter{todocount}
\newcounter{todosecc}
\newcounter{todoimac}
\setcounter{todocount}{0}
\setcounter{todosecc}{0}
\setcounter{todoimac}{0}
\newcommand{\todo}{\stepcounter{todocount}\textbf{\textcolor{magenta}{TODO \arabic{todocount} }}}
\newcommand{\todosec}{\stepcounter{todosecc}\textbf{\textcolor{blue}{TODO-sec \arabic{todosecc} }}}
\newcommand{\todoimage}{\stepcounter{todoimac}\textbf{\textcolor{red}{TODO-fig \arabic{todoimac} }}}
% shortcut for cantus firmus in italic font
\newcommand{\gap}{\textit{Gradus ad Parnassum}}
\newcommand{\cf}{\textit{cantus firmus}}
\newcommand{\cfs}{\textit{cantus firmus }}
\newcommand{\cp}{counterpoint}
\newcommand{\cfdot}{\textit{cantus firmus. }}
\newcommand{\cpdot}{counterpoint.}
\newcommand{\cfcomma}{\textit{cantus firmus, }}
% shortcut for species
\newcommand{\species}[1]{\nth{#1} species}
% shortcut for B in mathcal
\newcommand{\B}{\mathcal{B}}
% shortcut for N in mathcal
\newcommand{\N}{\mathcal{N}}
% shortcut for R in mathcal
\newcommand{\R}{\mathcal{R}}
% \newcommand{\R}{R}
% shortcut for C in mathcal
\newcommand{\C}{\mathcal{C}}
% shortcut for I in mathcal
\newcommand{\I}{\mathcal{I}}
% shortcut for Lisp ID
\newcommand{\lid}[1]{$\lambda$ID = \texttt{#1}}
% shortcut for default value
\newcommand{\df}[1]{\qquad \textit{DFLT: <#1>}}
\newcommand{\dft}[1]{\textit{DFLT: <#1>}}
\newcommand{\dfts}[1]{\textit{<#1>}}
% shortcut for forall values
\newcommand{\forj}{\forall j \in [0, m-1)}
\newcommand{\forn}{\forall i \in \B, \forall j \in [0, m)}
\newcommand{\fornm}{\forall i \in \B, \forall j \in [0, m-1)}
\newcommand{\fornmm}{\forall i \in \B, \forall j \in [0, m-2)}
\newcommand{\forp}{\forall \rho \in positions(m)}
\newcommand{\forpm}{\forall \rho \in positions(m-1)}
\newcommand{\forpmm}{\forall \rho \in positions(m-2)}
% size used for the height of figures
\newcommand{\fhs}{1.1in}
\newcommand{\fh}{1.2in}
\newcommand{\fhl}{1.3in}
\newcommand{\reddot}{\textcolor{red}{\textbullet} }


% auto cite from bib
\DeclareCiteCommand{\citea}
{\boolfalse{citetracker}\boolfalse{pagetracker}\usebibmacro{prenote}}
{\href{\thefield{url}}{here}}
{\multicitedelim}
{\usebibmacro{postnote}}
% add "Score available here." sending to url + reference beside
\newcommand{\listen}[1]{Score available \citea{#1} \parencite{#1}}
% add "Listen here." sending to url + reference beside
\newcommand{\listenyt}[1]{and listen \href{#1}{here} \parencite{EvalYT}.}

% Load Bibliography
\addbibresource{Bibliography/cite.bib}

\title{Formalizing Fux's Theory of Musical Counterpoint Using Constraint Programming}
\author{Thibault Wafflard}
\date{June 2023}

% Document settings
% \usepackage[margin=1.3in]{geometry}
\usepackage{geometry}
\geometry{
    a4paper,
    % total={170mm,257mm},
    left=1.3in,
    right=1.3in,
    top=1in,
    bottom=1in,
}

\begin{document}
%\pagenumbering{roman}
% \maketitle
%\includepdf[pages=1]{FrontPage.pdf}
%\null
%\thispagestyle{empty}
%\addtocounter{page}{-1}
%\newpage
%\newgeometry{left=1.65in,right=1.65in,top=2in,bottom=2in}
%\begin{abstract}
%    \large{
%        This master's thesis presents FuxCP, a tool for computer-aided contrapuntal composition. The objective is to assist composers without programming skills by automating repetitive and time-consuming tasks. The tool is based on constraint programming with Gecode and formalizes musical rules as constraints. Thanks to this approach, the tool provides transparency and control over the generated solutions, allowing composers to shape their desired music. This thesis focuses on formalizing the rules of two-voice counterpoint from Fux's \textit{Gradus ad Parnassum}. The research highlights the advantages of constraint programming over other approaches, as it allows the tool to "understand" the generated music. The thesis covers the formalization of counterpoint species-specific rules as mathematical constraints, the evaluation of the tool compared to Fux, and suggestions for future development. The conclusion emphasizes the importance of a comprehensive set of rules for formalization, the need for additional constraints on melodic development, and the potential for more expert solvers in other musical genres. The findings indicate the potential of constraint programming in enhancing computer-aided composition across various musical styles.
%    }
%\end{abstract}

%\chapter*{Acknowledgements}
%\paragraph{Thanks to Peter Van Roy}, my supervisor, for giving me the opportunity to write this thesis combining two of my passions: computer science and music.

%\paragraph{Thanks to Damien Sprockeels} for his pragmatic feedback, his devotion to the project, and his great contribution to computer music.

%\paragraph{Thanks to Karim Haddad} from IRCAM for proposing the \textit{Gradus ad Parnassum} as a reference book.

%\paragraph{Thanks to Yves Deville} for reading this thesis.

%\paragraph{Thanks to UCLouvain} for allowing me to finish my studies on such an instructive project.

%\paragraph{Thanks to Justine Nagant} for supporting me throughout this long ordeal.

%\paragraph{It is thanks to} the energy and time of all these people that this thesis exceeded my expectations.
%\restoregeometry

%\tableofcontents

\chapter{Defining some concepts and redefining the variables} \label{chapter:defining-some-concepts-and-redifing-the-variables}
\section{Voices, parts and strata}\label{section:parts-and-strata}
Before we start this section, we need to look at some vocabulary to make sure we understand what we are discussing. The most important definition (and distinction) we introduce is the definition of the terms \textit{part} and \textit{stratum}. The need for these definitions arises from the increasing complexity of the rules of counterpoint when it is generalised to three voices. Indeed, the rules are no longer (as we shall see later) concerned solely with the counterpoints and the \cf, but also with new concepts, such as that referred to by Fux as 'the lowest voice'. As the term 'voice' is too generic (it is used in Fux's text to describe notions as different as 'counterpoint', '\cf', voice range and the so called 'lowest voice'), we need to create a precise vocabulary that is different from the word 'voice' to talk about these new concepts. 


With this in mind, let's explain what 'parts' and 'strata' are, and how they relate to the concept of 'voice'.

\subsubsection{Voices} Again, voices are that vague and \textit{general} concept, whereas parts and strata are more precise and \textit{specific} concepts. The concept of 'voice' includes both 'parts' and 'strata'. In other words, each of these two concepts is a type of voice. When we talk about a voice, we could be talking about either a part or a stratum. To make a metaphor out of it, we could say that parts and strata are a type of voice.

Since there are as many parts and layers as there are voices, in a composition with $n$ voices there will also be $n$ parts and $n$ layers.

\subsubsection{Parts}
Parts are an intuitive and concrete concept because each part corresponds to what a particular person sings or what a particular instrument plays. They correspond to a staff (each staff corresponds to a part). The term 'part' is the same as that used by Fux in his \gap. The three parts in a three-part composition are: the \cf, the first \cps and the second \cp. Fux distinguishes them by calling them by the name of their range, i.e. "bass", "tenor", "alto" or "soprano" (obviously you cannot have all four in a three-part composition).

\subsubsection{Strata}
As for the strata, they are defined like this: a stratum delineates discrete layers or levels of pitches at any given moment in the composition. It denotes a vertical alignment of simultaneous notes and organizes them into distinct strata. By definition, the lowest stratum encompasses the lowest sounding notes, the highest stratum comprises the highest sounding notes, and intermediary strata represent pitch levels in between.
This concept is very helpful in identifying and categorising the vertical placement of pitches, creating distinct categories of sound within the overall texture of the counterpoint composition. It provides a way of analysing and understanding the distribution of pitches across different parts, allowing more complex rules to be established: for example, it would now be possible to establish a rule between the notes of the cantus firmus and the highest sounding notes (no matter which part they come from). The full potential of strata lies in harmonic rules, but as we shall see, some melodic rules are also related to it.

\paragraph*{Important note concerning the strata}
Strata are an abstract concept, useful only in the mathematical formalisation of Fux's rules. They are necessary because we need a structure that is able to comprehend the lowest sounding note for each bar. The strata concept is obviously not needed to write counterpoint as a human being, and the aim behind its definition is not to create a new concept for music theory, but to enable us to use a tool in our constraint programming way of conceiving counterpoint composition.

\paragraph{}
\begin{wrapfigure}{r}{0.3\textwidth}
    \centering
    \includegraphics[width=\linewidth]{Images/rainbow-sediment.jpg}
    \caption{Geological strata, for illustration}
    \label{fig:geological-strata}
\end{wrapfigure}

The term stratum was chosen in this context for its visual impact. In geology, a 'stratum' "is a rock layer with a lithology (texture, color, grain size, composition, fossils, etc.) different from the adjacent ones"~\cite{mcnair2023}, see figure~\ref{fig:geological-strata}.
\paragraph{}
When Fux speaks about the lowest stratum, he often uses the word 'bass'. It was deliberately chosen to speak about the 'lowest stratum' instead of the 'bass' (like Fux does), because 'bass' is also the name of a range of voices (on a par with soprano and alto, for example), and there is already enough complexity in all the terminology to add even further ambiguity.

These new terms (parts and strata) are used where the distinction between the concepts is important. Whenever this distinction is not relevant, the more general term 'voice' is used to reduce the complexity of reading. In this case, the 'voice' could refer to both a stratum and a part. And since a picture is worth a thousand words, Figure~\ref{fig:lowest} illustrates the difference between parts (the blue lines) and strata (the red and orange lines). The lowest stratum is shown in its own colour (red) because it is the most meaningful stratum, and it is particularly important in the formalisation.

\begin{figure}[h]
  \centering
  \includegraphics[width=1\textwidth]{Images/strata_example.png}
  \caption{Parts and strata in a three voice composition}
  \label{fig:lowest}
\end{figure}

\paragraph{}
Here is also the mathematical representation for the notes of the lowest stratum (written $N(a)$, see section~\ref{section:changes induced} for the notations):
% todo a, b, c, et A
\begin{equation}
    \forall i \in [0, 3] \quad \forall j \in [0, m-1): N(a)[i,j] = \text{min} (N(cf)[i,j], N(cp_1)[i,j], N(cp_2)[i,j])
\end{equation}

Of the first upper stratum, or medium stratum (written $N(b)$, see section~\ref{section:changes induced} for the notations):
\begin{equation}
    \forall i \in [0, 3] \quad \forall j \in [0, m-1): N(b)[i,j] = \text{med}\footnote{Where $\text{med}(X)$ means the median value of X.} (N(cf)[i,j], N(cp_1)[i,j], N(cp_2)[i,j])
\end{equation}

And of the second upper stratum, or uppermost stratum (written $N(c)$, see section~\ref{section:changes induced} for the notations):
\begin{equation}
    \forall i \in [0, 3] \quad \forall j \in [0, m-1): N(c)[i,j] = \text{max} (N(cf)[i,j], N(cp_1)[i,j], N(cp_2)[i,j])
\end{equation}

\subsubsection{One part per stratum and one stratum per part} \label{subsubsection:one-part-per-stratum}
It is important to note that, for each musical measure, there is a bijection between the individual parts and the corresponding strata. This means that, for any given measure, each stratum uniquely corresponds to a single part, and vice versa. Put differently, if two parts within a measure share the same pitch, they do not constitute the same stratum. Instead, one part corresponds with one stratum, and the other one to a separate stratum.

To illustrate this, consider a scenario in a two-voice composition (see figure~\ref{fig:one-voice-max-can-be-a}), where part 'cf' and part 'cp1' in measure X both have a pitch value of 67 (representing a G). Despite having identical pitches at the same moment, one part is categorised as the lowest stratum, while the other is designated as the uppermost stratum. This distinction becomes crucial for subsequent analysis, especially when calculating aspects like motions.

To know which part gets to be the lowest stratum in such situations, an arbitrary hierarchical rule is implemented. If the ambivalence is between the \cfs and another part, the \cfs is always prioritised and assigned the role of the lowest stratum, over any other part. In the case of a ambivalence between the first \cp and the second \cp, the first \cps is given the status of the lowest stratum. 

\begin{figure}[h]
    \centering
    \includegraphics[width=.5\textwidth]{Images/one-voice-max-can-be-a.png}
    \caption{Establishing which part corresponds to the lowest stratum}
    \label{fig:one-voice-max-can-be-a}
  \end{figure}

\section{Exploring the interaction of the parts with the lowest stratum} \label{exploring-interaction-p-a}

One of the major differences between the composition of two voices (i.e., one \cfs and one counterpoint) and the generalisation to three voices (i.e., one \cfs and two counterpoints) is that the rules no longer necessarily apply between the counterpoints and the \cf, but instead of this are mostly applied \textbf{between the different parts and the lowest stratum}. 

If we go back to the rules for two voices, we see that each of them applied between the single counterpoint and the \cf. For example, when it was stated that each interval must be consonant, this referred to the harmonic interval between the counterpoint and the \cf.
On the other hand, in his second part (where he describes the rules for composing in three voices), Fux explains that the rules are not necessarily to be observed between each of the counterpoints and the \cf, but rather between "each of the voices and the lowest voice" (i.e. the lowest stratum). Again, if we take the example of the need for consonance between the voices, consonance will be required in the intervals between the notes of any voice and those of the lowest voice (whether or not the latter is the \cf).
Fux approaches the concept of lowest stratum without ever stating it clearly, mentioning for example that the lowest voice can change (sometimes the bass is the lowest voice, sometimes the tenor, ...), and that at any given moment the lowest voice should be considered. In other words, Fux says that the rules apply between the parts and the lowest stratum.

In summary, the constraints are as follows:
\begin{itemize}
    \item Most of the constraints apply:
    \begin{itemize}
        \item Between the \cfs and the lowest stratum.
        \item Between the first \cps and the lowest stratum.
        \item Between the second \cps and the lowest stratum.
    \end{itemize}
    \item Some constraints apply:
    \begin{itemize}
        \item Between the \cfs and the first \cp.
        \item Between the \cfs and the second \cp.
        \item Between the first \cps and the second \cp.
        \item Between the three parts altogether (harmonic rules only).
    \end{itemize}

\end{itemize}


\subsubsection{Generalisation of two-voice counterpoint}
One might be tempted to conclude that three-part composition breaks completely with two-part composition, but that would be too hasty a conclusion. Indeed, on closer inspection, the way the rules worked in two-part composition (from counterpoint to \textit{cantus firmus}) is just one particular case of this new vision of things. In two-part composition, too, the rules apply between the parts and the lowest stratum. But of course, since there were only two voices, the lowest stratum was either counterpoint or cantus firmus. This means that when links were established between the upper part and the lowest stratum, links were also established between the counterpoint and the cantus firmus. Considering the rules as being established between the counterpoint and the \cfs was just a simplification of reality, although it was perfectly correct. We were therefore considering a convenient particular case, and not the general case. Please note that when we talk about "applying constraints from voice A to voice B", it is clear that the constraints are bidirectional and that they also apply from voice B to voice A. What is shown here is rather the philosophy behind the application of these constraints, and the reasons why they were imposed.

The particular case happening when composing with two parts is illustrated in figures~\ref{fig:cp2cf-2v} and~\ref{fig:p2l-2v}. As we can see on those pseudo-compositions, it does not change anything to apply the constraints between the counterpoints and the \cfs or between the parts and the lowest stratum.

\vspace{.5cm}
\begin{minipage}{0.46\textwidth}
    \centering
    \includegraphics[width=\textwidth]{Images/cp2cf-2v.png}
    \captionof{figure}{Applying the constraints between the counterpoint and the \cf}
    \label{fig:cp2cf-2v}
    \end{minipage}
    \hfill
    \begin{minipage}{0.46\textwidth}
      \centering
      \includegraphics[width=\textwidth]{Images/p2l-2v.png}
      \captionof{figure}{Applying the constraints between the parts and the lowest stratum~~~~~~~}
      \label{fig:p2l-2v}
\end{minipage}
\vspace{.5cm}

However, when it comes to generalising the composition of counterpoint for three voices, the same simplification is no longer possible. We are now forced to establish our rules between the parts and the lowest stratum, and no longer between the counterpoints and the \cf. In figures~\ref{fig:cp2cf-3v} and~\ref{fig:p2l-3v} it becomes clear that establishing the rules between the counterpoints and the \cfs is really different from applying them between the various parts to the lowest stratum. In these figures, the parts don't intersect and therefore fit perfectly with the strata, so the constraints are always applied to the same counterpoint. This was done for the sake of intelligibility of the graphs, but it is of course possible for the parts to cross and for the "target" of the constraints not always to be the same counterpoint.

\vspace{.5cm}
\begin{minipage}{0.46\textwidth}
    \centering
    \includegraphics[width=\textwidth]{Images/cp2cf-3v.png}
    \captionof{figure}{Wrong approach: applying the constraints between the \cps to the \cf.}
    \label{fig:cp2cf-3v}
    \end{minipage}
    \hfill
    \begin{minipage}{0.46\textwidth}
      \centering
      \includegraphics[width=\textwidth]{Images/p2l-3v.png}
      \captionof{figure}{Correct approach: applying the constraints between the parts to the lowest stratum.}
      \label{fig:p2l-3v}
\end{minipage}
\vspace{.5cm}

It is, of course, possible for the \cfs to be equal to the lowest stratum all along, in which case nothing changes from the perspective we had when composing for two voices. In this particular case, by applying the rules with respect to the \cf, we would find ourselves de facto applying the rules with respect to the lowest stratum (and we would be back to the situation described above, see figures~\ref{fig:cp2cf-3v} and~\ref{fig:p2l-3v}, only that there is now one more part). It is when the \cfs pitches are higher up than those of the counterpoints that considering the lowest stratum consideration becomes necessary.

\paragraph{}
A very important detail, and perhaps the biggest change brought about by this paradigm shift, is the following. Previously, we applied constraints between the counterpoints and the \cf, which guaranteed that the \cfs was taken into account in the constraints. But if we now apply the constraints between the counterpoints and the lowest stratum, there is no longer any guarantee that the \cfs will be linked to the other voices by any constraints, for example if the \cfs is not the lowest stratum. Nevertheless, it is important that the relationship between the \cfs and the lowest stratum is \textit{also} taken into account, not just the relationship between the counterpoints and the lowest stratum. This means that when we apply the constraints to the parts, we also apply them to the \cfs (since the \cfs is a part, like any of the counterpoints), unless explicitly stated otherwise. For example, some rules that only apply to parts don't apply to \cfs, such as the variety cost (see~\ref{rule:variety}).

A second point to bear in mind, and not the least, is that all this does not mean that \textit{all} the rules are established between the parts and the lowest stratum. Certain rules continue to apply between the different parts, regardless of whether they are high, low or intermediate.

\section{(Re-)Definitions of the variables used in the formalisation} \label{section:changes induced}
Many variables are already defined in T. Wafflard's work. In order to generalise his work to three voices, many of these variables are reused. But in order to generalise, some changes had to be made to these variables. As a result, many variables are redefined in relation to T. Wafflard's work. Some other variables had to be added to express realities that emerged with the addition of a third voice. All these (re)definitions are explained in detail in this section.

\subsection{Linking the variables to a voice}
One major change affects all the variables, namely: the variables are linked to a voice. To understand this, let's take an example. In a two-part composition, it was obvious that the harmonic interval array described the intervals between the \cfs and the only counterpoint. It was also obvious that the motions variable described the motions of the single counterpoint. And so it is with all the variables. When writing a three-voice composition, we have many possibilities when we talk about intervals or motions. Intervals between which voices? Movements of which counterpoint? To deal with this, each variable is now related to a voice.

The relationship between a variable and a voice is expressed as a function. $X(v)$ represents the variable $X$ of the voice $v$. The arguments of the function can be either:
\begin{itemize}
    \item $\mathit{cf}$ - for linking the variable to the \cf.
    \item $cp_1$ - for linking the variable to the second \cp.
    \item $cp_2$ - for linking the variable to the third \cp.
    \item $a$ - for linking the variable to the lowest stratum.
    \item $b$ - for linking the variable to the intermediate stratum.
    \item $c$ - for linking the variable to the uppermost stratum.
  \end{itemize}

\noindent For example, $X(\mathit{cf})$ refers to the variable $X$ of the \cf.

\paragraph{}
When a variable is not explicitly linked to a voice, it is implied that the relation expressed for it is true for all \textit{parts}. In other words, if the variable $X$ is written without any precision, it means that we are speaking about the variable X of all parts. Formally, $X \equiv \forall v \in \{\mathit{cf}, cp_1, cp_2\}: X(v)$.

\paragraph{}
\noindent As mentioned before, linking the variables and the voices is something that applies to all variables, namely\footnote{This list contains all the variables used in this thesis and a short description of them. If no formal definition or redefinition is mentioned in this work, it means that the applicable definition is the one given in T. Wafflard's thesis.}:
\begin{itemize}
    \item \textbf{N}(v) - the notes (pitches) of the voice v. This is the same variable as the variable 'cp' in T. Wafflard's thesis (an explanation of the renaming can be found in the corresponding section (section~\ref{subsection:modified_variables})).
    \item \textbf{H}(v$_1$, v$_2$) - the harmonic intervals between voice v$_1$ and voice v$_2$. This variable is particular, as it needs to arguments to be meaningful.
    \item \textbf{M}(v) - the melodic intervals of the voice v, 
    \item \textbf{P}(v) - the motions of the voice v, 
    \item \textbf{IsCfB}(v) - the boolean array representing whether the cantus firmus is lower than the voice v,
    \item \textbf{IsCons}(v) - to the boolean array representing whether the voice v is consonant with the lowest stratum or not.
\end{itemize}

\noindent It also applies to \textit{some} constants, namely:
\begin{itemize}
    \item \textbf{species}(p) - the species of part p,
    \item \textbf{n}(p) - the number of notes in part p,
    \item \textbf{lb}(p) - the lower bound of the range of part p,
    \item \textbf{ub}(p) - the upper bound of the range of part p,
    \item $\mathcal{R}$(p) - the range of part p,
    \item \textbf{borrow}(p) - the borrowing scale of part p,
    \item $\mathcal{N}$(p) - the extended domain of part p,
    \item $\mathcal{B}$(p) - the set of beats\footnote{To make it clearer: for the first species, the only beat in a measure is $\{0\}$, as there is only a note on the first beat. For the second species, the set of beats is $\{0, 2\}$. For the third species, it is: $\{0, 1, 2, 3\}$. For the fourth species: $\{0, 2\}$. And for the fifth species: $\{0, 1, 2, 3\}$.} in a measure according to the species of part p,
    \item b(p) - the number of beats\footnote{Thus, it is always equal to the size of the set $\mathcal{B}$(p).} in a measure according to the species of part p,
    \item d(p) - the duration of a note\footnote{For the first species, it is equal to $1$, as each note is a whole note. For the second species, it is $\dfrac{1}{2}$, for the third, it is $\dfrac{1}{4}$, for the fourth, it is $\dfrac{1}{2}$, and for the fifth, it is $\dfrac{1}{4}$. It is always equal to $\dfrac{1}{b(p)}$.} according to the species of part p.
\end{itemize}
Please note that the constants can only be linked to the parts, never to a stratum. Indeed, it would have no sense to speak about the species of a stratum or about the extended domain of a stratum.

The costs are also affected by the change, except for $\mathcal{C}$ (the cost factors) and $\tau$ (the total cost). The latter two remain global and are not duplicated.

\paragraph{}
To make sure that those notations are clear, here are some examples: the notation $N(a)$ corresponds to the variable representing the notes (pitches) of the lowest stratum, whereas $N(\mathit{cf})$ are the notes of the \cf. The species of the second counterpoint is written $species(cp_2)$. If only $N$ is written, then the equation in which $N$ is located holds true for any possible \textit{part}. That is, the relationship $N[0, 0] < 60$ would mean: the pitch of the first note \textit{of all parts} must be lower than a middle C (whose representation is 60 is Open Music).

\subsubsection{Note regarding the fourth species}\label{nota-bene-4th-species} Let's recall that the fourth species behaves in a particular way compared to the other species. First of all, it is exclusively composed of syncopations. Its notes are half notes, always linked two by two from bar to bar, producing a pitch change in the middle of the measure, on the upbeat. This gives the impression of hearing a whole note that is constantly shifted by two beats, in other words: syncopation.

Concretely, and as Fux explains it, the syncopation means that the beats of the fourth species should be considered as "shifted": its upbeat should be considered as the downbeat, and its downbeat as the upbeat of the previous measure. This means that in the majority of cases, the equations for the fourth species would have to be rewritten, swapping the 0 and 2 indexes (H[2, j] becomes H[0, j] and H[0, j+1] becomes H[2, j]). To avoid duplicating each of the equations (a first equation if it is not of the fourth species and a second equation if it is of the fourth species) and also to avoid equations that are too complex and difficult to read, it was decided that the index swap would be implicit.

Here is an example: $H[0, 0] \in Cons_{h\_triad}$ should be understood as $H[2, 0] \in Cons_{h\_triad}$ if it concerns a fourth-species counterpoint.

\subsection{Added constants}
Here are described some added constants, that are useful throughout the whole work.

\vspace{.5cm} \noindent \textbf{NumberOfParts} \hspace{.2cm} \texttt{*N-PARTS}

This integer describes how many parts there are in a given composition. It can either be equal to two (two-part composition) or to three (three-part composition). It is mainly used in the loops of the program as an end-condition, like in \texttt{(dotimes (i *N-PARTS))}.

\vspace{.5cm} \noindent \textbf{Cons$_{h\_triad}$} \hspace{.2cm} \texttt{H\_TRIAD\_CONS}

Set representing all consonances that belong to the harmonic triad

\subsection{Added variables}
\vspace{.5cm} \noindent \textbf{A} \hspace{.cm} \texttt{is-lowest} \label{is-lowest}
% expliquer en EN que on peut pas lambda(cp1) ET lambda(cf)

A is an array of boolean variables with a size of $m$, where each variable indicates whether the corresponding part is the lowest stratum. In other words, $A(v)$ is true if v is the lowest stratum. The notation "A" was chosen as the uppercase of "a", which itself represents the lowest stratum. 
It is also worth to be noted that only one of the parts can be the lowest stratum at the time. This does not mean that two parts cannot equal the lowest stratum at the same time, it \textit{is} indeed possible that two parts blend in unison in the final chord, and that both pitches are the lowest sounding notes. It means that only one of those two is going to be considered to \textit{be} the lowest stratum (and the other one will be the intermediate stratum). This is needed in order for motions to work well. See~\ref{subsubsection:one-part-per-stratum} for the details.

Here is the mathematical definition of the A array:
\begin{equation}
\begin{aligned}
\forall j \in [0, m-1)& \colon  \\
A(cf)[j] &= \,  
\begin{cases}
    \top & \text{if } N(cf)[0,j] = N(a)[0,j] \\
    \bot & \text{else }
\end{cases}\\
A(cp_1)[j] &= \,  
\begin{cases}
    \top & \text{if } (N(cp_1)[0,j] = N(a)[0,j]) \land \neg A(cf)[j] \\
    \bot & \text{else }
\end{cases}\\
A(cp_2)[j] &= \,  
\begin{cases}
    \top & \text{if } \neg A(cf) \land \neg A(cp_1)\\
    \bot & \text{else }
\end{cases}
\end{aligned}
\end{equation}

As can be seen in these equations, only the downbeat of each measure is taken into account when computing the A array. The reason for this is that it is the downbeat note that determines which chord will be \textit{the} chord of the measure, and the other beats are just fioritures. Another reason for this is also that it is only going to serve in contexts where the first note of the measure is relevant.

\paragraph{}
In practice, there is only an \texttt{is-not-bass} array in the code (which is then equal to $\neg A$), as it is almost always more useful to know if a part is \textit{not} the lowest stratum than knowing if it is the lowest one. 

\subsection{Modified constants} \label{subsection:modified_constants}
\noindent \textbf{species} \hspace{.2cm} \texttt{species} 

The species constant represents the species of a given part: 1 for the first species, 2 for the second, and so on, with each part having its own species constant. This constant can now also take the value zero: this means that we are talking about \cfs (which can be understood as a simplified first species counterpoint). This constant is more useful in the code than in the mathematical notations.
\begin{equation}
species(v) = 0 \iff v = cf
\end{equation}

\subsection{Redefined variables with respect to the definitions in T. Wafflard's thesis} \label{subsection:modified_variables}
Since of the rules now apply between parts and the lowest stratum, the meaning of the variables has been modified to reflect this reality. Throughout this section, when reference is made to the past ("this variable used to be", "this variable keeps the same meaning", ...), it means that reference is made to the previous definition of the variable, which was the one defined in T. Wafflard's work.

\paragraph{Nota bene}
Please take into consideration that all the rules from T. Wafflard's thesis (which can be found in Appendix~\ref{appendix:complete-set-of-rule}) are compatible with the new definitions of the variables, as discussed in~\ref{exploring-interaction-p-a}. 

\vspace{.5cm} \noindent \textbf{N}(v) \hspace{.2cm} \texttt{notes} 

N is the array corresponding the pitches of each voice. Its size is $s_m$. It is the same array as the one named \texttt{cp} in T. Wafflard's thesis, and it got renamed to \texttt{N} (for notes), for the sake of clarity. As we have now three of those arrays (one for the first counterpoint, one for the second counterpoint, and even one for the \cf), it needed a less ambiguous name than the one it had before.



\vspace{.5cm} \noindent \textbf{H}$_{(\text{abs})}$(v$_1$, v$_2$) \hspace{.2cm} \texttt{h-intervals}\hspace{.2cm} \texttt{h-intervals-abs}\hspace{.2cm} \texttt{h-intervals-to-cf}\hspace{.2cm}  ...

This variable is an array of size $s_m$ and it represents the harmonic interval between one voice and another. The previous definition of this array was that it represented the harmonic intervals between a given voice and the \cf. This definition has been extended to include intervals other than that between the counterpoint and the \cf. In order to do so, H now accepts two arguments, and it represents the interval between those two arguments. Thus, $H(v1,v2)[i,j]$ represents the intervals between the $i$th beat of voice $v_1$ and the \textit{first} beat of voice $v_2$. $v_1$ may be a part and $v_2$ may be a stratum, as you can calculate harmonic intervals between a part and a stratum. When no $v_2$ is precised, it is equal to $a$, by default. In other words, $H(v_1)$ represents the intervals between the voice $v_1$ and the lowest stratum: $H(v_1) \equiv H(v_1,a)$. This default value for $v_2$ was chosen since it is the most frequently used, and for a good reason: most relevant harmonic intervals are those between the parts and the lowest stratum.

Here is the generalisation explained above, matching to the current definition of the harmonic intervals array:
\begin{equation}
\begin{aligned}
    &\forall v_1, v_2 \in \{cf, cp_1, cp_2, a, b, c\}, \quad \forall i \in \mathcal{B}(v_1), \quad \forall j \in [0, m):\\
    &H_{abs}(v_1,v_2)[i, j] = \left|N(v_1)[i, j] - N(v_2)[0,j]\right|\\
    &H(v_1-v_2)[i, j] = H_{abs}[i, j]\ \text{mod}\ 12\\
    &\text{where } H_{abs}[i, j] \in [0, 127], H[i, j] \in [0, 11]
\end{aligned}
\end{equation}

\vspace{.5cm}
\noindent \textbf{M}$_{\text{brut}}$(v) \hspace*{.2cm} \texttt{m-intervals-brut}

The variable M represents the melodic intervals of a voice. It can either be evaluated on a part or on a stratum, each of those situations leading to different behaviours. M(p) (i.e. when related to a part) keeps representing the melodic intervals of the voice v and its way of working remains intact as in T. Wafflard's thesis. M(s) (i.e. when related to a stratum) has it own way of working, that is defined in the next paragraph. We are going to focus specifically on M(a), that is, the melodic intervals of the lowest stratum.

Since strata don't have melodic intervals \textit{per se} (they actually do have melodic intervals, but it doesn't really make sense to consider them), we need to redefine what we mean when speaking about the melodic intervals of a stratum. If it is not clear why strata have no inherent melodic intervals, remember that strata are an abstract concept that is used only in mathematical relationships (and respective constraints). People who listen to the music hear the different parts (be they different tessitura, different instruments, ...) and the way these parts interact together in melodic movements and harmonic convergences, rendering a beautiful music, or not. Strata are an abstraction of the harmonic interactions between the parts, and because of this, they are a consequence of the parts: they exist because the parts exist, and not the other way round! And since they are defined according to harmonic principles (as was suggested before, they are successions of vertical alignments), speaking about the proper \textit{melodic} intervals of a stratum makes no sense. One could then conclude that melodic intervals do not apply to strata, and go ahead. Nevertheless, Fux \textit{does} speak about computing the motions between a part and the lowest stratum. And to be able to compute motions, one needs to compare two different melodic intervals. So we need to have a definition for the melodic intervals of a stratum. 

\paragraph{}
To understand how we arrive at a definition for the melodic intervals of a layer, we need to remember that the lowest layer is just the collection of all the lowest-sounding notes in the composition. It is therefore quite logical to think of the melodic intervals of the lowest layer as the melodic intervals that lead to all those lowest-sounding notes. If the lowest stratum consists of the notes [C$_{cp_1}$, E$_{cf}$, G$_{cp_2}$] (where C$_{cp_1}$ indicates that the C belongs to the first \cp), in the \cfs the interval that lead to the E is a +0 (i.e. staying on the same note), and in the second \cps the interval that lead to the G was a -4 (getting down of two tones), the corresponding melodic intervals array of the lowest stratum would be [+0, -2]. This example has been written again in a more visual way in equation~\ref{eq:defining-m-intervals-bass} to make it easier to understand. To the left of the equation is the pitch array of each voice mentioned. To the right of the equation is the melodic interval array of each voice mentioned. The numbers in bold red are those corresponding to the lowest stratum.


\begin{equation}
    \begin{aligned}        
    N(cf) &= [64,\quad  \textcolor{darkred}{\textbf{64}},\quad  71] \quad 
    &M_{brut}(cf) &= [\textcolor{darkred}{\textbf{+0}}, \quad +7]\\
    N(cp_1) &= [\textcolor{darkred}{\textbf{60}},\quad  67,\quad  74] \quad 
    &M_{brut}(cp_2) &= [+7, \quad +7]\\
    N(cp_2) &= [72,\quad  71,\quad  \textcolor{darkred}{\textbf{67}}] \quad 
    &M_{brut}(cp_2) &= [-1, \quad \textcolor{darkred}{\textbf{-4}}]\\
    \\
    N(a) &= [\textcolor{darkred}{\textbf{60}},\quad  \textcolor{darkred}{\textbf{64}},\quad  \textcolor{darkred}{\textbf{67}}] \quad 
    &M_{brut}(a) &= [\textcolor{darkred}{\textbf{+0}}, \quad \textcolor{darkred}{\textbf{-4}}]\\
\end{aligned}
\label{eq:defining-m-intervals-bass}
\end{equation}


The formal definition of the melodic intervals of the lowest stratum is hence as follows: the melodic interval in measure $j$ of the lowest stratum is equal to the last melodic interval in measure $j$ of the part that is the lowest stratum in measure $j+1$. Remember that this complex definition is needed in order for the computation of the motions to work fine, and that the motions of the lowest stratum are an abstract notion that serves only in formulas and constraints and \textit{does not intend to represent any concrete motion really happening in the composition, nor does it correspond to the melodic intervals between the pitches of the lowest stratum}.

\begin{equation}
    \begin{aligned}
        &\forall j \in [0, m-2):\\
        &M_{brut}(a)[j] = \,  
        \begin{cases}
            M_{brut}(cf)[0][j] & \text{if } A(cf)[j+1]\\
            M_{brut}(cp_1)[\text{max}(\mathcal{B}(cp_1))][j] & \text{if } A(cp_1)[j+1]\\
            M_{brut}(cp_2)[\text{max}(\mathcal{B}(cp_1))][j] & \text{if } A(cp_2)[j+1]\\
        \end{cases}
    \end{aligned}
\end{equation}

\noindent It might be helpful to have a look at figures~\ref{fig:stratum-m-intervals-1} and~\ref{fig:stratum-m-intervals-2} to understand better how the melodic intervals arrays for the lowest stratum.

\vspace{.5cm}
\begin{minipage}{0.46\textwidth}
    \centering
    \includegraphics[width=\textwidth]{Images/stratum-m-intervals.png}
    \captionof{figure}{Understanding the melodic intervals of the lowest stratum with a first species counterpoint}
    \label{fig:stratum-m-intervals-1}
    \end{minipage}
    \hfill
    \begin{minipage}{0.46\textwidth}
      \centering
      \includegraphics[width=\textwidth]{Images/stratum-m-intervals2.png}
      \captionof{figure}{Understanding the melodic intervals of the lowest stratum with a second species counterpoint}
      \label{fig:stratum-m-intervals-2}
\end{minipage}

As can be seen in figures~\ref{fig:stratum-m-intervals-1} and~\ref{fig:stratum-m-intervals-2}, the melodic intervals of the lowest stratum are those that lead to the notes of the lowest stratum.


\vspace{.5cm}
\noindent \textbf{P}(p) \hspace*{.2cm} \texttt{motions}

The motions array represents the motions\footnote{Reminder: there are three types of motion: direct, when both voices move together, contrary, when one voice moves up and the other moves down, and oblique, when one voice doesn't move and the other does} of a voice v with respect to the lowest stratum. The change from the previous work (where the motions array represented the motions with respect to the \cf) was made since Fux considers that the motions should be considered between each voice and the lowest voice. Of course, to be able to compute the motions between two voices, we must compare their melodic intervals, hence, we must deal with melodic intervals of a stratum. This is not a problem anymore since we have defined what the melodic intervals of the lowest stratum mean in the previous sub-section.
However, a problem arises when computing the motions of the part that is also the lowest stratum in some measures. When this happens, we end up calculating motions between a part and itself. Any part is inevitably moving in direct motion with itself, and this situation leads to only direct motions being calculated. This becomes problematic when considering costs (it is bad to have direct motions, but it obviously should not be bad to be the lowest stratum), and when considering some constraints. To tackle this problem, the motions of a part are now equal to -1 when the part is also the lowest stratum (which is denoted A(p), see section~\ref{is-lowest}). 

\begin{equation}
\begin{aligned}
&\forall p \in \{cf, cp_1, cp_2\}, \quad \forall x \in \{1, 2\}, \quad \forall i \in B, \quad \forall j \in [0, m - 1),\quad x := b - i\\
    &motion(p)[i,j] = \,  
    \begin{cases}
        0 &\text{if } (M_{brut}^{x}(p)[i, j] > 0 > M(a)_{brut}[j]) \\ & \quad \quad \quad \quad \quad \quad \quad \quad \quad  \vee (M_{brut}^{x}(p)[i, j] < 0 < M(a)_{brut}[j]) \\
        &\\
        1 &\text{if } M_{brut}^{x}(p)[i, j] = 0  \oplus M(a)_{brut}[j]=0 \\
        &\\
        2 &\text{if } (M_{brut}^{x}(p)[i, j] > 0 \land M(a)_{brut}[j] > 0) \\ & \quad \quad \quad \quad \quad \quad \quad \quad \quad   \vee  (M_{brut}^{x}(p)[i, j] < 0 \land M(a)_{brut}[j] <0)\\
        &\quad \quad \quad \quad \quad \quad \quad \quad \quad \vee (M_{brut}^{x}(p)[i, j] = 0 = M(a)_{brut}[j])
    \end{cases} 
    \\
    &P(p)[i,j] = \,  
    \begin{cases}
        -1 & \text{if } A(p)[j] \\
        motion(p)[i,j] & \text{if } \neg A(p)[j]
    \end{cases}
\end{aligned}
\label{eq:motions}
\end{equation}

This equation~\ref{eq:motions} may seem daunting, but it's actually very simple (just a little verbose).
It works like this: 

For each beat in the composition:
\begin{itemize}
    \item If a part is also the lowest stratum, P is -1 (i.e. non applicable, otherwise we would calculate the motion between the part and itself)
    \item If the part moves in the opposite direction to the lowest stratum, P is 0.
    \item If the part stays where it is and the lowest stratum moves (or vice versa), P is 1.
    \item If the part moves in the same direction as the lowest stratum, P is 2.
\end{itemize}
\chapter{Formalizing Fux rules into English}
This section will be all about extracting all the rules Fux mentions in his work and making sure they are unambiguous.
It will consist of six subsections: one for each species plus one for all species, because even though Fux doesn't clearly mentions rules for all species, some of his rules for the first species are more than certainly meant to be observed for all of them. 

\section{First species}
All rules described in this subsection marked with a red dot (\reddot) apply not only to the first species but to all of them. When described in this section applies to all species, it means that it applies to the first beat of every species, unless mentioned otherwise, and excepted for the fourth species, where it applies to the the beat.

\subsection{This species consists of three whole notes in each instance (p.71)}
This pretty straightforward rule is the very definition of the firsts species. It adds nothing in comparison with the rules for the two part comparison. It is hence already implemented by the first species for two voices and does not need any consideration. 

\subsection{\reddot This species consists of three notes, the upper two being consonant with the lowest (p.71)} \label{rule:consonant}
This rule is a new one, and it implicitly overrides the rule 1.H1 (from T. Wafflard) saying that \textit{all} intervals must be consonants. Here Fux states that only the upper voices and the lowest one are.

This rule was already enforced in the two parts composition, but as discussed in section \ref{section:parts-and-strata}, the big change here is that Fux mentions that it should not be applied between the counterpoints and the \textit{cantus firmus}, but between the voices and the lowest one. Therefore, we are changing the constraint %todo write nb of constraint
a little bit to adapt to this new rule:

\subsection {\reddot The harmonic triad should be employed in every measure if there is no special reason against it (p.71)}
As the footnote on page 71 states it, Fux refers to the "harmonic triad" as being a chord in this position: 1-3-5 (contrary to what is today understood as a harmonic triad).
The rule says it is not obligated, but it is preferred, to use the 1-3-5 chord, considering that 1 is the lowest voice. As this is a preference and not an absolute rule, it has been implemented as a cost. If harmonic triad is used, then the cost is 0. Else, it is 1.
% todo should not be 1 absolutely, should depend on the user
\begin{equation}
\begin{aligned}
\forall j \in [0, m-1) \colon &\\
(\neg (H_{u_1}[0, j] = 3 \lor H_{u_1}[0, j] = 4) &\lor \neg (H_{u_2}[0, j] = 7)) \\
\iff cost_{prefer-harmonic-triad}&[j] = 1
\end{aligned}
\end{equation}

This rule is considered to be true also for the other species.

\subsection{\reddot Occasionally, one uses a consonance not properly belonging to the triad, namely, a sixth or an octave (p.72)}
Here, Fux explains that when it is not possible to have a harmonic triad, you can use sixths or octaves instead. Remember that the sixths or the octaves are calculated from the lowest stratum. Since the rule \ref{rule:consonant} obligates the use of a perfect consonance (i.e. a third, a fifth, a sixth or an octave), when the harmonic triad cannot be used, it is already naturally replaced by a third or a sixth, because no other intervals are allowed. It is thus not a new rule but a restatement of rule \ref{rule:consonant}.

\subsection{\reddot The necessity of avoiding  the succession of two perfect consonances [...] (p.72)} \label{rule:succ-p-cons}
Fux here implies that there should be no two successive perfect consonances. He does not specify whether this rule applies to all three parts at once (i.e. if there was a consonance at bar X between part 1 and part 2, there cannot be one between part 2 and 3 at bar X+1), or whether it applies to each pair of parts separately. That said, in his example (Fig. 91 of the English version), we can clearly see that there is perfect consonance in every bar (parts 1-3, then 1-2, then 1-3, then 2-3, then 1-2). From this we can deduce that for each pair of parts it is forbidden for two perfect consonances to follow each other.

\begin{equation} \begin{aligned}
\forall v_1, v_2 \in \{cf, cp_1, cp_2\}, \quad v_1 \neq v_2 \quad \\
\forall j \in [0, m-2) \colon H^{v_1-v_2}[0, j] \in Cons_p \\
\implies H^{v_1-v_2}[0, j+1] \notin Cons_p
\end{aligned} \end{equation}

Which means a harmonic interval and the following one cannot be consonant at the same time.

\textbf{N.B.} This being said, Fux doesn't seem to follow this rule strictly. Maybe it should be converted as a cost.

\subsection{\reddot [Each part] follows the natural order closely (p.73)}
As this is not really clear, Fux later complements his explication by saying the counterpoints should be "moving gracefully, stepwise without any skip". This is clearly a preference, and has already been covered when implementing the first species for two voices. It can thus be ignored in the scope of this thesis.

\subsection{\reddot [Each part] follows the principle of variety (p.73)}
Fux never defines clearly what he means by "principle of variety". Nevertheless, the examples he provides are of a great help as he corrects his student not following the principle, by augmenting the variety of different notes in a single voice. This means, the principle of variety can be understood as having as many different notes in a single voice. As it is not explained either if this has to be true for the whole partition or only for two following notes, it has been chosen as an arbitrary split in two of the debate to use the principle of variety over 4 following notes. This means that the solution is penalized if a note in measure X was already present in measures [X-3, X+3].

\begin{equation} \begin{aligned}
\forall cp \in \{cp_1, cp_2\}, \quad \forall j \in [0, m-1), \quad \forall k \in [j+1, min(j+3, m-1)] :\\ cp[0, j] = cp[0, j+k]\iff cost_{diversity}[j+m*k]= 1
\end{aligned} \end{equation}

\subsection{\reddot To allow enough space for the voices to move toward each other by contrary motion, the upper voices begin distant from the bass (p.75)}
This is not a strict rule but an indication to make easier for the composer to have contrary motions. It is not an obligation, nor a preference, so it was simply added as a heuristic for the solver.

\subsection{\reddot All voices ascend[ing] [is] a progression which can hardly be managed without awkwardness resulting (p.76)}
What Fux says here is that the three parts cannot be moving in the same direction. 
To prohibit this, we just have to look at the motions between the parts and the bass. If one of their motion is contrary, then it is ensured that the three voices are not going in the same direction (as at least one is contrary). Same goes if one motion is oblique. The problem occurs if both motions are direct, as it would mean that the three voices are going in the same direction. So, we have a prohibit this, by constraining that the two motions can't be direct at the same time. 
\begin{equation} \begin{aligned}
 \forall j \in [0, m-2) \colon \neg (M_{cp_1}[0, j] = 2) \lor \neg (M_{cp_2}[0, j] = 2)
\end{aligned} \end{equation}

\subsection{\reddot [Your may reach] a perfect consonance by direct motion [if] there is no other possibility (p.77)}
This is a relaxation of the 1st species for two voices constraint saying that you cannot reach a perfect consonance by direct motion. Because it is sometimes mandatory with three voices to break this rule (as there are no other possibilities), you may derogate from this rule. Since there is no way in constraint programming to implement a rule that must not be obeyed only if possible other 
than by using a cost, the initial constraint (1.P1-W)  was rewritten to use one :

\begin{equation} \begin{aligned}
\forall v \in \{cp_1, cp_2\}, \quad \forall j \in [0, m-1) : H^{v}[0, j+1] \in Cons_{p} \land P^{v}[0, j] = 2 \\
\iff \text{{Cost}}_{\text{{direct\_move\_to\_p\_cons}}}[j] = 8
\end{aligned} \end{equation}

\subsection{\reddot One feels that the degree of perfection and repose which is required of the final chord does not become sufficiently positive with this imperfect consonance [(speaking about a tenth)] (p.77)}
When Fux says this, he takes a tenth as an example, but it here understood that the final chord cannot include a tenth (third + octave), nor an eight-teenth (third + two octaves), etc. Third are accepted though. The rule then becomes: 

\begin{equation} \begin{aligned}
&\forall v \in \{u_1, u_2\} \quad \\
&\forall h \in \{4 + (12 \times k) \mid k \in \mathbb{N} \setminus \{1\}\} \colon \\
&H_{brut}^{v}[0, m-1] \neq h
\end{aligned} \end{equation}


\subsection{\reddot Ascending sixths on the downbeat sound harsh (p.77)}
This rule is pretty straightforward and states that if an interval is a sixth, the next one cannot be one.
\begin{equation} \begin{aligned}
&\forall j \in [0, m-2)   \quad \forall v_1 \in \{cf, cp_1, cp_2\} \quad \forall v_2 \in \{cf, cp_1, cp_2\} \quad (v_1 \neq v_2) \colon\\
&\neg (H^{v_1-v_2}[0, j] \in \{8, 9\} ) \lor \neg (H^{v_1-v_2}[0, j+1] \in \{8, 9\})\\
&\lor \neg (v_1[0,j] < v_1[0,j+1]) \lor \neg (v_2[0,j] < v_2[0,j+1]))
\end{aligned} \end{equation}


\subsection{\reddot Unison is less harmonious than the octave (p.79)}
This rule is already covered by the no-unison constraint in the first species for two voices.

\subsection{\reddot One should not exceed the limits of the five lines without grave necessity (p.79)}
This rule is already covered by the melodic costs in first species for two voices.

\subsection{\reddot Skip of a major sixth is prohibited (p.79)}
Fux then explains that not only are the skips of a major sixth prohibited, but all bigger skips. This rule is already covered by the constraints of the first species for two voices. 

\subsection{\reddot The minor third is not capable of giving a sense of conclusion (p.80)}
Actually, Fux later states that minor modes should not include a third altogether, but that sometimes it is impossible to do without it, so you can use major third in a minor mode.
\begin{equation} \begin{aligned}
\forall v \in \{u_1, u_2\} \colon H^{v}[0, m-1] \neq 3
\end{aligned} \end{equation}


\subsection{\reddot First and last notes have not to be perfect consonances anymore*} \label{rule:last-chord-not-perfect-anymore}
\footnote{As a reminder, an asterisk at the end of a rule means that the rule is implicit.}

Fux doesn't state this in his text, but in many of his examples, we see that now we have 3 voices, not all voices must be perfect consonances to the first one in the first and last measure.

\subsection{\reddot Last note must be composed of the notes of the a harmonic triad*} \label{rule:last-chord-h-triad}
Again, this isn't stated explicitly but we see that all of his examples end with a chord containing exclusively the notes of the harmonic triad.

\begin{equation} \begin{aligned}
\forall v \in \{u_1, u_2\} \colon H^{v}[0, m-1] \in \{0, 3, 4, 7\}
\end{aligned} \end{equation}


\subsection{\reddot The last chord must have the same fundamental as the one of the scale used throughout the composition*}
This rule emanates from an observation of Fux's examples throughout the chapter. He always ends his examples with the same fundamental as the on of the scale.
When the \textit{cantus firmus} is the lowest stratum, this is not a problem, as the \textit{cantus firmi} always end with the fundamental note of the scale. But when not, it has to be imposed by a constraint, or we may end up with surprising results. Since the fundamental of the scale is defined by being the first note of the \textit{cantus firmus}, we impose that the last note of the lowest stratum must be equal to the first one of the \textit{cantus firmus} (taking the modulos into account).


\begin{equation} \begin{aligned}
\lambda[0, m-1] \mod 12 = N^{cf}[0, 0] \mod 12
\end{aligned} \end{equation}



\section{Second species}
\subsection{A half note may, for the sake of the harmonic triad, occasionally make a succession of two parallel fifth acceptable - which can be effected by the skip of a third (p.86)}
Fux didn't speak about prohibiting two parallel (i.e. consecutive) fifths in the second species for two voices. That being said, it is indeed prohibited in three parts composition as you cannot have two successive perfect consonances (see constraint \ref{rule:succ-p-cons}. We thus have to relax this constraint in order to accept two successive consonances, when the two successive fifths flank a third.


\subsection{Ligatures have no place in this species [except] in the final cadence (p.87)}
Fux explains that in some cases, you have no other option than ligaturing the fourth-to-last and the third-to-last notes. The reasons he gives for this are all part of the previous mentioned rules (no successive perfect consonances, no unison, ...).
This is a relaxation of the two-voice constraint and is done quite easily by modifying the constraint that says "no consecutive notes cannot be the same" unisson between two consecutive notes" to except cp[2, m-1] and cp[0, m]. The reason why this has not been implemented as a cost but as a constraint relaxation is because Fux seems not to say that there is any counterpart at ligaturing the fourt-to-last and the third-to-last notes.

\subsection{A major third [may] appear in the last chord. (p.87)}
This is a consequence of now using three voices instead of two. Fux just made explicit here a rule we had already defined (\ref{rule:last-chord-not-perfect-anymore} and \ref{rule:last-chord-h-triad}). It has thus already been implemented in the first species for two voices.

\subsection{The half notes are always concordant with the two whole notes (p.88)}
If Fux meant "consonant" when he wrote "concordant", then this rule is pretty straightforward. The concern here is that he doesn't seem to follow his own rule. Anyway, here is the mathematical notation for this rule:
\begin{equation} \begin{aligned}
    &\forall v_1 \in \{cp_1, cp_2, cf\} \quad \forall v_1 \in \{cp_1, cp_2, cf\}, v_1 \neq v_2 \colon\\
    &species^{v_1} = 2 \iff H^{v_1-v_2} \in Cons
\end{aligned} \end{equation}

\subsection{A whole note may occasionally be used in the next to last measure (p.93)}
This rule about the second species is written in a footnote of the chapter about the third species, but it seems to apply not only to when the second species is used in combination with the third species, but also in other cases. (fig. 134, 173 and 174 of the English version)

\section{Third species}
\subsection{The quarters have to concur with the whole notes of the other voices (p.91)}
When using the word "concur", it is plausible that Fux meant "are put in relation", and not "be consonant", as he said that the quarters \textit{concur} with the \textit{cantus firmus} in two voices composition. It is not a rule \textit{per se}, Fux is only annunciating that some rules are going to be introduced.

\subsection{Take care whenever you cannot use the harmonic triad on the first quarter occurring on the upbeat, to use it on the second or third quarters (p.91)}
This rule is clear, and states that we should use the harmonic triad in the second or third beat when we were not able to use it in the first one. This is not an absolute rule, but either an advice, and thus it was treated as a cost.



\section{Fourth species}
\subsection{The ligature is nothing but a delaying of the note following (p.95)}
In other words, $cp[n]$ should be associated to $cf[n-1]$ when performing calculations. This is already made by the computations of the two voices counterpoints. It is nevertheless useful to be well aware of this fact, since it will induce many special cases in the implementation.

\subsection{If the ligatures were removed for the sake of the harmonic triad - which, hower, would be impossible because of another consideration, the immediate succession of several fifths (p.95)}
By stating the rule that prohibits the succession of fifths (which is a particular case of the rule saying that you cannot have two successive perfect consonances, see \ref{rule:succ-p-cons}) and by saying that this rule \textit{would} have been valid in the other species, Fux is telling us in a roundabout way that it is not valid in this species. he further complements by saying "there is great power in ligatures - the ability to avoid or improve incorrect passages". All this means that successions of fifths are allowed in the 4th species.

We shall then amend rule \ref{rule:succ-p-cons} and rewrite it as:
\begin{equation} \begin{aligned}
    &\forall v_1, v_2 \in \{cf, cp_1, cp_2\}, \quad v_1 \neq v_2 \quad \forall n \in \{0, 1\} \\
    &i_n := \,  
    \begin{cases}
        0 & \text{if } species^{v_n} \neq 4\\
        2 & \text{if } species^{v_n} = 4\\
    \end{cases}\\
    &\forall j \in [0, m-2) \colon H^{v_1-v_2}[i_n, j] = 0 \iff H^{v_1-v_2}[i_n, j+1] = 0
\end{aligned} \end{equation}

\subsection{The fifth is a perfect consonance, the octave a more perfect one, and the unison the most perfect of all; and the more perfect a consonance, the less harmony it has (p.97)}
This is almost covered by the existing costs, as a perfect consonance has a cost of 1, where an imperfect consonance has a cost of 0. This precision in the rule (fifth is better than octave) could be solver by either putting a cost of 2 to octaves and a cost of 1 to fifths, or to put the cost for fifth before the cost for octaves in the lexicographical array of costs.
\subsection{A dissonance that resolves to a fifth is more acceptable than a dissonance that resolves to an octave (p.98)}
This is be covered by the previous cost, as the fifth is preferred to the octave.

\subsection{If I said that the first note of the ligature must always be consonant, that applies only to the instances in which the bass remains on a pedal point, that is, in the same position. In such a case a ligature involving only dissonances is not only correct but even very beautiful (p.98)}
The rule evoked here is changing a previous rule about consonance (4.H1W\footnote{E2.2W means "Equation 2.2 from T. Wafflard's thesis".

S1.3W means "Section 1.3 from T. Wafflard's thesis".}). From now on, if the lowest stratum has a stationary movement, the corresponding note in the fourth species must be a dissonance, instead of a consonance.

\begin{equation}
\begin{aligned}
&\forall j \in [0, m-1):\\
&M^\lambda \neq 0 \iff H[2, j] \in Cons\\
&M^\lambda = 0 \iff H[2, j] \in Dis
\end{aligned}
    \label{eq:arsiscons} 
\end{equation}

\subsection{The tenor takes the place of the bass in the first measure - a thing that not only the tenor may do, but also the alto and even the soprano}
Fux speaks here about our concept of records. The tenor can become the lowest record, just like the alto and the soprano. This is a fundamental concept of the generalization of Fux counterpoint to three voices, and has already been extensively discussed before (see section \ref{section:parts-and-strata}).

\subsection{To avoid hidden fifths, we may leave the first note of the second counterpoint void.}
I am not sure to have understood the hidden fifths. This is the last constraint that needs to be implemented.

\section{Fifth species}
No supplementary rules have been observed by Fux for the fifth species.

However, in order to have variety when composing with two fifth species counterpoints, it was decided to impose a rule that Fux never mentioned: knowing that the fifth species is just all other combined, not more than 50\% of the composition of the parts can be made using the same species for the same measure. This translates as:
\begin{equation}
\begin{aligned}
species^{cp_1} = 5 = species^{cp_2}  \iff \sum_{i=0}^{3} \sum_{j=0}^{m-1} (S^{cp_1}[i,j] = S^{cp_2}[i,j]) < \frac{s_m}{2}
\end{aligned}
\end{equation}

\chapter{Searching for the best existing solution}
Three parts composition brings in way more possibilities than two parts composition. 
But more possibilities also mean an increased computation complexity. The search space has been extended a lot by adding a whole new set of variables, and the time taken for a solution to be found might to be too elevated if one does not think about optimizing the search. In addition to that, adding a third voice to a composition is not bringing many new constraints (which would help discarding some potential solutions faster), but instead comes with many preferences, which in constraint programming, are translated to costs.

In addition, we need to find a way of arranging the costs in a way that comes as close as possible to what Fux was trying to express in his book. There are several ways to do this, which we will go through and discuss.  

This chapter will begin by explaining the search algorithm that is used to find a solution, continue by discussing the different ways of considering costs, and end by analysing the results that these different perspectives produce.

\section{Using Branch-And-Bound as a search algorithm}

To cope with the increased complexity brought about by the three-part composition, it was decided to switch from the Depth First Search algorithm (used in T. Wafflard's thesis) to a more efficient Branch and Bound (BAB). This allows us to handle costs properly and to find faster solutions. Moreover, the BAB algorithm can also produce non-optimal results, which is very valuable since finding the best overall solution can be time-consuming. When starting the search for a solution, it is now possible to ask for the next solution (i.e. a better solution than the one found previously, and if none was found previously, then just any valid solution), or for the best solution. In the latter case, the solver will continue to search until it finds the best solution or until it is stopped, returning a better solution each time it finds one.


\subsection{Heuristics}
When it comes to finding a solution, we obviously need some heuristics to guide the search, as there are so many different possibilities for a three-part composition. 
To know which heuristics we apply, we simply think about the most important variable to fix first. The most important 


\section{Designing the costs to be as faithful as possible to \gap} \label{costs}

Knowing that we are looking for the solution whose cost must be as low as possible, the question arises: how can we calculate the cost in order to best reflect the preferences expressed in \gap?

The way to translate each preference into a corresponding cost has of course been formalised in the previous sections, but that's not the crux of the matter. The question we face here is: what is the best way to combine all these individual costs to get the most accurate result in terms of what Fux is trying to convey?

Three main ways of doing this have been identified: a linear combination between costs, a search that minimises costs by lexicographic order, and a cost ordering that involves the calculation of minima. We will first describe each of these techniques and their respective advantages, and then compare them (and the results they produce).



\subsection{Linear Combination}


The first method of calculating our costs is a linear combination. This is the technique used in T. Wafflard's thesis. More precisely, it uses a linear combination in which all the weights are equal to one.



To be more precise about the method used to calculate the total cost in T. Wafflard's thesis, here is a more detailed explanation: there exists a total cost, $\tau$, which is equal to the sum of all individual costs, $\mathcal{C}$. The next step is to minimise $\tau$. Each $\mathcal{C}_i$ is usually itself a sum of sub-costs. Take, for example, the cost of motions, $\mathcal{C}_{motions} = \sum_j P_{costs}[j] $. This cost is the sum of all sub-costs of the motions (one per motion): by default, a contrary motion has a sub-cost of 0, an oblique motion has a sub-cost of 1 and a direct motion has a sub-cost of 2. These default values can be changed by the user to be set somewhere on a scale that ranges from $0$ to $64m$. For example, the user could set the oblique motion cost to be equal to $0$, and the cost for direct and contrary motion to be equal to $64m$, in order to get a composition filled with as many oblique motions as possible (always in accordance with the basic rules from \gap, i.e. all voices are never going to go in the same direction, see \ref{rule:same-movement}).

As mentioned at the beginning of this subsection, this procedure can be understood as a linear combination with weights of one only. However, since the cost factors are given different values according to the user's choices, this method is actually more like a regular linear combination, except that the weights are not multiplied by the costs once the latter have been set, but the costs are themselves made larger or smaller before the linear combination is calculated.

The linear combination has two major advantages: ease of implementation and high comprehensibility.


However, it has a major drawback: since the total cost $\tau$ we are minimising in a linear combination is the sum of all costs $\mathcal{C}$, the best solution might be a solution where one cost is absolutely huge and all the others are small. This might not be a problem if the outstanding cost is not really relevant, but if it is the cost of not using a harmonic triad, it goes completely against the preferences that Fux conveys in his work, making the solution inappropriate. A representation of this situation can be found in the figure \ref{fig:outstanding-cost}.

\begin{figure}[h]
    \centering
    \includegraphics[width=1\textwidth]{Images/outstanding-costs.png}
    \caption{Example of a situation where a solution with an outstanding cost is preferred to a solution with equivalent low costs when using a linear combination}
    \label{fig:outstanding-cost}
\end{figure}

Another drawback of linear combination is that the result is pretty and unpredictable: changing the value of the cost may or may not make a difference, and you may need to set huge values to see a real effect. For example, if a composer really wants oblique motion, they may be forced to set the cost of the other types of motion to a huge value, or they may not see the difference between the default solution and their personalised solution. This is due to the fact that all the costs are mixed together and form an indistinguishable soup that the solver considers as a whole, and a small increase in the cost of the direct and contrary motions is very likely to be absorbed into this soup without any change being noticed.

These two drawbacks make the linear combination solution for the costs hardly acceptable when it comes to representing the preferences.
We will therefore examine the other two options for adjusting costs.

\subsection {Minimum Comparison}
In order to overcome the problem of outstanding costs that we encountered when considering the linear combination solution, one might consider using some minimums when calculating $\tau$, the total cost. For example, $\tau$ could be the maximum of all costs. By doing this, the solver would try to find a solution where the focus is on the worst cost and try to reduce it before trying to reduce the other costs.

The problem of this method arises when one cost is significately higher than the others, because it was defined like this. Let's get back to our example of the compositor wanting as many oblique motions as possible. They are going to set the cost for direct motions and contrary motions to the highest cost possible, and will start the search. As we've already discussed it, it is not possible to have only oblique motions, as it would contradict the rule stating that not all voices can move in the same direction (\ref{rule:same-movement}). In consequence, there will always exist contrary motions, and since the cost for them was set really high, it would be impossible for the solver to converge on a good solution. This creates a bottleneck effect, in which when the solver reached the best potential value of the worst cost, it cannot go on finding better solutions. 

Furthermore, even when taking into account a less extreme case (for example the default setting), this method requires a normalisation of the costs: there exists $3\times (m-2)$ sub-costs of the variety cost, $3\times (m-1)$ sub-costs of the motions cost, but only $m$ sub-costs for the octave cost. This means that without a normalisation, the motions cost is going to be in average three times bigger than the octave cost, which consequently means the solver will put three times more effort in minimizing the motions cost than the octave cost, which is unfair and unpractical.

from a given solution to a better one. 

\subsection{Lexicographical Order}
The second way of dealing with the costs is to arrange them in an array and then perform a lexicographic minimisation. In other words, the costs would be arranged in order of importance: from most important to least important. The most important cost to minimise would be placed first in this array, and the solver would only try to minimise the other costs if the first cost remained the same or decreased. This method makes a lot of sense when you think about the rules that emanate of \gap. For example, Fux says that perfect consonance can be achieved by direct motion if there is no other possibility. This means that, all other things being equal, we would prefer to achieve perfect consonance by oblique or contrary motion, but that between a bad solution (respecting almost no preferences) in which perfect consonance is not achieved by direct motion, and a good solution (respecting almost all preferences) in which perfect consonance is achieved by direct motion, we would choose the good solution. 

Some costs are also more important than others in absolute terms. For example, when Fux says that an imperfect consonance is preferred to a fifth, which is preferred to an octave. This amounts to lexicographically ranking the cost of using an octave first (because we really don't want octaves), and then the cost of using a fifth (and there is no cost of using an imperfect consonance, since Fux indicates that this is preferable).
\begin{figure}[h]
    \begin{equation}
        \begin{aligned}
            \tau = [\underset{\text{minimise this first}}{\underbrace{\mathcal{C}_\text{octaves}}}, \mathcal{C}_\text{fifths}]
        \end{aligned}
    \end{equation}
    \caption{Array of costs demonstrating the practicality of a lexicographical order solving.}
\end{figure}

A second example, which ties in particularly well with the first, is that Fux tells us that the harmonic triad must be used in every measure unless a rule forbids it. In saying this, he places the preference for the harmonic triad above all other preferences, because the only reason that can prevent the use of a harmonic triad is a fixed constraint (and not a preference). You'll notice that the harmonic triad consists of a fifth (which is a perfect consonant), so Fux is telling us that we'd rather use a fifth in a harmonic triad than an imperfect consonant outside a harmonic triad. The lexicographic order search is the only one that allows this kind of concept to be taken into account, because in a linear combination these two preferences would be mutually "exclusive"\footnote{in the sense that their effects would work against each other.}: the first preference would add a cost where the second preference would not, and the second preference would add a cost where the first would not.

\begin{figure}[h]
    \begin{equation}
        \begin{aligned}
            \tau = [& \underset{\text{\fontsize{7}{11}\selectfont{minimize this first}}}{\underbrace{\mathcal{C}_\text{harmonic\_triad}}}, \underset{\text{\fontsize{7}{11}\selectfont\parbox{4cm}{and start minimizing this only if it is not possible anymore to minimize the harmonic triad cost}}}{\underbrace{\mathcal{C}_\text{octaves}}},\quad  \mathcal{C}_\text{fifths}]
        \end{aligned}
    \end{equation}
    \caption{Array of costs demonstrating the practicality of a lexicographical order solving.}
\end{figure}

And in this way we can keep integrating the different costs until we get a full array $\tau$ with all the costs ordered in a lexicographical way.

Taking into account all costs, as defined by Fux in \gap, this is the order we get $\tau =$
\begin{multicols}{3}
    \begin{enumerate}
        \item $\mathcal{C}_\text{not-cambiata}$
        \item $\mathcal{C}_\text{penult-thesis}$
        \item $\mathcal{C}_\text{off-key}$
        \item $\mathcal{C}_\text{successive-p-cons}$
        \item $\mathcal{C}_\text{no-syncope}$
        \item $\mathcal{C}_\text{harmonic\_triad}$
        \item $\mathcal{C}_\text{harmonic\_triad\_3rd\_species}$
        \item $\mathcal{C}_\text{octaves}$
        \item $\mathcal{C}_\text{fifths}$
        \item $\mathcal{C}_\text{variety}$
        \item $\mathcal{C}_\text{motions}$
        \item $\mathcal{C}_\text{m-degrees}$
        \item $\mathcal{C}_\text{m2-eq-zero}$
        \item $\mathcal{C}_\text{direct-move-to-p-cons}$
    \end{enumerate}
\end{multicols}
    

\subsection*{Comparison between the three types of costs.}


\begin{center}
    \centering
    \captionof{table}{Comparison of Three Methods According to Criteria}
    \label{tab:comparison}
    \begin{tabularx}{\textwidth}{|>{\centering\arraybackslash}p{4cm}|>{\centering\arraybackslash}X|>{\centering\arraybackslash}X|>{\centering\arraybackslash}X|}
        \hline
        \textbf{Criteria} & \textbf{Linear Combination} & \textbf{Capping with minima} & \textbf{Lexicographic Search} \\
        \hline
        Outsanding costs & \cellcolor{red!25}Yes & \cellcolor{green!25}No & \cellcolor{green!25}Only for minor costs \\
        \hline
        One cost might be a bottleneck & \cellcolor{green!25}No & \cellcolor{red!25}Yes & \cellcolor{green!25}No \\
        \hline
        Possibility to ensure a preference of one cost over another  & \cellcolor{red!25}No & \cellcolor{red!25}No & \cellcolor{green!25}Yes \\
        \hline
        Need for hierarchisation of costs & \cellcolor{green!25}No & \cellcolor{green!25}No & \cellcolor{red!25}Yes \\
        \hline
        Need to normalise costs& \cellcolor{green!25}No & \cellcolor{red!25}Yes & \cellcolor{green!25}No \\
        \hline
    \end{tabularx}
\end{center}
blabla
%\addcontentsline{toc}{chapter}{Bibliography}
%\printbibliography
\appendix

% \bibliographystyle{plain}
% \bibliography{Bibliography/cite}

%\includepdf[pages=1]{BackPage.pdf}
\end{document}