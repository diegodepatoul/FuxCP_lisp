% \documentclass[12pt]{report}
\documentclass[11pt,table,xcdraw]{report}

% Fonts
% Greek font
% \usepackage[T1]{fontenc} %% LGR encoding is needed for loading the package gfsneohellenic
% \usepackage[default]{gfsneohellenic}

% Sans serif font
% \usepackage[T1]{fontenc}
% \usepackage{sansmathfonts}
% \renewcommand*\familydefault{\sfdefault} %% Only if the base font of the document is to be sans serif
% \usepackage{lmodern}
% \usepackage{fix-cm}

% TeX Gyre Pagella
\usepackage[T1]{fontenc}
\usepackage{tgpagella}
\usepackage[scaled]{beramono}
% \usepackage{palatino}
% \renewcommand*\familydefault{\sfdefault} %% Only if the base font of the document is to be sans serif
\usepackage{fix-cm}

% Reduce space before chapter title
\usepackage{titlesec}
\titleformat{\chapter}[display]
{\normalfont\huge\bfseries}{\chaptertitlename\ \thechapter}{20pt}{\Huge}
\titlespacing*{\chapter}{0pt}{0pt}{20pt}

% Packages
\usepackage{graphicx}
\usepackage{float}
\usepackage{mathalpha}
\usepackage{amsmath}
\usepackage{amsfonts}
\usepackage{amssymb}
\usepackage{enumitem}
\usepackage{hyperref}
\usepackage{multicol}
\usepackage{multirow}
\usepackage[super]{nth}
\usepackage{subcaption}
\usepackage[export]{adjustbox}
\usepackage{pdfpages}
\usepackage{stackengine}
\usepackage{booktabs}
\usepackage{adjustbox} 
\usepackage{tabularx}
\usepackage{array}
\usepackage{wrapfig}
\usepackage{rotating}
% TABLES
\usepackage{xcolor}
\usepackage[normalem]{ulem}
\useunder{\uline}{\ul}{}
\definecolor{llgray}{gray}{0.98}

% Listing in Lisp
\usepackage{listings}
\lstset{
    numbers=left,
    numberstyle=\tiny,
    numbersep=9pt,
    language=Lisp,
    % stringstyle=\ttfamily\small,
    stringstyle=\ttfamily\footnotesize,
    basicstyle=\ttfamily\footnotesize,
    % basicstyle=\ttfamily\scriptsize,
    showstringspaces=false,
    breaklines=true,
    frame=single,
    keywordstyle=\color{purple},
    commentstyle=\color{darkgray},
    backgroundcolor=\color{llgray},
    morekeywords = {gil, om, for, if, setq}
}
% Captions package
\usepackage[hypcap=false]{caption}
\DeclareCaptionType{trans}[Transcription][List of transcriptions]
% Bibliography package
\usepackage[
    backend=bibtex,
    %backend=biber,
    sorting=none
    % style=authoryear-icomp
]{biblatex}

% Others
\newenvironment{Figure}
  {\par\medskip\noindent\minipage{\linewidth}}
  {\endminipage\par\medskip}

% External files
% \usepackage{xr}
% \externaldocument{Sections/sct_1SP}

% New commands
% TODO counter
\newcounter{todocount}
\newcounter{todosecc}
\newcounter{todoimac}
\setcounter{todocount}{0}
\setcounter{todosecc}{0}
\setcounter{todoimac}{0}
\newcommand{\todo}{\stepcounter{todocount}\textbf{\textcolor{magenta}{TODO \arabic{todocount} }}}
\newcommand{\todosec}{\stepcounter{todosecc}\textbf{\textcolor{blue}{TODO-sec \arabic{todosecc} }}}
\newcommand{\todoimage}{\stepcounter{todoimac}\textbf{\textcolor{red}{TODO-fig \arabic{todoimac} }}}
% shortcut for cantus firmus in italic font
\newcommand{\gap}{\textit{Gradus ad Parnassum}}
\newcommand{\gaps}{\textit{Gradus ad Parnassum }}
\newcommand{\cf}{\textit{cantus firmus}}
\newcommand{\cfs}{\textit{cantus firmus }}
\newcommand{\cp}{counterpoint}
\newcommand{\cps}{counterpoint }
\newcommand{\cfdot}{\textit{cantus firmus. }}
\newcommand{\cpdot}{counterpoint.}
\newcommand{\cfcomma}{\textit{cantus firmus, }}
% shortcut for species
\newcommand{\species}[1]{\nth{#1} species}
% shortcut for B in mathcal
\newcommand{\B}{\mathcal{B}}
% shortcut for N in mathcal
\newcommand{\N}{\mathcal{N}}
% shortcut for R in mathcal
\newcommand{\R}{\mathcal{R}}
% \newcommand{\R}{R}
% shortcut for C in mathcal
\newcommand{\C}{\mathcal{C}}
% shortcut for I in mathcal
\newcommand{\I}{\mathcal{I}}
% shortcut for Lisp ID
\newcommand{\lid}[1]{$\lambda$ID = \texttt{#1}}
% shortcut for default value
\newcommand{\df}[1]{\qquad \textit{DFLT: <#1>}}
\newcommand{\dft}[1]{\textit{DFLT: <#1>}}
\newcommand{\dfts}[1]{\textit{<#1>}}
% shortcut for forall values
\newcommand{\forj}{\forall j \in [0, m-1)}
\newcommand{\forn}{\forall i \in \B, \forall j \in [0, m)}
\newcommand{\fornm}{\forall i \in \B, \forall j \in [0, m-1)}
\newcommand{\fornmm}{\forall i \in \B, \forall j \in [0, m-2)}
\newcommand{\forp}{\forall \rho \in positions(m)}
\newcommand{\forpm}{\forall \rho \in positions(m-1)}
\newcommand{\forpmm}{\forall \rho \in positions(m-2)}
% size used for the height of figures
\newcommand{\fhs}{1.1in}
\newcommand{\fh}{1.2in}
\newcommand{\fhl}{1.3in}
\definecolor{darkred}{RGB}{139, 0, 0}
\newcommand{\reddot}{\textcolor{red}{\textbullet}}
\newcommand{\reddots}{\textcolor{red}{\textbullet} }
\newcommand{\greendot}{\textcolor{green}{\textbullet}}
\newcommand{\greendots}{\textcolor{green}{\textbullet} }


% auto cite from bib
\DeclareCiteCommand{\citea}
{\boolfalse{citetracker}\boolfalse{pagetracker}\usebibmacro{prenote}}
{\href{\thefield{url}}{here}}
{\multicitedelim}
{\usebibmacro{postnote}}
% add "Score available here." sending to url + reference beside
\newcommand{\listen}[1]{Score available \citea{#1} \parencite{#1}}
% add "Listen here." sending to url + reference beside
\newcommand{\listenyt}[1]{and listen \href{#1}{here} \parencite{EvalYT}.}

% Load Bibliography
\addbibresource{Bibliography/cite.bib}

\title{Formalizing Fux's Theory of Musical Counterpoint Using Constraint Programming}
\author{Anton Lamotte}
\date{January 2024}

% Document settings
% \usepackage[margin=1.3in]{geometry}
\usepackage{geometry}
\geometry{
    a4paper,
    % total={170mm,257mm},
    left=1.3in,
    right=1.3in,
    top=1in,
    bottom=1in,
}

\begin{document}
\pagenumbering{roman}
%\maketitle
\includepdf[pages=1]{FrontPage.pdf}
\null
\thispagestyle{empty}
\addtocounter{page}{-1}
\newpage
\newgeometry{left=1.65in,right=1.65in,top=2in,bottom=2in}
\begin{abstract}
    \large{
        This thesis presents a formalisation of three-part musical counterpoint according to the classical theory of Fux. This thesis is an extension of Thibault Wafflard's previous thesis, which formalised two-part counterpoint and implemented it in FuxCP, a software tool designed specifically for composers to help them compose counterpoint without the need for technical expertise. Counterpoint consists of several musical voices that are independent and distinct from each other, yet balanced and beautiful in sound. It consists of a fixed part, the \cf, and one or more parts derived from it. Three-part counterpoint therefore consists of one \cf and two counterpoints. This is much more expressive than two-part counterpoint because of the interactions between the two derived parts.
        FuxCP is implemented in OpenMusic, a musical interface, and uses Gecode, a well-known constraint solver, to automatically generate counterpoints. The implementation is based on Johann Joseph Fux's \gap, a seminal treatise on counterpoint published in 1725, by translating the rules of this treatise into formal logic and implementing them as constraints. In particular, the extension to three voices places special emphasis on the lowest voice, introducing innovative concepts and variables to address this key aspect.
        This work contributes to the research and understanding of automated contrapuntal composition by overcoming the challenge of generalising the interaction between voices. It also addresses preferences, which are treated as optional rules, introducing nuance into the generation of musical solutions and enhancing the overall aesthetic considerations in automated counterpoint composition. Importantly, this work builds seamlessly on T. Wafflard's previous efforts, ensuring full compatibility with his thesis.   }
\end{abstract}

\chapter*{Acknowledgements}
\dots
%\paragraph{Thanks to Zou} for reminding me to write those acknowledgements.
%\paragraph{Thanks to Alex} for her good metaphors. Only the good ones.
\restoregeometry 

\tableofcontents

\chapter{Introduction and context of this work}
\pagenumbering{arabic}
This thesis is a formalisation for three-part counterpoint based on the theory of Johann Joseph Fux, as given in his classic treatise of 1725. It provides a mathematical formalisation of Fux's rules and a computer environment capable of implementing these logical rules in a concrete way to produce Fux-style counterpoint.


This thesis will therefore be divided into several parts: we will first immerse ourselves in \gap, Fux's central work, from which we will meticulously extract the rules laid down by its author. We will briefly discuss these rules to make them unambiguous, and then translate them into formal logic, so that each rule Fux had in mind when writing his work is mathematically recorded. On this basis, we will create a computer implementation using constraint programming. We will then look at how this implementation finds results, discussing the search algorithm and heuristics used. We then discuss the cost techniques used to obtain the best possible results. Finally, we will analyse the musical compositions produced by the tool created.

It is very important to know that this thesis is based on T. Wafflard's thesis "FuxCP: a constraint programming based tool formalizing Fux's musical theory of counterpoint"~\cite{wafflard2023} and article "A Constraint Formalization of Fux's Counterpoint"~\cite{sprockeels2023constraint}. The present work takes up the concepts and definitions of T. Wafflard and could only be understood in its full depth by reading and fully understanding his works as well. This thesis also assumes a basic knowledge of music theory, which can be found in chapter 1 of T. Wafflard's thesis. %A short summary is given in section~\ref{section:thomas-in-a-nutshell}.



\section{A brief history of counterpoint: from Bach to algorithmic generation}
Before delving into the formalities of our study, let's first examine Fux's theory of counterpoint, which forms the basis of the formalisation undertaken in this work. Counterpoint is a compositional technique in which there are several musical lines (or voices) that are independent and distinct from each other, but that are balanced and sound beautiful~\cite{CpSachs}. No voice is dominant over the others, and all are main voices, although some may take a small precedence during part of the composition~\cite{hess2016}.


Counterpoint has been central to the work of many famous composers from different artistic movements, such as Bach in the Baroque era, Mozart in the Classical era and Beethoven in the Romantic era~\cite{kramer1987gradus}. It is still present in some modern music~\cite{altozano2017contrapunto}, and has aroused interest over the centuries with the development of key texts on the subject, such as Schenker's Counterpoint~\cite{schenker1906} or Jeppesen's Analysis~\cite{jeppesen1960}. And while Bach mastered counterpoint to an unprecedented level for his time~\cite{yearsley2002}, the central and foundational work in the teaching of counterpoint belongs to another great Baroque composer: the Austrian Johann Joseph Fux and his treatise \gap. In it, this composer gives a detailed analysis of the writing of two-, three- and four-part counterpoint, all narrated as a conversation between a master and his pupil. \gaps is one of the works that deal with species counterpoint\footnote{There are many other types of counterpoint, such as free counterpoint, dissonant counterpoint, linear counterpoint, ...}, a way of conceiving counterpoint in five different types that could then be combined. It is on this work that this dissertation is based.

\subsection{Software tool for writing species counterpoint}
We have been discussing the longstanding tradition of counterpoint, a musical technique shaped by countless generations of composers. As technology advanced, the idea of automating counterpoint composition emerged. This section takes a look at some of the work that has been done in the field of counterpoint generation.

\paragraph{} An early attempt, by Schottstaedt in 1984~\cite{bill1984}, involved an expert system that is also based on Fux's rules. His approach used over 300 if-else clauses. However his method had obvious limitations compared to what modern constraints are capable of, since if-else clauses are unidirectional, whereas constraints are bidirectional, which ensures better propagation of the constraints. More importantly constraint modelled problems don't just lead to single solutions, they represent sets of potential solutions. This flexibility is a significant improvement over the directional nature of if-else clauses. 
Furthermore, constraint systems offer an advantage in specifying intricate search heuristics. This adaptability and efficiency highlights the stark contrast between the outdated approach of if-else clauses and the modern capabilities of bidirectional constraint systems in the realm of counterpoint composition.

\paragraph{}
In 1997, a genetic programming and symbiosis approach to automatic counterpoint generation was developed by J. Polito et al. This team from Michigan used a genetic approach to optimise counterpoints of the 5th species and make them more attractive~\cite{polito1997musica}. A similar approach was used in 2004 to generate fugues (another musical technique that relies a lot on counterpoint), also using genetic algorithms~\cite{garay2004fugue}. The results are quite promising, and generate more than interesting results, but the end result is still far from being able to provide a complete counterpoint composition.

Many years later, in 2010, G. Aguilera et al., from the University of Malaga, developed an automated method for the generation of first-species counterpoint using probabilistic logic~\cite{Aguilera2010}. Their approach was specifically tailored to compositions in C major, providing a generated counterpoint in response to a given \cf. However, this application evaluates only the harmonic attributes of the composition, ignoring the melodic aspect of the counterpoint.

Two years later, D. Herremans and K. Sörensen developed a way to generate high-quality first-species counterpoint using a variable neighbourhood search algorithm~\cite{Herremans2012}. Their research was limited to first-species counterpoint, but they addressed issues such as preferences (finding the best counterpoint) and user-friendly interface. Once again, their results are more than impressive, but their research is limited to the first two-voice species.

Finally, a research was carried out in 2015 on Fux's counterpoint~\cite{komosinski2015automatic}, with the aim of generating the first species counterpoint using dominance relations, has yielded fairly good results. The search demonstrates the use of this paradigm and its applicability, and is a good starting point for composing counterpoints of other species based on the same concept.

\paragraph{}
If we now focus on applications that have gone as far as the user interface and are now ready to use, we should mention two namesakes, both called 'Counterpointer', which have the merit of offering a functional tool for composing counterpoint.

The first Counterpointer tool~\cite{counterpointer_ms}, which anyone\footnote{Anyone... or almost, as it is a paid tool.} can use to check the validity (or not) of their counterpoint. Its last release was in 2019 as a desktop application, and it works like this: an apprentice composer tries to write a counterpoint, and then submits it to the tool. The tool then decides whether the counterpoint is valid according to the traditional rules of counterpoint\footnote{Not only Fux's rules, but also those of other authors.}. It also provides feedback to help the student composer improve their future counterpoint writing. The tool is not able to write counterpoints automatically, nor is it explicit about how it works, as it is completely closed source and has no accessible report. It is therefore impossible to know the paradigm it uses or the exact rules it follows.

Another attempt at automatic counterpoint writing is the Counterpointer project in 2021, created by a team of students at Brown University as part of a software engineering course~\cite{counterpointer_project}. The project is less accomplished than the aforementioned application, but it has the merit of being able to generate two-voice counterpoints of the first, second and third species. It is an entirely free and open source project. While the results are encouraging, the project has been discontinued as it was a course project and their method of finding a counterpoint seems much less efficient than the efficiency that a constraint solver can achieve. 


\paragraph{}
This brief overview leads us to conclude that there is no satisfactory tool for composing counterpoint in a user-friendly way, with good quality, quickly and with several voices. It is to fill this gap that this research has been carried out. This was the aim of T. Wafflard's thesis and it is therefore natural that this thesis should follow in his footsteps.


\section{Fux's theory of counterpoint for two-, three- and four-part composition}
As with so many other authors who have attempted to automate the writing of counterpoint, it is only natural that this work should be based on Fux's theory. He was one of the first to theorise counterpoint in such a comprehensive way, and although his theory has been extended many times since, it remains a very good foundation.

For Fux, as for many other authors, species counterpoint is governed by many different rules, and it is these rules that interest us in the present work. The rules are based on old concepts that can be traced back to older styles and have been studied and discussed by generations of authors~\cite{crocker1962}. Those concepts include, for example, the notion of opposite motion, or the notion of consonance (which in turn can be either perfect or imperfect). These concepts and their application to counterpoint are particularly interesting because they allow us to consider the composition of counterpoint both in a 'vertical' way, in which we consider the harmony of the notes played together, and in a 'horizontal' way, in which we consider the melodic development of each of the parts individually, which provides the independence of the counterpoints from each other and their melodic beauty.


All the rules defined by Fux can be divided into three categories: melodic rules, harmonic rules and motion rules. We will examine them here to get an initial sense of what they mean, in order to be able to formalise them afterwards.

\subsubsection{Melodic Rules}

Fux explains that there are rules that apply within parts (the horizontal rules). These rules are concerned with the intervals between one note and the next ones: we find, for example, that a melody is more "beautiful"\footnote{Throughout this work we will speak of the "beauty of music". This beauty is highly subjective, and therefore reference will be made to the Fuxian concept of music to define whether a melody is beautiful or not. In other words, music is considered beautiful if it conforms to Fux's rules, and vice versa.} when the intervals between its successive notes are small, when there is no chromatic succession between the notes, when the notes are varied, and so on. These 'horizontal' rules are called 'melodic' rules because they are concerned only the melody of one given voice, and they therefore apply within this voice.


\subsubsection{Harmonic Rules}


If there is a horizontal perspective to counterpoint, there is also, of course, a vertical perspective. This perspective is expressed in a harmonic relationship between the different voices. At each point in the composition, a series of rules apply that concern harmony alone, and puts some constraints on the harmonic intervals that the voices can have between each other. Here are some harmonic rules from Fux, given as an example: the harmonic interval between any voice and the lowest one must be either a third, a fifth, a sixth or an octave; thirds and sixths are preferred to fifths, which in turn are preferred to octaves; and the voices can't use the same note at the same time. These rules apply between the voices.

\subsubsection{Motion Rules}
Finally, there is a third type of rule: the motion rules. These rules are a hybrid of the two discussed above in that they consider not only vertical interaction, i.e. harmony, but also horizontal interaction, i.e. melody. They can therefore be seen as 'diagonal' rules that relate the unique melody of each counterpoint to its respective harmonies. There exists three types of motions: the direct motion, when two voices move together, the oblique motion, when one voice stays idle and another one moves, and the contrary motion, when both voices move in opposite directions. When coupled to harmonic concepts, we get rules such as: contrary motions are preferred to direct motions; there should be no successive fifths or successive octaves between the voices; and a direct motion should not lead to a perfect consonance. As you can see, these rules take into account not just two voices at a given point, but over several measures. They ensure a harmonious interaction between the voices.

\subsubsection{Preferences}\label{subsection:preferences-vs-hard-rules}
This last point is one of the most important in Fux's theory: the preferences. Preferences are hints that Fux gives in \gaps in order to write even better counterpoints. As their name suggests, preferences are optional and not compulsory to follow, as other strict rules would be. However, preferences are crucial to Fux's work because they allow us to distinguish between two valid solutions (those that obey all the strict rules) and decide which is the best, thus allowing the composer to control how the strict theory is applied.

Fux is never clear about whether a rule\footnote{We use the generic term "rule" to refer to both mandatory rules and preferences.} is a preference or a strict rule --- and that's normal, what he conveys is mostly intuition, and human beings are quite capable of understanding whether a rule is a preference or an obligation; Fux probably didn't expect someone to try to formalise his work three centuries later.

These preferences should be respected whenever possible. However, if a preference cannot be respected, the solution is still valid. Here is a good example: Fux indicates that we prefer to have as many different notes as possible in the composition. This is not a strict rule, but a preference. The more variety there is in the composition, the more beautiful and the more preferable it will be.

\subsection{Species counterpoint}\label{section:species-counterpoint}
When discussing species counterpoint, we refer to five categories of counterpoint, each of which represents a distinct concept with its own characteristics. A detailed explanation of these species is given below. First, let's concentrate on how writing counterpoint works. In counterpoint composition, the starting point is a fixed melodic line known as the \cf, which is a fundamental melody composed entirely of whole notes. This melodic line serves as the basis for composing the entire piece of counterpoint. It's important to emphasise that once the composition is complete, the \cf is neither more nor less important than the other voices. It has the same degree of melodic independence as the other voices and acts as a starting point rather than a more important voice in the compositional process.

Let's take a look at the five species:
\begin{enumerate}
    \item \textbf{First species}: Note against note \textendash{} the first species counterpoint is composed entirely of whole notes, and the composition is a sequence of harmonies sounding on the first beat between the counterpoint and the other voices.
    \item \textbf{Second species}: Half notes against whole notes \textendash{} the second species counterpoint is composed entirely of half notes, which introduce dissonant harmonies.
    \item \textbf{Third species}: Quarters against whole notes \textendash{} the third species counterpoint is made up entirely of quarter notes, which allow more different movements and more freedom in the composition.
    \item \textbf{Fourth species}: The ligature \textendash{} the fourth species counterpoint is delayed by two beats, creating syncopation. The notes are all half notes, tied two by two, which creates the effect of having only delayed whole notes.
    \item \textbf{Fifth species}: Florid counterpoint \textendash{} the fifth species counterpoint is a mixture of all the other species and is the richest form of counterpoint. It allows great freedom of composition while respecting the rules of the other types.
\end{enumerate}

These different species can be combined to form a composition with a \cf, a counterpoint of one species and another counterpoint of another species. Nevertheless, Fux seems to prefer writing compositions that are made of a \cf, a counterpoint of the first species and another counterpoint of another species, probably for pedagogical reasons.

\section{Tools and implementation}\label{subsection:tools-and-implementation}
Now that we have a better understanding of the species counterpoint, let's focus on how Fux's rules are implemented in practice.
This subsection discusses the implementation of FuxCP (the tool to automatically generate counterpoints). To do so, we briefly explain how constraint programming works, the tools used by FuxCP, and the actual implementation of it.

\subsection{Constraint Programming}
Constraint Programming (CP) is a programming paradigm used to solve large combinatorial problems, such as planning and scheduling problems. It works by defining constraints between variables that limit the values these variables can potentially take~\cite{rossi2008constraint}. In doing so, the domains of these variables are reduced. R. Barták explains in a very clear way what a constraint really is, as he describes in his "Guide to Constraint Programming"~\cite{bartak1998constraint}:

\begin{quote}
    A constraint is simply a logical relation among several unknowns (or variables), each taking a value in a given domain. A constraint thus restricts the possible values that variables can take, it represents some partial information about the variables of interest. For instance, "the circle is inside the square" relates two objects without precisely specifying their positions, i.e., their coordinates. Now, one may move the square or the circle and he or she is still able to maintain the relation between these two objects. Also, one may want to add other object, say triangle, and introduce another constraint, say "square is to the left of the triangle". From the user (human) point of view, everything remains absolutely transparent.
\end{quote}

The question now is: what is the connection between this problem-solving method and counterpoint composition? Interestingly, music happens to be an eminently suitable application for Constraint Programming. All aspects of a composition can be represented by variables, rules can be established between these variables, and the solver is responsible for finding a valid solution. In fact, in music, it is never just one factor that determines whether the music is beautiful or not, but an interaction of many factors. As humans, it is sometimes difficult to find a good solution (i.e. to compose music that sounds good) because the range of possibilities and the interactions between factors are so numerous. However, exploring a search space in which numerous constraints are defined is something that a constraint solver does very well. The most arduous task then becomes identifying the rules that make music beautiful, and this is the task that many musicologists and composers, like Fux, have set themselves. Once these rules have been defined, it is "simply" a matter of formalising them and passing them to the constraint solver so that it can compose a melody that respects these rules.

What's more, by defining a rigorous way of distinguishing a good composition from a bad one, the solver can even find increasingly beautiful solutions.
\paragraph{}
To make sure that Constraint Programming is a well understood concept, we here review its main concepts:

\paragraph{Constraint propagation}
Each time a constraint is defined, the domain of the affected variables is reduced according to the possibilities left by the constraint. This is called constraint propagation. Let's imagine a variable $A$ whose domain is $\{0, 1, 2, 3\}$ and a variable $B$ whose domain is $\{0, 1, 2\}$. When the constraint $A<B$ is applied, the domain of $A$ is reduced to $\{0, 1\}$, because the new constraint makes it impossible for $A$ to have a value of $2$ or $3$ without violating the constraint.

A constraint solver is an implementation that systematically searches the search space for a solution. A given problem may have many solutions, only one, or none at all.

A solution is found when all variables are fixed, i.e. their domain is reduced to a single value. We know that a problem has no solution when the domain of any variable becomes empty (because this means that no value can be found for that variable without violating a constraint).

\paragraph{Branching}
Obviously, declaring constraints is not enough to magically find a solution. In the example we proposed earlier (with A and B), the single constraint placed on the search space doesn't allow us to determine the values of A or B, and their domains remain composed of more than one value. 
To actually find a solution, the constraint performs a branching. That is, it studies two antagonic possibilities and splits the search space into two subproblems accordingly. More specifically, it chooses a variable and studies the case where that variable is equal to a certain value, and the case where it is not equal to that value. For example, the solver might study the case where $B=0$ (the first branch) and the case where $B \neq 0$ (the second branch). If the solver finds an inconsistency in either case, it knows that the entire branch can be discarded (as it does not lead to a valid solution). Immediately after branching, the solver again performs constraint propagation, since constraint propagation occurs whenever any domain is modified, and consists of adjusting all domains to which the modified domain is linked by a constraint. In our case, after setting $B$ to $0$, the solver propagates all constraints linked to $B$, i.e. the only constraint in our problem (i.e. $A<B$). This affects the domain of $A$, reducing it to an empty domain (because no value in the domain of $A$ is less than $0$). The solver, noticing that one domain is empty, concludes that this branch contains no solution and therefore knows that the only possible branch is the other one, i.e. the one in which we assumed $B\neq 0$. We can therefore safely remove $0$ from the domain of $B$, since this value is not contained in any solution. Repeating the branching process each time it is necessary produces three solutions: $A=0$ and $B=1$, $A=0$ and $B=2$, and finally $A=1$ and $B=2$.

\paragraph{Heuristics}
As with all problems involving searching a space in quest of a solution, it is very useful to have heuristics allowing for an efficient search. A heuristic is a rule or strategy used to make informed decisions about variable assignments and value choices during the solution search. These rules are designed to exploit the characteristics and structures of the problem to improve the chances of finding solutions more quickly. A common heuristic is variable ordering, where the algorithm selects variables to assign values to based on factors such as the size of the domain or the number of associated constraints. Another important heuristic is value ordering, which determines the order in which values are tested for a given variable assignment. By incorporating heuristics, constraint solvers can prioritise the most promising branches of the search tree, effectively reducing the search space and speeding up the identification of feasible solutions. While heuristics speed up the solving process, it's important to strike a balance between exploration and exploitation, as overly aggressive heuristics risk missing potentially valuable solution paths.


To return to our previous example, a value-ordering heuristic might be "branch first on low values of $A$ and high values of $B$, since we know that we are looking for a solution where $A$ is less than $B$, and we can reason that there are more chances of satisfying this constraint when $B$ is large and $A$ is small.

\paragraph{Optimal solutions}
Constraint programming can also be used to find optimal solutions, i.e. solutions that have minimum cost with respect to a cost function that assigns a cost to each solution.

For example, in our simple example, we could define cost as the sum of $A$ and $B$ and want to minimise the cost. This means that we are looking for a valid solution where $A$ and $B$ are as small as possible. In this case, only the first of the three solutions mentioned above is chosen, i.e. $A=0$ and $B=1$, as it is the best possible solution (with a cost of 1).

\paragraph{Branch and Bound}
Branch and Bound is a systematic algorithm used in Constraint Programming to efficiently explore the solution space and find an optimal solution with respect to a given cost or objective function. This technique extends the basic Constraint Programming approach by introducing a mechanism to prune unpromising branches of the search tree, thereby reducing the computational effort required to find the optimal solution. The algorithm starts with the initial problem and iteratively divides it into subproblems, called branches, by making decisions about variable assignments. For each branch, the algorithm evaluates its feasibility and its potential to lead to a better solution. If a branch is deemed infeasible or cannot possibly improve on the current best-known solution, it is pruned from further consideration. This process continues until all branches have been explored, or until the algorithm converges on the optimal solution. Branch and Bound is particularly valuable for large combinatorial problems because it efficiently narrows the search space, allowing the solver to focus on promising regions and accelerating the discovery of optimal solutions in constrained programming scenarios~\cite{morrison2016branch}. 


\subsection{OpenMusic}
OpenMusic is a powerful and innovative visual programming environment, written in CommonLisp, designed specifically for composers, researchers and musicians involved in computer-aided composition~\cite{OpenMusic}. Developed by the Institute for Research and Coordination in Acoustics/Music (IRCAM) in Paris, OpenMusic provides a graphical interface that allows users to create and manipulate musical structures using a variety of predefined modules. This visual programming environment facilitates the representation of complex musical ideas, algorithms and data flows through a user-friendly interface, making it accessible to both novice and experienced composers. 


FuxCP is a library for OpenMusic, which means that OpenMusic is the interface for using FuxCP. Any user wishing to use FuxCP writes their \cfs (FuxCP's input) into OpenMusic and then launches the solution search from within OpenMusic. Specifically, FuxCP retrieves the \cfs from OpenMusic and then defines the constraint problem. When a solution (the counterpoints) is found, it is passed to OpenMusic and the user gets it as an OpenMusic object.

\subsection{GiL and Gecode}
GiL is an interface between OpenMusic and Gecode, which in turn is an open source toolkit for developing constraint-based systems~\cite{Gecode}, which provides a high-level C++ library for efficiently modelling and solving constraint problems. GiL allows to use the Gecode tool within a Lisp environment, thanks to the Common Foreign Function Interface (CFFI)~\cite{CFFI}.

\subsection{Concrete implementation}
To put it all together: FuxCP gets its input (the \cf) from the user interface, which is OpenMusic. It then defines a constraint programming problem in Gecode, using GiL. 

As for the way it defines the problem, here is a little clarification: first, when the \cfs is received, a whole series of constants and variables are defined: for example, the length of the \cf, the arrays representing the pitches of the counterpoints, ...
Then all these variables are constrained according to the constraints defined in the formalisation of Fux's rules (discussed in chapter~\ref{chapter:species}). The constraints are set sequentially: first the constraints on the \cf, then the constraints on the first counterpoint, and finally the constraints on the second counterpoint. A diagram of the code architecture and the integration of FuxCP with other tools can be found in Appendix~\ref{fig:softwarearchitecure}.



\section{Standing on the shoulders of giants: underlying works and editions of \gaps used}
As has been said, this work is the continuation of T. Wafflard's work. However, it also relies heavily on the work of:
\begin{itemize}
    \item \textcite{GiLthesis}, who presented an interface for using Gecode functions in Lisp called "GiL". This interface was then tested with some rhythm-oriented constraints.
    \item \textcite{Melothesis}, who explored the use of constraint programming in OpenMusic using GiL. The tool that was produced in this thesis is capable of producing songs with basic harmonic and melodic constraints.
    \item \textcite{Melo2thesis}, who created a tool capable of combining the strengths of the first two implementations while continuing to develop support for GiL.
\end{itemize}

As with T. Wafflard, the musical reference work chosen is Fux's \gap, because it is a pillar of counterpoint theory and because it is fairly easy to extract rules from it, although Fux is sometimes very vague about his intentions. And as with any book published several centuries ago (1725 in the case of \gap), there are many versions and translations. This is good news, as Fux can sometimes be really unclear about what he means, and having many versions (some annotated, some not) from many people who also had to interpret Fux to translate it is a great treasure, as it helps to clarify Fux's meanings. This work is therefore based on several different editions and translations of the book, although it is mainly based on Alfred Mann's English translation~\cite{GaPEng}. French (both Chevalier's~\cite{GaPFrChevalier} and Denis's~\cite{GaPFrDenis}), German~\cite{GaPDe} and Latin~\cite{GapLa} translations are used when it is necessary to remove an ambiguity or clarify an unclear rule. These translations have been chosen because French is the lingua franca of the team; German is the language of Fux and the environment in which he evolved; and Latin is the original version, so we can hope that it is the most faithful to what he wanted to convey.


%\section{T. Wafflard's thesis in a nutshell}\label{section:thomas-in-a-nutshell}

%In 2023, T. Wafflard proposed a complete formalisation of Fux's two-voice counterpoint~\cite{wafflard2023}. This formalisation takes each of the rules given by Fux concerning two-voice composition and translates them into formal logic. Those formal relations are then translated into constraints and given to a constraint solver. When given a \cfs as an input, the solver applies all the constraints it was given and returns a counterpoint as an output. The following subsection is not intended to be an exhaustive summary of all of T. Wafflard's excellent work, but rather a brief outline of the idea behind it and the procedure followed. This brief overview is given to the reader because many of the concepts in T. Wafflard's work are at the heart of the three-voice generalisation and will be key to understanding the three-voice formalisation.



%\subsection{In practice}
%In practice, to solve this constraint programming problem, the constraints are written in Lisp. Thanks to the Gecode Interface Lisp (GiL)~\cite{GiL}, the constraints are passed to the Gecode constraint solver~\cite{Gecode} to find a solution. To make the constraints work, we need a starting point. The starting point is the \cf (see~\ref{section:species-counterpoint}).
%When given a \cf, the solver defines a set of variables (those mentioned in the formalisation) to which constraints are then applied (the relations from the formalisation) and produces a counterpoint that obeys all the rules that have been defined and whose quality can be given by the cost. As explained in~\ref{Wafflard-variables}, in the case of T. Wafflard's implementation, the total cost is the sum of all the costs, and this cost is minimised by a depth first search algorithm that finds the lowest cost, and then gives the corresponding solution.
 
 
%As for the front-end, all the user sees and interacts with is OpenMusic~\cite{OpenMusic}, an object-oriented visual programming environment for musical composition based on Common Lisp~\cite{commonlisp} developed by the Parisian institute IRCAM (Institute for Acoustic/Music Research and Coordination)~\cite{IRCAM}.
 
%Everything that has been said about T. Wafflard's thesis also applies to this thesis. The present thesis, as well as the implementation it provides, is an extension of T. Wafflard's thesis and is fully compatible with it.

\section{The contributions of this thesis}
The aim of this work is to generalise T. Wafflard's formalisation to three-voice counterpoint, still based on Fux's work, and to create the corresponding implementation. It would be too easy to believe (wrongly) that three-voice counterpoint is nothing more than the combination of two two-voice counterpoints. Following this point of view, we would then calculate a first counterpoint according to the \cf, and then a second counterpoint again according to the \cf, and that's it. Unfortunately, this view is too simplistic and doesn't really capture all the interactions between three voices.
An entire chapter of this thesis is devoted to the unique characteristics that arise from the introduction of a third voice.
Another chapter is dedicated to translating the rules from \gaps into formal logic. A final but not less important section discusses and analysing the impact of costs, and the musicality of the solutions. The following is a more detailed summary of the contributions of this thesis. 
\begin{itemize}
    \item \textbf{Concepts and variables to three-part counterpoint}:
    As we have just mentioned, a three-part composition is much more than a (two+one) part composition. So we define a whole series of concepts to adapt to this reality. The creation of the (lowest, middle and highest) stratum concept is part of this, and is essential for formalising Fux's counterpoint constraints. All of this is discussed in Chapter~\ref{chapter:defining-some-concepts-and-redifing-the-variables}.
    \item \textbf{Mathematical formalisation of three-part counterpoint}: As with the two parts of the formalisation, we rewrote Fux's explanations into unambiguous English and then translated them into logical notation. This formalisation builds on the previous formalisation for two voices, and sometimes (rarely) has to modify it. This formalisation can be found in Chapter~\ref{chapter:species}.
    \item \textbf{Implementation of a working constraint solver for a three-voice composition}: Those logical rules were then implemented as constraints and the solver was adapted to allow a search for two counterpoints. The whole code of this implementation can be found in Appendix~\ref{chapter:whole-code}, and its architecture in the Appendix~\ref{chapter:architecture}.
    \item \textbf{Researching the best way to express Fux's preferences}: Three-part composition introduces so many possibilities for result composition that it is important to rethink the way we think about preferences. These preferences are understood by the solvers as costs (where a preferred solution in Fux's sense has a lower cost to the solver). Therefore, some techniques for managing these preferences are discussed to find out the best way to implement them as costs. This is very important as it allows the solver to produce solutions with high musicality. These techniques are discussed in Chapter~\ref{chapter:search}.
    \item \textbf{Musical analysis of the solutions generated by the solver}: Finding the best solution also means being able to assess the quality of current solutions. For further details, please refer to Chapter~\ref{chapter:musicality}. 
    \item \textbf{User interface for three-point counterpoint that allows a composer to specify how preferences are used in the solver.}: All the new capabilities of the solver and the costing techniques must also be accessible to the user: it is now possible for a user to freely combine any number of species to form a three-part composition, and to set a cost importance order to indicate their preferences to the solver (in addition to the already existing ability to set personalised costs). A guide to its installation and use can be found in Appendix~\ref{chapter:user-guide}.
\end{itemize}


% citer Fux page 71 
\chapter{Defining some concepts and redefining the variables} \label{chapter:defining-some-concepts-and-redifing-the-variables}
\section{Voices, parts and strata}\label{section:parts-and-strata}
Before we start this section, we need to look at some vocabulary to make sure we understand what we are discussing. The most important definition (and distinction) we introduce is the definition of the terms \textit{part} and \textit{stratum}. The need for these definitions arises from the increasing complexity of the rules of counterpoint when it is generalised to three voices. Indeed, the rules are no longer (as we shall see later) concerned solely with the counterpoints and the \cf, but also with new concepts, such as that referred to by Fux as 'the lowest voice'. As the term 'voice' is too generic (it is used in Fux's text to describe notions as different as 'counterpoint', '\cf', voice range and the so called 'lowest voice'), we need to create a precise vocabulary that is different from the word 'voice' to talk about these new concepts. 


With this in mind, let's explain what 'parts' and 'strata' are, and how they relate to the concept of 'voice'.

\subsubsection{Voices} Again, voices are that vague and \textit{general} concept, whereas parts and strata are more precise and \textit{specific} concepts. The concept of 'voice' includes both 'parts' and 'strata'. In other words, each of these two concepts is a type of voice. When we talk about a voice, we could be talking about either a part or a stratum. To make a metaphor out of it, we could say that parts and strata are a type of voice.

Since there are as many parts and layers as there are voices, in a composition with $n$ voices there will also be $n$ parts and $n$ layers.

\subsubsection{Parts}
Parts are an intuitive and concrete concept because each part corresponds to what a particular person sings or what a particular instrument plays. They correspond to a staff (each staff corresponds to a part). The term 'part' is the same as that used by Fux in his \gap. The three parts in a three-part composition are: the \cf, the first \cps and the second \cp. Fux distinguishes them by calling them by the name of their range, i.e. "bass", "tenor", "alto" or "soprano" (obviously you cannot have all four in a three-part composition).

\subsubsection{Strata}
As for the strata, they are defined like this: a stratum delineates discrete layers or levels of pitches at any given moment in the composition. It denotes a vertical alignment of simultaneous notes and organizes them into distinct strata. By definition, the lowest stratum encompasses the lowest sounding notes, the highest stratum comprises the highest sounding notes, and intermediary strata represent pitch levels in between.
This concept is very helpful in identifying and categorising the vertical placement of pitches, creating distinct categories of sound within the overall texture of the counterpoint composition. It provides a way of analysing and understanding the distribution of pitches across different parts, allowing more complex rules to be established: for example, it would now be possible to establish a rule between the notes of the cantus firmus and the highest sounding notes (no matter which part they come from). The full potential of strata lies in harmonic rules, but as we shall see, some melodic rules are also related to it.

\paragraph*{Important note concerning the strata}
Strata are an abstract concept, useful only in the mathematical formalisation of Fux's rules. They are necessary because we need a structure that is able to comprehend the lowest sounding note for each bar. The strata concept is obviously not needed to write counterpoint as a human being, and the aim behind its definition is not to create a new concept for music theory, but to enable us to use a tool in our constraint programming way of conceiving counterpoint composition.

\paragraph{}
\begin{wrapfigure}{r}{0.3\textwidth}
    \centering
    \includegraphics[width=\linewidth]{Images/rainbow-sediment.jpg}
    \caption{Geological strata, for illustration}
    \label{fig:geological-strata}
\end{wrapfigure}

The term stratum was chosen in this context for its visual impact. In geology, a 'stratum' "is a rock layer with a lithology (texture, color, grain size, composition, fossils, etc.) different from the adjacent ones"~\cite{mcnair2023}, see figure~\ref{fig:geological-strata}.
\paragraph{}
When Fux speaks about the lowest stratum, he often uses the word 'bass'. It was deliberately chosen to speak about the 'lowest stratum' instead of the 'bass' (like Fux does), because 'bass' is also the name of a range of voices (on a par with soprano and alto, for example), and there is already enough complexity in all the terminology to add even further ambiguity.

These new terms (parts and strata) are used where the distinction between the concepts is important. Whenever this distinction is not relevant, the more general term 'voice' is used to reduce the complexity of reading. In this case, the 'voice' could refer to both a stratum and a part. And since a picture is worth a thousand words, Figure~\ref{fig:lowest} illustrates the difference between parts (the blue lines) and strata (the red and orange lines). The lowest stratum is shown in its own colour (red) because it is the most meaningful stratum, and it is particularly important in the formalisation.

\begin{figure}[h]
  \centering
  \includegraphics[width=1\textwidth]{Images/strata_example.png}
  \caption{Parts and strata in a three voice composition}
  \label{fig:lowest}
\end{figure}

\paragraph{}
Here is also the mathematical representation for the notes of the lowest stratum (written $N(a)$, see section~\ref{section:changes induced} for the notations):
% todo a, b, c, et A
\begin{equation}
    \forall i \in [0, 3] \quad \forall j \in [0, m-1): N(a)[i,j] = \text{min} (N(cf)[i,j], N(cp_1)[i,j], N(cp_2)[i,j])
\end{equation}

Of the first upper stratum, or medium stratum (written $N(b)$, see section~\ref{section:changes induced} for the notations):
\begin{equation}
    \forall i \in [0, 3] \quad \forall j \in [0, m-1): N(b)[i,j] = \text{med}\footnote{Where $\text{med}(X)$ means the median value of X.} (N(cf)[i,j], N(cp_1)[i,j], N(cp_2)[i,j])
\end{equation}

And of the second upper stratum, or uppermost stratum (written $N(c)$, see section~\ref{section:changes induced} for the notations):
\begin{equation}
    \forall i \in [0, 3] \quad \forall j \in [0, m-1): N(c)[i,j] = \text{max} (N(cf)[i,j], N(cp_1)[i,j], N(cp_2)[i,j])
\end{equation}

\subsubsection{One part per stratum and one stratum per part} \label{subsubsection:one-part-per-stratum}
It is important to note that, for each musical measure, there is a bijection between the individual parts and the corresponding strata. This means that, for any given measure, each stratum uniquely corresponds to a single part, and vice versa. Put differently, if two parts within a measure share the same pitch, they do not constitute the same stratum. Instead, one part corresponds with one stratum, and the other one to a separate stratum.

To illustrate this, consider a scenario in a two-voice composition (see figure~\ref{fig:one-voice-max-can-be-a}), where part 'cf' and part 'cp1' in measure X both have a pitch value of 67 (representing a G). Despite having identical pitches at the same moment, one part is categorised as the lowest stratum, while the other is designated as the uppermost stratum. This distinction becomes crucial for subsequent analysis, especially when calculating aspects like motions.

To know which part gets to be the lowest stratum in such situations, an arbitrary hierarchical rule is implemented. If the ambivalence is between the \cfs and another part, the \cfs is always prioritised and assigned the role of the lowest stratum, over any other part. In the case of a ambivalence between the first \cp and the second \cp, the first \cps is given the status of the lowest stratum. 

\begin{figure}[h]
    \centering
    \includegraphics[width=.5\textwidth]{Images/one-voice-max-can-be-a.png}
    \caption{Establishing which part corresponds to the lowest stratum}
    \label{fig:one-voice-max-can-be-a}
  \end{figure}

\section{Exploring the interaction of the parts with the lowest stratum} \label{exploring-interaction-p-a}

One of the major differences between the composition of two voices (i.e., one \cfs and one counterpoint) and the generalisation to three voices (i.e., one \cfs and two counterpoints) is that the rules no longer necessarily apply between the counterpoints and the \cf, but instead of this are mostly applied \textbf{between the different parts and the lowest stratum}. 

If we go back to the rules for two voices, we see that each of them applied between the single counterpoint and the \cf. For example, when it was stated that each interval must be consonant, this referred to the harmonic interval between the counterpoint and the \cf.
On the other hand, in his second part (where he describes the rules for composing in three voices), Fux explains that the rules are not necessarily to be observed between each of the counterpoints and the \cf, but rather between "each of the voices and the lowest voice" (i.e. the lowest stratum). Again, if we take the example of the need for consonance between the voices, consonance will be required in the intervals between the notes of any voice and those of the lowest voice (whether or not the latter is the \cf).
Fux approaches the concept of lowest stratum without ever stating it clearly, mentioning for example that the lowest voice can change (sometimes the bass is the lowest voice, sometimes the tenor, ...), and that at any given moment the lowest voice should be considered. In other words, Fux says that the rules apply between the parts and the lowest stratum.

In summary, the constraints are as follows:
\begin{itemize}
    \item Most of the constraints apply:
    \begin{itemize}
        \item Between the \cfs and the lowest stratum.
        \item Between the first \cps and the lowest stratum.
        \item Between the second \cps and the lowest stratum.
    \end{itemize}
    \item Some constraints apply:
    \begin{itemize}
        \item Between the \cfs and the first \cp.
        \item Between the \cfs and the second \cp.
        \item Between the first \cps and the second \cp.
        \item Between the three parts altogether (harmonic rules only).
    \end{itemize}

\end{itemize}


\subsubsection{Generalisation of two-voice counterpoint}
One might be tempted to conclude that three-part composition breaks completely with two-part composition, but that would be too hasty a conclusion. Indeed, on closer inspection, the way the rules worked in two-part composition (from counterpoint to \textit{cantus firmus}) is just one particular case of this new vision of things. In two-part composition, too, the rules apply between the parts and the lowest stratum. But of course, since there were only two voices, the lowest stratum was either counterpoint or cantus firmus. This means that when links were established between the upper part and the lowest stratum, links were also established between the counterpoint and the cantus firmus. Considering the rules as being established between the counterpoint and the \cfs was just a simplification of reality, although it was perfectly correct. We were therefore considering a convenient particular case, and not the general case. Please note that when we talk about "applying constraints from voice A to voice B", it is clear that the constraints are bidirectional and that they also apply from voice B to voice A. What is shown here is rather the philosophy behind the application of these constraints, and the reasons why they were imposed.

The particular case happening when composing with two parts is illustrated in figures~\ref{fig:cp2cf-2v} and~\ref{fig:p2l-2v}. As we can see on those pseudo-compositions, it does not change anything to apply the constraints between the counterpoints and the \cfs or between the parts and the lowest stratum.

\vspace{.5cm}
\begin{minipage}{0.46\textwidth}
    \centering
    \includegraphics[width=\textwidth]{Images/cp2cf-2v.png}
    \captionof{figure}{Applying the constraints between the counterpoint and the \cf}
    \label{fig:cp2cf-2v}
    \end{minipage}
    \hfill
    \begin{minipage}{0.46\textwidth}
      \centering
      \includegraphics[width=\textwidth]{Images/p2l-2v.png}
      \captionof{figure}{Applying the constraints between the parts and the lowest stratum~~~~~~~}
      \label{fig:p2l-2v}
\end{minipage}
\vspace{.5cm}

However, when it comes to generalising the composition of counterpoint for three voices, the same simplification is no longer possible. We are now forced to establish our rules between the parts and the lowest stratum, and no longer between the counterpoints and the \cf. In figures~\ref{fig:cp2cf-3v} and~\ref{fig:p2l-3v} it becomes clear that establishing the rules between the counterpoints and the \cfs is really different from applying them between the various parts to the lowest stratum. In these figures, the parts don't intersect and therefore fit perfectly with the strata, so the constraints are always applied to the same counterpoint. This was done for the sake of intelligibility of the graphs, but it is of course possible for the parts to cross and for the "target" of the constraints not always to be the same counterpoint.

\vspace{.5cm}
\begin{minipage}{0.46\textwidth}
    \centering
    \includegraphics[width=\textwidth]{Images/cp2cf-3v.png}
    \captionof{figure}{Wrong approach: applying the constraints between the \cps to the \cf.}
    \label{fig:cp2cf-3v}
    \end{minipage}
    \hfill
    \begin{minipage}{0.46\textwidth}
      \centering
      \includegraphics[width=\textwidth]{Images/p2l-3v.png}
      \captionof{figure}{Correct approach: applying the constraints between the parts to the lowest stratum.}
      \label{fig:p2l-3v}
\end{minipage}
\vspace{.5cm}

It is, of course, possible for the \cfs to be equal to the lowest stratum all along, in which case nothing changes from the perspective we had when composing for two voices. In this particular case, by applying the rules with respect to the \cf, we would find ourselves de facto applying the rules with respect to the lowest stratum (and we would be back to the situation described above, see figures~\ref{fig:cp2cf-3v} and~\ref{fig:p2l-3v}, only that there is now one more part). It is when the \cfs pitches are higher up than those of the counterpoints that considering the lowest stratum consideration becomes necessary.

\paragraph{}
A very important detail, and perhaps the biggest change brought about by this paradigm shift, is the following. Previously, we applied constraints between the counterpoints and the \cf, which guaranteed that the \cfs was taken into account in the constraints. But if we now apply the constraints between the counterpoints and the lowest stratum, there is no longer any guarantee that the \cfs will be linked to the other voices by any constraints, for example if the \cfs is not the lowest stratum. Nevertheless, it is important that the relationship between the \cfs and the lowest stratum is \textit{also} taken into account, not just the relationship between the counterpoints and the lowest stratum. This means that when we apply the constraints to the parts, we also apply them to the \cfs (since the \cfs is a part, like any of the counterpoints), unless explicitly stated otherwise. For example, some rules that only apply to parts don't apply to \cfs, such as the variety cost (see~\ref{rule:variety}).

A second point to bear in mind, and not the least, is that all this does not mean that \textit{all} the rules are established between the parts and the lowest stratum. Certain rules continue to apply between the different parts, regardless of whether they are high, low or intermediate.

\section{(Re-)Definitions of the variables used in the formalisation} \label{section:changes induced}
Many variables are already defined in T. Wafflard's work. In order to generalise his work to three voices, many of these variables are reused. But in order to generalise, some changes had to be made to these variables. As a result, many variables are redefined in relation to T. Wafflard's work. Some other variables had to be added to express realities that emerged with the addition of a third voice. All these (re)definitions are explained in detail in this section.

\subsection{Linking the variables to a voice}
One major change affects all the variables, namely: the variables are linked to a voice. To understand this, let's take an example. In a two-part composition, it was obvious that the harmonic interval array described the intervals between the \cfs and the only counterpoint. It was also obvious that the motions variable described the motions of the single counterpoint. And so it is with all the variables. When writing a three-voice composition, we have many possibilities when we talk about intervals or motions. Intervals between which voices? Movements of which counterpoint? To deal with this, each variable is now related to a voice.

The relationship between a variable and a voice is expressed as a function. $X(v)$ represents the variable $X$ of the voice $v$. The arguments of the function can be either:
\begin{itemize}
    \item $\mathit{cf}$ - for linking the variable to the \cf.
    \item $cp_1$ - for linking the variable to the second \cp.
    \item $cp_2$ - for linking the variable to the third \cp.
    \item $a$ - for linking the variable to the lowest stratum.
    \item $b$ - for linking the variable to the intermediate stratum.
    \item $c$ - for linking the variable to the uppermost stratum.
  \end{itemize}

\noindent For example, $X(\mathit{cf})$ refers to the variable $X$ of the \cf.

\paragraph{}
When a variable is not explicitly linked to a voice, it is implied that the relation expressed for it is true for all \textit{parts}. In other words, if the variable $X$ is written without any precision, it means that we are speaking about the variable X of all parts. Formally, $X \equiv \forall v \in \{\mathit{cf}, cp_1, cp_2\}: X(v)$.

\paragraph{}
\noindent As mentioned before, linking the variables and the voices is something that applies to all variables, namely\footnote{This list contains all the variables used in this thesis and a short description of them. If no formal definition or redefinition is mentioned in this work, it means that the applicable definition is the one given in T. Wafflard's thesis.}:
\begin{itemize}
    \item \textbf{N}(v) - the notes (pitches) of the voice v. This is the same variable as the variable 'cp' in T. Wafflard's thesis (an explanation of the renaming can be found in the corresponding section (section~\ref{subsection:modified_variables})).
    \item \textbf{H}(v$_1$, v$_2$) - the harmonic intervals between voice v$_1$ and voice v$_2$. This variable is particular, as it needs to arguments to be meaningful.
    \item \textbf{M}(v) - the melodic intervals of the voice v, 
    \item \textbf{P}(v) - the motions of the voice v, 
    \item \textbf{IsCfB}(v) - the boolean array representing whether the cantus firmus is lower than the voice v,
    \item \textbf{IsCons}(v) - to the boolean array representing whether the voice v is consonant with the lowest stratum or not.
\end{itemize}

\noindent It also applies to \textit{some} constants, namely:
\begin{itemize}
    \item \textbf{species}(p) - the species of part p,
    \item \textbf{n}(p) - the number of notes in part p,
    \item \textbf{lb}(p) - the lower bound of the range of part p,
    \item \textbf{ub}(p) - the upper bound of the range of part p,
    \item $\mathcal{R}$(p) - the range of part p,
    \item \textbf{borrow}(p) - the borrowing scale of part p,
    \item $\mathcal{N}$(p) - the extended domain of part p,
    \item $\mathcal{B}$(p) - the set of beats\footnote{To make it clearer: for the first species, the only beat in a measure is $\{0\}$, as there is only a note on the first beat. For the second species, the set of beats is $\{0, 2\}$. For the third species, it is: $\{0, 1, 2, 3\}$. For the fourth species: $\{0, 2\}$. And for the fifth species: $\{0, 1, 2, 3\}$.} in a measure according to the species of part p,
    \item b(p) - the number of beats\footnote{Thus, it is always equal to the size of the set $\mathcal{B}$(p).} in a measure according to the species of part p,
    \item d(p) - the duration of a note\footnote{For the first species, it is equal to $1$, as each note is a whole note. For the second species, it is $\dfrac{1}{2}$, for the third, it is $\dfrac{1}{4}$, for the fourth, it is $\dfrac{1}{2}$, and for the fifth, it is $\dfrac{1}{4}$. It is always equal to $\dfrac{1}{b(p)}$.} according to the species of part p.
\end{itemize}
Please note that the constants can only be linked to the parts, never to a stratum. Indeed, it would have no sense to speak about the species of a stratum or about the extended domain of a stratum.

The costs are also affected by the change, except for $\mathcal{C}$ (the cost factors) and $\tau$ (the total cost). The latter two remain global and are not duplicated.

\paragraph{}
To make sure that those notations are clear, here are some examples: the notation $N(a)$ corresponds to the variable representing the notes (pitches) of the lowest stratum, whereas $N(\mathit{cf})$ are the notes of the \cf. The species of the second counterpoint is written $species(cp_2)$. If only $N$ is written, then the equation in which $N$ is located holds true for any possible \textit{part}. That is, the relationship $N[0, 0] < 60$ would mean: the pitch of the first note \textit{of all parts} must be lower than a middle C (whose representation is 60 is Open Music).

\subsubsection{Note regarding the fourth species}\label{nota-bene-4th-species} Let's recall that the fourth species behaves in a particular way compared to the other species. First of all, it is exclusively composed of syncopations. Its notes are half notes, always linked two by two from bar to bar, producing a pitch change in the middle of the measure, on the upbeat. This gives the impression of hearing a whole note that is constantly shifted by two beats, in other words: syncopation.

Concretely, and as Fux explains it, the syncopation means that the beats of the fourth species should be considered as "shifted": its upbeat should be considered as the downbeat, and its downbeat as the upbeat of the previous measure. This means that in the majority of cases, the equations for the fourth species would have to be rewritten, swapping the 0 and 2 indexes (H[2, j] becomes H[0, j] and H[0, j+1] becomes H[2, j]). To avoid duplicating each of the equations (a first equation if it is not of the fourth species and a second equation if it is of the fourth species) and also to avoid equations that are too complex and difficult to read, it was decided that the index swap would be implicit.

Here is an example: $H[0, 0] \in Cons_{h\_triad}$ should be understood as $H[2, 0] \in Cons_{h\_triad}$ if it concerns a fourth-species counterpoint.

\subsection{Added constants}
Here are described some added constants, that are useful throughout the whole work.

\vspace{.5cm} \noindent \textbf{NumberOfParts} \hspace{.2cm} \texttt{*N-PARTS}

This integer describes how many parts there are in a given composition. It can either be equal to two (two-part composition) or to three (three-part composition). It is mainly used in the loops of the program as an end-condition, like in \texttt{(dotimes (i *N-PARTS))}.

\vspace{.5cm} \noindent \textbf{Cons$_{h\_triad}$} \hspace{.2cm} \texttt{H\_TRIAD\_CONS}

Set representing all consonances that belong to the harmonic triad

\subsection{Added variables}
\vspace{.5cm} \noindent \textbf{A} \hspace{.cm} \texttt{is-lowest} \label{is-lowest}
% expliquer en EN que on peut pas lambda(cp1) ET lambda(cf)

A is an array of boolean variables with a size of $m$, where each variable indicates whether the corresponding part is the lowest stratum. In other words, $A(v)$ is true if v is the lowest stratum. The notation "A" was chosen as the uppercase of "a", which itself represents the lowest stratum. 
It is also worth to be noted that only one of the parts can be the lowest stratum at the time. This does not mean that two parts cannot equal the lowest stratum at the same time, it \textit{is} indeed possible that two parts blend in unison in the final chord, and that both pitches are the lowest sounding notes. It means that only one of those two is going to be considered to \textit{be} the lowest stratum (and the other one will be the intermediate stratum). This is needed in order for motions to work well. See~\ref{subsubsection:one-part-per-stratum} for the details.

Here is the mathematical definition of the A array:
\begin{equation}
\begin{aligned}
\forall j \in [0, m-1)& \colon  \\
A(cf)[j] &= \,  
\begin{cases}
    \top & \text{if } N(cf)[0,j] = N(a)[0,j] \\
    \bot & \text{else }
\end{cases}\\
A(cp_1)[j] &= \,  
\begin{cases}
    \top & \text{if } (N(cp_1)[0,j] = N(a)[0,j]) \land \neg A(cf)[j] \\
    \bot & \text{else }
\end{cases}\\
A(cp_2)[j] &= \,  
\begin{cases}
    \top & \text{if } \neg A(cf) \land \neg A(cp_1)\\
    \bot & \text{else }
\end{cases}
\end{aligned}
\end{equation}

As can be seen in these equations, only the downbeat of each measure is taken into account when computing the A array. The reason for this is that it is the downbeat note that determines which chord will be \textit{the} chord of the measure, and the other beats are just fioritures. Another reason for this is also that it is only going to serve in contexts where the first note of the measure is relevant.

\paragraph{}
In practice, there is only an \texttt{is-not-bass} array in the code (which is then equal to $\neg A$), as it is almost always more useful to know if a part is \textit{not} the lowest stratum than knowing if it is the lowest one. 

\subsection{Modified constants} \label{subsection:modified_constants}
\noindent \textbf{species} \hspace{.2cm} \texttt{species} 

The species constant represents the species of a given part: 1 for the first species, 2 for the second, and so on, with each part having its own species constant. This constant can now also take the value zero: this means that we are talking about \cfs (which can be understood as a simplified first species counterpoint). This constant is more useful in the code than in the mathematical notations.
\begin{equation}
species(v) = 0 \iff v = cf
\end{equation}

\subsection{Redefined variables with respect to the definitions in T. Wafflard's thesis} \label{subsection:modified_variables}
Since of the rules now apply between parts and the lowest stratum, the meaning of the variables has been modified to reflect this reality. Throughout this section, when reference is made to the past ("this variable used to be", "this variable keeps the same meaning", ...), it means that reference is made to the previous definition of the variable, which was the one defined in T. Wafflard's work.

\paragraph{Nota bene}
Please take into consideration that all the rules from T. Wafflard's thesis (which can be found in Appendix~\ref{appendix:complete-set-of-rule}) are compatible with the new definitions of the variables, as discussed in~\ref{exploring-interaction-p-a}. 

\vspace{.5cm} \noindent \textbf{N}(v) \hspace{.2cm} \texttt{notes} 

N is the array corresponding the pitches of each voice. Its size is $s_m$. It is the same array as the one named \texttt{cp} in T. Wafflard's thesis, and it got renamed to \texttt{N} (for notes), for the sake of clarity. As we have now three of those arrays (one for the first counterpoint, one for the second counterpoint, and even one for the \cf), it needed a less ambiguous name than the one it had before.



\vspace{.5cm} \noindent \textbf{H}$_{(\text{abs})}$(v$_1$, v$_2$) \hspace{.2cm} \texttt{h-intervals}\hspace{.2cm} \texttt{h-intervals-abs}\hspace{.2cm} \texttt{h-intervals-to-cf}\hspace{.2cm}  ...

This variable is an array of size $s_m$ and it represents the harmonic interval between one voice and another. The previous definition of this array was that it represented the harmonic intervals between a given voice and the \cf. This definition has been extended to include intervals other than that between the counterpoint and the \cf. In order to do so, H now accepts two arguments, and it represents the interval between those two arguments. Thus, $H(v1,v2)[i,j]$ represents the intervals between the $i$th beat of voice $v_1$ and the \textit{first} beat of voice $v_2$. $v_1$ may be a part and $v_2$ may be a stratum, as you can calculate harmonic intervals between a part and a stratum. When no $v_2$ is precised, it is equal to $a$, by default. In other words, $H(v_1)$ represents the intervals between the voice $v_1$ and the lowest stratum: $H(v_1) \equiv H(v_1,a)$. This default value for $v_2$ was chosen since it is the most frequently used, and for a good reason: most relevant harmonic intervals are those between the parts and the lowest stratum.

Here is the generalisation explained above, matching to the current definition of the harmonic intervals array:
\begin{equation}
\begin{aligned}
    &\forall v_1, v_2 \in \{cf, cp_1, cp_2, a, b, c\}, \quad \forall i \in \mathcal{B}(v_1), \quad \forall j \in [0, m):\\
    &H_{abs}(v_1,v_2)[i, j] = \left|N(v_1)[i, j] - N(v_2)[0,j]\right|\\
    &H(v_1-v_2)[i, j] = H_{abs}[i, j]\ \text{mod}\ 12\\
    &\text{where } H_{abs}[i, j] \in [0, 127], H[i, j] \in [0, 11]
\end{aligned}
\end{equation}

\vspace{.5cm}
\noindent \textbf{M}$_{\text{brut}}$(v) \hspace*{.2cm} \texttt{m-intervals-brut}

The variable M represents the melodic intervals of a voice. It can either be evaluated on a part or on a stratum, each of those situations leading to different behaviours. M(p) (i.e. when related to a part) keeps representing the melodic intervals of the voice v and its way of working remains intact as in T. Wafflard's thesis. M(s) (i.e. when related to a stratum) has it own way of working, that is defined in the next paragraph. We are going to focus specifically on M(a), that is, the melodic intervals of the lowest stratum.

Since strata don't have melodic intervals \textit{per se} (they actually do have melodic intervals, but it doesn't really make sense to consider them), we need to redefine what we mean when speaking about the melodic intervals of a stratum. If it is not clear why strata have no inherent melodic intervals, remember that strata are an abstract concept that is used only in mathematical relationships (and respective constraints). People who listen to the music hear the different parts (be they different tessitura, different instruments, ...) and the way these parts interact together in melodic movements and harmonic convergences, rendering a beautiful music, or not. Strata are an abstraction of the harmonic interactions between the parts, and because of this, they are a consequence of the parts: they exist because the parts exist, and not the other way round! And since they are defined according to harmonic principles (as was suggested before, they are successions of vertical alignments), speaking about the proper \textit{melodic} intervals of a stratum makes no sense. One could then conclude that melodic intervals do not apply to strata, and go ahead. Nevertheless, Fux \textit{does} speak about computing the motions between a part and the lowest stratum. And to be able to compute motions, one needs to compare two different melodic intervals. So we need to have a definition for the melodic intervals of a stratum. 

\paragraph{}
To understand how we arrive at a definition for the melodic intervals of a layer, we need to remember that the lowest layer is just the collection of all the lowest-sounding notes in the composition. It is therefore quite logical to think of the melodic intervals of the lowest layer as the melodic intervals that lead to all those lowest-sounding notes. If the lowest stratum consists of the notes [C$_{cp_1}$, E$_{cf}$, G$_{cp_2}$] (where C$_{cp_1}$ indicates that the C belongs to the first \cp), in the \cfs the interval that lead to the E is a +0 (i.e. staying on the same note), and in the second \cps the interval that lead to the G was a -4 (getting down of two tones), the corresponding melodic intervals array of the lowest stratum would be [+0, -2]. This example has been written again in a more visual way in equation~\ref{eq:defining-m-intervals-bass} to make it easier to understand. To the left of the equation is the pitch array of each voice mentioned. To the right of the equation is the melodic interval array of each voice mentioned. The numbers in bold red are those corresponding to the lowest stratum.


\begin{equation}
    \begin{aligned}        
    N(cf) &= [64,\quad  \textcolor{darkred}{\textbf{64}},\quad  71] \quad 
    &M_{brut}(cf) &= [\textcolor{darkred}{\textbf{+0}}, \quad +7]\\
    N(cp_1) &= [\textcolor{darkred}{\textbf{60}},\quad  67,\quad  74] \quad 
    &M_{brut}(cp_2) &= [+7, \quad +7]\\
    N(cp_2) &= [72,\quad  71,\quad  \textcolor{darkred}{\textbf{67}}] \quad 
    &M_{brut}(cp_2) &= [-1, \quad \textcolor{darkred}{\textbf{-4}}]\\
    \\
    N(a) &= [\textcolor{darkred}{\textbf{60}},\quad  \textcolor{darkred}{\textbf{64}},\quad  \textcolor{darkred}{\textbf{67}}] \quad 
    &M_{brut}(a) &= [\textcolor{darkred}{\textbf{+0}}, \quad \textcolor{darkred}{\textbf{-4}}]\\
\end{aligned}
\label{eq:defining-m-intervals-bass}
\end{equation}


The formal definition of the melodic intervals of the lowest stratum is hence as follows: the melodic interval in measure $j$ of the lowest stratum is equal to the last melodic interval in measure $j$ of the part that is the lowest stratum in measure $j+1$. Remember that this complex definition is needed in order for the computation of the motions to work fine, and that the motions of the lowest stratum are an abstract notion that serves only in formulas and constraints and \textit{does not intend to represent any concrete motion really happening in the composition, nor does it correspond to the melodic intervals between the pitches of the lowest stratum}.

\begin{equation}
    \begin{aligned}
        &\forall j \in [0, m-2):\\
        &M_{brut}(a)[j] = \,  
        \begin{cases}
            M_{brut}(cf)[0][j] & \text{if } A(cf)[j+1]\\
            M_{brut}(cp_1)[\text{max}(\mathcal{B}(cp_1))][j] & \text{if } A(cp_1)[j+1]\\
            M_{brut}(cp_2)[\text{max}(\mathcal{B}(cp_1))][j] & \text{if } A(cp_2)[j+1]\\
        \end{cases}
    \end{aligned}
\end{equation}

\noindent It might be helpful to have a look at figures~\ref{fig:stratum-m-intervals-1} and~\ref{fig:stratum-m-intervals-2} to understand better how the melodic intervals arrays for the lowest stratum.

\vspace{.5cm}
\begin{minipage}{0.46\textwidth}
    \centering
    \includegraphics[width=\textwidth]{Images/stratum-m-intervals.png}
    \captionof{figure}{Understanding the melodic intervals of the lowest stratum with a first species counterpoint}
    \label{fig:stratum-m-intervals-1}
    \end{minipage}
    \hfill
    \begin{minipage}{0.46\textwidth}
      \centering
      \includegraphics[width=\textwidth]{Images/stratum-m-intervals2.png}
      \captionof{figure}{Understanding the melodic intervals of the lowest stratum with a second species counterpoint}
      \label{fig:stratum-m-intervals-2}
\end{minipage}

As can be seen in figures~\ref{fig:stratum-m-intervals-1} and~\ref{fig:stratum-m-intervals-2}, the melodic intervals of the lowest stratum are those that lead to the notes of the lowest stratum.


\vspace{.5cm}
\noindent \textbf{P}(p) \hspace*{.2cm} \texttt{motions}

The motions array represents the motions\footnote{Reminder: there are three types of motion: direct, when both voices move together, contrary, when one voice moves up and the other moves down, and oblique, when one voice doesn't move and the other does} of a voice v with respect to the lowest stratum. The change from the previous work (where the motions array represented the motions with respect to the \cf) was made since Fux considers that the motions should be considered between each voice and the lowest voice. Of course, to be able to compute the motions between two voices, we must compare their melodic intervals, hence, we must deal with melodic intervals of a stratum. This is not a problem anymore since we have defined what the melodic intervals of the lowest stratum mean in the previous sub-section.
However, a problem arises when computing the motions of the part that is also the lowest stratum in some measures. When this happens, we end up calculating motions between a part and itself. Any part is inevitably moving in direct motion with itself, and this situation leads to only direct motions being calculated. This becomes problematic when considering costs (it is bad to have direct motions, but it obviously should not be bad to be the lowest stratum), and when considering some constraints. To tackle this problem, the motions of a part are now equal to -1 when the part is also the lowest stratum (which is denoted A(p), see section~\ref{is-lowest}). 

\begin{equation}
\begin{aligned}
&\forall p \in \{cf, cp_1, cp_2\}, \quad \forall x \in \{1, 2\}, \quad \forall i \in B, \quad \forall j \in [0, m - 1),\quad x := b - i\\
    &motion(p)[i,j] = \,  
    \begin{cases}
        0 &\text{if } (M_{brut}^{x}(p)[i, j] > 0 > M(a)_{brut}[j]) \\ & \quad \quad \quad \quad \quad \quad \quad \quad \quad  \vee (M_{brut}^{x}(p)[i, j] < 0 < M(a)_{brut}[j]) \\
        &\\
        1 &\text{if } M_{brut}^{x}(p)[i, j] = 0  \oplus M(a)_{brut}[j]=0 \\
        &\\
        2 &\text{if } (M_{brut}^{x}(p)[i, j] > 0 \land M(a)_{brut}[j] > 0) \\ & \quad \quad \quad \quad \quad \quad \quad \quad \quad   \vee  (M_{brut}^{x}(p)[i, j] < 0 \land M(a)_{brut}[j] <0)\\
        &\quad \quad \quad \quad \quad \quad \quad \quad \quad \vee (M_{brut}^{x}(p)[i, j] = 0 = M(a)_{brut}[j])
    \end{cases} 
    \\
    &P(p)[i,j] = \,  
    \begin{cases}
        -1 & \text{if } A(p)[j] \\
        motion(p)[i,j] & \text{if } \neg A(p)[j]
    \end{cases}
\end{aligned}
\label{eq:motions}
\end{equation}

This equation~\ref{eq:motions} may seem daunting, but it's actually very simple (just a little verbose).
It works like this: 

For each beat in the composition:
\begin{itemize}
    \item If a part is also the lowest stratum, P is -1 (i.e. non applicable, otherwise we would calculate the motion between the part and itself)
    \item If the part moves in the opposite direction to the lowest stratum, P is 0.
    \item If the part stays where it is and the lowest stratum moves (or vice versa), P is 1.
    \item If the part moves in the same direction as the lowest stratum, P is 2.
\end{itemize}
\chapter{Formalizing Fux rules into English}
This section will be all about extracting all the rules Fux mentions in his work and making sure they are unambiguous.
It will consist of six subsections: one for each species plus one for all species, because even though Fux doesn't clearly mentions rules for all species, some of his rules for the first species are more than certainly meant to be observed for all of them. 

\section{First species}
All rules described in this subsection marked with a red dot (\reddot) apply not only to the first species but to all of them. When described in this section applies to all species, it means that it applies to the first beat of every species, unless mentioned otherwise, and excepted for the fourth species, where it applies to the the beat.

\subsection{This species consists of three whole notes in each instance (p.71)}
This pretty straightforward rule is the very definition of the firsts species. It adds nothing in comparison with the rules for the two part comparison. It is hence already implemented by the first species for two voices and does not need any consideration. 

\subsection{\reddot This species consists of three notes, the upper two being consonant with the lowest (p.71)} \label{rule:consonant}
This rule is a new one, and it implicitly overrides the rule 1.H1 (from T. Wafflard) saying that \textit{all} intervals must be consonants. Here Fux states that only the upper voices and the lowest one are.

This rule was already enforced in the two parts composition, but as discussed in section \ref{section:parts-and-strata}, the big change here is that Fux mentions that it should not be applied between the counterpoints and the \textit{cantus firmus}, but between the voices and the lowest one. Therefore, we are changing the constraint %todo write nb of constraint
a little bit to adapt to this new rule:

\subsection {\reddot The harmonic triad should be employed in every measure if there is no special reason against it (p.71)}
As the footnote on page 71 states it, Fux refers to the "harmonic triad" as being a chord in this position: 1-3-5 (contrary to what is today understood as a harmonic triad).
The rule says it is not obligated, but it is preferred, to use the 1-3-5 chord, considering that 1 is the lowest voice. As this is a preference and not an absolute rule, it has been implemented as a cost. If harmonic triad is used, then the cost is 0. Else, it is 1.
% todo should not be 1 absolutely, should depend on the user
\begin{equation}
\begin{aligned}
\forall j \in [0, m-1) \colon &\\
(\neg (H_{u_1}[0, j] = 3 \lor H_{u_1}[0, j] = 4) &\lor \neg (H_{u_2}[0, j] = 7)) \\
\iff cost_{prefer-harmonic-triad}&[j] = 1
\end{aligned}
\end{equation}

This rule is considered to be true also for the other species.

\subsection{\reddot Occasionally, one uses a consonance not properly belonging to the triad, namely, a sixth or an octave (p.72)}
Here, Fux explains that when it is not possible to have a harmonic triad, you can use sixths or octaves instead. Remember that the sixths or the octaves are calculated from the lowest stratum. Since the rule \ref{rule:consonant} obligates the use of a perfect consonance (i.e. a third, a fifth, a sixth or an octave), when the harmonic triad cannot be used, it is already naturally replaced by a third or a sixth, because no other intervals are allowed. It is thus not a new rule but a restatement of rule \ref{rule:consonant}.

\subsection{\reddot The necessity of avoiding  the succession of two perfect consonances [...] (p.72)} \label{rule:succ-p-cons}
Fux here implies that there should be no two successive perfect consonances. He does not specify whether this rule applies to all three parts at once (i.e. if there was a consonance at bar X between part 1 and part 2, there cannot be one between part 2 and 3 at bar X+1), or whether it applies to each pair of parts separately. That said, in his example (Fig. 91 of the English version), we can clearly see that there is perfect consonance in every bar (parts 1-3, then 1-2, then 1-3, then 2-3, then 1-2). From this we can deduce that for each pair of parts it is forbidden for two perfect consonances to follow each other.

\begin{equation} \begin{aligned}
\forall v_1, v_2 \in \{cf, cp_1, cp_2\}, \quad v_1 \neq v_2 \quad \\
\forall j \in [0, m-2) \colon H^{v_1-v_2}[0, j] \in Cons_p \\
\implies H^{v_1-v_2}[0, j+1] \notin Cons_p
\end{aligned} \end{equation}

Which means a harmonic interval and the following one cannot be consonant at the same time.

\textbf{N.B.} This being said, Fux doesn't seem to follow this rule strictly. Maybe it should be converted as a cost.

\subsection{\reddot [Each part] follows the natural order closely (p.73)}
As this is not really clear, Fux later complements his explication by saying the counterpoints should be "moving gracefully, stepwise without any skip". This is clearly a preference, and has already been covered when implementing the first species for two voices. It can thus be ignored in the scope of this thesis.

\subsection{\reddot [Each part] follows the principle of variety (p.73)}
Fux never defines clearly what he means by "principle of variety". Nevertheless, the examples he provides are of a great help as he corrects his student not following the principle, by augmenting the variety of different notes in a single voice. This means, the principle of variety can be understood as having as many different notes in a single voice. As it is not explained either if this has to be true for the whole partition or only for two following notes, it has been chosen as an arbitrary split in two of the debate to use the principle of variety over 4 following notes. This means that the solution is penalized if a note in measure X was already present in measures [X-3, X+3].

\begin{equation} \begin{aligned}
\forall cp \in \{cp_1, cp_2\}, \quad \forall j \in [0, m-1), \quad \forall k \in [j+1, min(j+3, m-1)] :\\ cp[0, j] = cp[0, j+k]\iff cost_{diversity}[j+m*k]= 1
\end{aligned} \end{equation}

\subsection{\reddot To allow enough space for the voices to move toward each other by contrary motion, the upper voices begin distant from the bass (p.75)}
This is not a strict rule but an indication to make easier for the composer to have contrary motions. It is not an obligation, nor a preference, so it was simply added as a heuristic for the solver.

\subsection{\reddot All voices ascend[ing] [is] a progression which can hardly be managed without awkwardness resulting (p.76)}
What Fux says here is that the three parts cannot be moving in the same direction. 
To prohibit this, we just have to look at the motions between the parts and the bass. If one of their motion is contrary, then it is ensured that the three voices are not going in the same direction (as at least one is contrary). Same goes if one motion is oblique. The problem occurs if both motions are direct, as it would mean that the three voices are going in the same direction. So, we have a prohibit this, by constraining that the two motions can't be direct at the same time. 
\begin{equation} \begin{aligned}
 \forall j \in [0, m-2) \colon \neg (M_{cp_1}[0, j] = 2) \lor \neg (M_{cp_2}[0, j] = 2)
\end{aligned} \end{equation}

\subsection{\reddot [Your may reach] a perfect consonance by direct motion [if] there is no other possibility (p.77)}
This is a relaxation of the 1st species for two voices constraint saying that you cannot reach a perfect consonance by direct motion. Because it is sometimes mandatory with three voices to break this rule (as there are no other possibilities), you may derogate from this rule. Since there is no way in constraint programming to implement a rule that must not be obeyed only if possible other 
than by using a cost, the initial constraint (1.P1-W)  was rewritten to use one :

\begin{equation} \begin{aligned}
\forall v \in \{cp_1, cp_2\}, \quad \forall j \in [0, m-1) : H^{v}[0, j+1] \in Cons_{p} \land P^{v}[0, j] = 2 \\
\iff \text{{Cost}}_{\text{{direct\_move\_to\_p\_cons}}}[j] = 8
\end{aligned} \end{equation}

\subsection{\reddot One feels that the degree of perfection and repose which is required of the final chord does not become sufficiently positive with this imperfect consonance [(speaking about a tenth)] (p.77)}
When Fux says this, he takes a tenth as an example, but it here understood that the final chord cannot include a tenth (third + octave), nor an eight-teenth (third + two octaves), etc. Third are accepted though. The rule then becomes: 

\begin{equation} \begin{aligned}
&\forall v \in \{u_1, u_2\} \quad \\
&\forall h \in \{4 + (12 \times k) \mid k \in \mathbb{N} \setminus \{1\}\} \colon \\
&H_{brut}^{v}[0, m-1] \neq h
\end{aligned} \end{equation}


\subsection{\reddot Ascending sixths on the downbeat sound harsh (p.77)}
This rule is pretty straightforward and states that if an interval is a sixth, the next one cannot be one.
\begin{equation} \begin{aligned}
&\forall j \in [0, m-2)   \quad \forall v_1 \in \{cf, cp_1, cp_2\} \quad \forall v_2 \in \{cf, cp_1, cp_2\} \quad (v_1 \neq v_2) \colon\\
&\neg (H^{v_1-v_2}[0, j] \in \{8, 9\} ) \lor \neg (H^{v_1-v_2}[0, j+1] \in \{8, 9\})\\
&\lor \neg (v_1[0,j] < v_1[0,j+1]) \lor \neg (v_2[0,j] < v_2[0,j+1]))
\end{aligned} \end{equation}


\subsection{\reddot Unison is less harmonious than the octave (p.79)}
This rule is already covered by the no-unison constraint in the first species for two voices.

\subsection{\reddot One should not exceed the limits of the five lines without grave necessity (p.79)}
This rule is already covered by the melodic costs in first species for two voices.

\subsection{\reddot Skip of a major sixth is prohibited (p.79)}
Fux then explains that not only are the skips of a major sixth prohibited, but all bigger skips. This rule is already covered by the constraints of the first species for two voices. 

\subsection{\reddot The minor third is not capable of giving a sense of conclusion (p.80)}
Actually, Fux later states that minor modes should not include a third altogether, but that sometimes it is impossible to do without it, so you can use major third in a minor mode.
\begin{equation} \begin{aligned}
\forall v \in \{u_1, u_2\} \colon H^{v}[0, m-1] \neq 3
\end{aligned} \end{equation}


\subsection{\reddot First and last notes have not to be perfect consonances anymore*} \label{rule:last-chord-not-perfect-anymore}
\footnote{As a reminder, an asterisk at the end of a rule means that the rule is implicit.}

Fux doesn't state this in his text, but in many of his examples, we see that now we have 3 voices, not all voices must be perfect consonances to the first one in the first and last measure.

\subsection{\reddot Last note must be composed of the notes of the a harmonic triad*} \label{rule:last-chord-h-triad}
Again, this isn't stated explicitly but we see that all of his examples end with a chord containing exclusively the notes of the harmonic triad.

\begin{equation} \begin{aligned}
\forall v \in \{u_1, u_2\} \colon H^{v}[0, m-1] \in \{0, 3, 4, 7\}
\end{aligned} \end{equation}


\subsection{\reddot The last chord must have the same fundamental as the one of the scale used throughout the composition*}
This rule emanates from an observation of Fux's examples throughout the chapter. He always ends his examples with the same fundamental as the on of the scale.
When the \textit{cantus firmus} is the lowest stratum, this is not a problem, as the \textit{cantus firmi} always end with the fundamental note of the scale. But when not, it has to be imposed by a constraint, or we may end up with surprising results. Since the fundamental of the scale is defined by being the first note of the \textit{cantus firmus}, we impose that the last note of the lowest stratum must be equal to the first one of the \textit{cantus firmus} (taking the modulos into account).


\begin{equation} \begin{aligned}
\lambda[0, m-1] \mod 12 = N^{cf}[0, 0] \mod 12
\end{aligned} \end{equation}



\section{Second species}
\subsection{A half note may, for the sake of the harmonic triad, occasionally make a succession of two parallel fifth acceptable - which can be effected by the skip of a third (p.86)}
Fux didn't speak about prohibiting two parallel (i.e. consecutive) fifths in the second species for two voices. That being said, it is indeed prohibited in three parts composition as you cannot have two successive perfect consonances (see constraint \ref{rule:succ-p-cons}. We thus have to relax this constraint in order to accept two successive consonances, when the two successive fifths flank a third.


\subsection{Ligatures have no place in this species [except] in the final cadence (p.87)}
Fux explains that in some cases, you have no other option than ligaturing the fourth-to-last and the third-to-last notes. The reasons he gives for this are all part of the previous mentioned rules (no successive perfect consonances, no unison, ...).
This is a relaxation of the two-voice constraint and is done quite easily by modifying the constraint that says "no consecutive notes cannot be the same" unisson between two consecutive notes" to except cp[2, m-1] and cp[0, m]. The reason why this has not been implemented as a cost but as a constraint relaxation is because Fux seems not to say that there is any counterpart at ligaturing the fourt-to-last and the third-to-last notes.

\subsection{A major third [may] appear in the last chord. (p.87)}
This is a consequence of now using three voices instead of two. Fux just made explicit here a rule we had already defined (\ref{rule:last-chord-not-perfect-anymore} and \ref{rule:last-chord-h-triad}). It has thus already been implemented in the first species for two voices.

\subsection{The half notes are always concordant with the two whole notes (p.88)}
If Fux meant "consonant" when he wrote "concordant", then this rule is pretty straightforward. The concern here is that he doesn't seem to follow his own rule. Anyway, here is the mathematical notation for this rule:
\begin{equation} \begin{aligned}
    &\forall v_1 \in \{cp_1, cp_2, cf\} \quad \forall v_1 \in \{cp_1, cp_2, cf\}, v_1 \neq v_2 \colon\\
    &species^{v_1} = 2 \iff H^{v_1-v_2} \in Cons
\end{aligned} \end{equation}

\subsection{A whole note may occasionally be used in the next to last measure (p.93)}
This rule about the second species is written in a footnote of the chapter about the third species, but it seems to apply not only to when the second species is used in combination with the third species, but also in other cases. (fig. 134, 173 and 174 of the English version)

\section{Third species}
\subsection{The quarters have to concur with the whole notes of the other voices (p.91)}
When using the word "concur", it is plausible that Fux meant "are put in relation", and not "be consonant", as he said that the quarters \textit{concur} with the \textit{cantus firmus} in two voices composition. It is not a rule \textit{per se}, Fux is only annunciating that some rules are going to be introduced.

\subsection{Take care whenever you cannot use the harmonic triad on the first quarter occurring on the upbeat, to use it on the second or third quarters (p.91)}
This rule is clear, and states that we should use the harmonic triad in the second or third beat when we were not able to use it in the first one. This is not an absolute rule, but either an advice, and thus it was treated as a cost.



\section{Fourth species}
\subsection{The ligature is nothing but a delaying of the note following (p.95)}
In other words, $cp[n]$ should be associated to $cf[n-1]$ when performing calculations. This is already made by the computations of the two voices counterpoints. It is nevertheless useful to be well aware of this fact, since it will induce many special cases in the implementation.

\subsection{If the ligatures were removed for the sake of the harmonic triad - which, hower, would be impossible because of another consideration, the immediate succession of several fifths (p.95)}
By stating the rule that prohibits the succession of fifths (which is a particular case of the rule saying that you cannot have two successive perfect consonances, see \ref{rule:succ-p-cons}) and by saying that this rule \textit{would} have been valid in the other species, Fux is telling us in a roundabout way that it is not valid in this species. he further complements by saying "there is great power in ligatures - the ability to avoid or improve incorrect passages". All this means that successions of fifths are allowed in the 4th species.

We shall then amend rule \ref{rule:succ-p-cons} and rewrite it as:
\begin{equation} \begin{aligned}
    &\forall v_1, v_2 \in \{cf, cp_1, cp_2\}, \quad v_1 \neq v_2 \quad \forall n \in \{0, 1\} \\
    &i_n := \,  
    \begin{cases}
        0 & \text{if } species^{v_n} \neq 4\\
        2 & \text{if } species^{v_n} = 4\\
    \end{cases}\\
    &\forall j \in [0, m-2) \colon H^{v_1-v_2}[i_n, j] = 0 \iff H^{v_1-v_2}[i_n, j+1] = 0
\end{aligned} \end{equation}

\subsection{The fifth is a perfect consonance, the octave a more perfect one, and the unison the most perfect of all; and the more perfect a consonance, the less harmony it has (p.97)}
This is almost covered by the existing costs, as a perfect consonance has a cost of 1, where an imperfect consonance has a cost of 0. This precision in the rule (fifth is better than octave) could be solver by either putting a cost of 2 to octaves and a cost of 1 to fifths, or to put the cost for fifth before the cost for octaves in the lexicographical array of costs.
\subsection{A dissonance that resolves to a fifth is more acceptable than a dissonance that resolves to an octave (p.98)}
This is be covered by the previous cost, as the fifth is preferred to the octave.

\subsection{If I said that the first note of the ligature must always be consonant, that applies only to the instances in which the bass remains on a pedal point, that is, in the same position. In such a case a ligature involving only dissonances is not only correct but even very beautiful (p.98)}
The rule evoked here is changing a previous rule about consonance (4.H1W\footnote{E2.2W means "Equation 2.2 from T. Wafflard's thesis".

S1.3W means "Section 1.3 from T. Wafflard's thesis".}). From now on, if the lowest stratum has a stationary movement, the corresponding note in the fourth species must be a dissonance, instead of a consonance.

\begin{equation}
\begin{aligned}
&\forall j \in [0, m-1):\\
&M^\lambda \neq 0 \iff H[2, j] \in Cons\\
&M^\lambda = 0 \iff H[2, j] \in Dis
\end{aligned}
    \label{eq:arsiscons} 
\end{equation}

\subsection{The tenor takes the place of the bass in the first measure - a thing that not only the tenor may do, but also the alto and even the soprano}
Fux speaks here about our concept of records. The tenor can become the lowest record, just like the alto and the soprano. This is a fundamental concept of the generalization of Fux counterpoint to three voices, and has already been extensively discussed before (see section \ref{section:parts-and-strata}).

\subsection{To avoid hidden fifths, we may leave the first note of the second counterpoint void.}
I am not sure to have understood the hidden fifths. This is the last constraint that needs to be implemented.

\section{Fifth species}
No supplementary rules have been observed by Fux for the fifth species.

However, in order to have variety when composing with two fifth species counterpoints, it was decided to impose a rule that Fux never mentioned: knowing that the fifth species is just all other combined, not more than 50\% of the composition of the parts can be made using the same species for the same measure. This translates as:
\begin{equation}
\begin{aligned}
species^{cp_1} = 5 = species^{cp_2}  \iff \sum_{i=0}^{3} \sum_{j=0}^{m-1} (S^{cp_1}[i,j] = S^{cp_2}[i,j]) < \frac{s_m}{2}
\end{aligned}
\end{equation}

\chapter{Searching for the best existing solution}
Three parts composition brings in way more possibilities than two parts composition. 
But more possibilities also mean an increased computation complexity. The search space has been extended a lot by adding a whole new set of variables, and the time taken for a solution to be found might to be too elevated if one does not think about optimizing the search. In addition to that, adding a third voice to a composition is not bringing many new constraints (which would help discarding some potential solutions faster), but instead comes with many preferences, which in constraint programming, are translated to costs.

In addition, we need to find a way of arranging the costs in a way that comes as close as possible to what Fux was trying to express in his book. There are several ways to do this, which we will go through and discuss.  

This chapter will begin by explaining the search algorithm that is used to find a solution, continue by discussing the different ways of considering costs, and end by analysing the results that these different perspectives produce.

\section{Using Branch-And-Bound as a search algorithm}

To cope with the increased complexity brought about by the three-part composition, it was decided to switch from the Depth First Search algorithm (used in T. Wafflard's thesis) to a more efficient Branch and Bound (BAB). This allows us to handle costs properly and to find faster solutions. Moreover, the BAB algorithm can also produce non-optimal results, which is very valuable since finding the best overall solution can be time-consuming. When starting the search for a solution, it is now possible to ask for the next solution (i.e. a better solution than the one found previously, and if none was found previously, then just any valid solution), or for the best solution. In the latter case, the solver will continue to search until it finds the best solution or until it is stopped, returning a better solution each time it finds one.


\subsection{Heuristics}
When it comes to finding a solution, we obviously need some heuristics to guide the search, as there are so many different possibilities for a three-part composition. 
To know which heuristics we apply, we simply think about the most important variable to fix first. The most important 


\section{Designing the costs to be as faithful as possible to \gap} \label{costs}

Knowing that we are looking for the solution whose cost must be as low as possible, the question arises: how can we calculate the cost in order to best reflect the preferences expressed in \gap?

The way to translate each preference into a corresponding cost has of course been formalised in the previous sections, but that's not the crux of the matter. The question we face here is: what is the best way to combine all these individual costs to get the most accurate result in terms of what Fux is trying to convey?

Three main ways of doing this have been identified: a linear combination between costs, a search that minimises costs by lexicographic order, and a cost ordering that involves the calculation of minima. We will first describe each of these techniques and their respective advantages, and then compare them (and the results they produce).



\subsection{Linear Combination}


The first method of calculating our costs is a linear combination. This is the technique used in T. Wafflard's thesis. More precisely, it uses a linear combination in which all the weights are equal to one.



To be more precise about the method used to calculate the total cost in T. Wafflard's thesis, here is a more detailed explanation: there exists a total cost, $\tau$, which is equal to the sum of all individual costs, $\mathcal{C}$. The next step is to minimise $\tau$. Each $\mathcal{C}_i$ is usually itself a sum of sub-costs. Take, for example, the cost of motions, $\mathcal{C}_{motions} = \sum_j P_{costs}[j] $. This cost is the sum of all sub-costs of the motions (one per motion): by default, a contrary motion has a sub-cost of 0, an oblique motion has a sub-cost of 1 and a direct motion has a sub-cost of 2. These default values can be changed by the user to be set somewhere on a scale that ranges from $0$ to $64m$. For example, the user could set the oblique motion cost to be equal to $0$, and the cost for direct and contrary motion to be equal to $64m$, in order to get a composition filled with as many oblique motions as possible (always in accordance with the basic rules from \gap, i.e. all voices are never going to go in the same direction, see \ref{rule:same-movement}).

As mentioned at the beginning of this subsection, this procedure can be understood as a linear combination with weights of one only. However, since the cost factors are given different values according to the user's choices, this method is actually more like a regular linear combination, except that the weights are not multiplied by the costs once the latter have been set, but the costs are themselves made larger or smaller before the linear combination is calculated.

The linear combination has two major advantages: ease of implementation and high comprehensibility.


However, it has a major drawback: since the total cost $\tau$ we are minimising in a linear combination is the sum of all costs $\mathcal{C}$, the best solution might be a solution where one cost is absolutely huge and all the others are small. This might not be a problem if the outstanding cost is not really relevant, but if it is the cost of not using a harmonic triad, it goes completely against the preferences that Fux conveys in his work, making the solution inappropriate. A representation of this situation can be found in the figure \ref{fig:outstanding-cost}.

\begin{figure}[h]
    \centering
    \includegraphics[width=1\textwidth]{Images/outstanding-costs.png}
    \caption{Example of a situation where a solution with an outstanding cost is preferred to a solution with equivalent low costs when using a linear combination}
    \label{fig:outstanding-cost}
\end{figure}

Another drawback of linear combination is that the result is pretty and unpredictable: changing the value of the cost may or may not make a difference, and you may need to set huge values to see a real effect. For example, if a composer really wants oblique motion, they may be forced to set the cost of the other types of motion to a huge value, or they may not see the difference between the default solution and their personalised solution. This is due to the fact that all the costs are mixed together and form an indistinguishable soup that the solver considers as a whole, and a small increase in the cost of the direct and contrary motions is very likely to be absorbed into this soup without any change being noticed.

These two drawbacks make the linear combination solution for the costs hardly acceptable when it comes to representing the preferences.
We will therefore examine the other two options for adjusting costs.

\subsection {Minimum Comparison}
In order to overcome the problem of outstanding costs that we encountered when considering the linear combination solution, one might consider using some minimums when calculating $\tau$, the total cost. For example, $\tau$ could be the maximum of all costs. By doing this, the solver would try to find a solution where the focus is on the worst cost and try to reduce it before trying to reduce the other costs.

The problem of this method arises when one cost is significately higher than the others, because it was defined like this. Let's get back to our example of the compositor wanting as many oblique motions as possible. They are going to set the cost for direct motions and contrary motions to the highest cost possible, and will start the search. As we've already discussed it, it is not possible to have only oblique motions, as it would contradict the rule stating that not all voices can move in the same direction (\ref{rule:same-movement}). In consequence, there will always exist contrary motions, and since the cost for them was set really high, it would be impossible for the solver to converge on a good solution. This creates a bottleneck effect, in which when the solver reached the best potential value of the worst cost, it cannot go on finding better solutions. 

Furthermore, even when taking into account a less extreme case (for example the default setting), this method requires a normalisation of the costs: there exists $3\times (m-2)$ sub-costs of the variety cost, $3\times (m-1)$ sub-costs of the motions cost, but only $m$ sub-costs for the octave cost. This means that without a normalisation, the motions cost is going to be in average three times bigger than the octave cost, which consequently means the solver will put three times more effort in minimizing the motions cost than the octave cost, which is unfair and unpractical.

from a given solution to a better one. 

\subsection{Lexicographical Order}
The second way of dealing with the costs is to arrange them in an array and then perform a lexicographic minimisation. In other words, the costs would be arranged in order of importance: from most important to least important. The most important cost to minimise would be placed first in this array, and the solver would only try to minimise the other costs if the first cost remained the same or decreased. This method makes a lot of sense when you think about the rules that emanate of \gap. For example, Fux says that perfect consonance can be achieved by direct motion if there is no other possibility. This means that, all other things being equal, we would prefer to achieve perfect consonance by oblique or contrary motion, but that between a bad solution (respecting almost no preferences) in which perfect consonance is not achieved by direct motion, and a good solution (respecting almost all preferences) in which perfect consonance is achieved by direct motion, we would choose the good solution. 

Some costs are also more important than others in absolute terms. For example, when Fux says that an imperfect consonance is preferred to a fifth, which is preferred to an octave. This amounts to lexicographically ranking the cost of using an octave first (because we really don't want octaves), and then the cost of using a fifth (and there is no cost of using an imperfect consonance, since Fux indicates that this is preferable).
\begin{figure}[h]
    \begin{equation}
        \begin{aligned}
            \tau = [\underset{\text{minimise this first}}{\underbrace{\mathcal{C}_\text{octaves}}}, \mathcal{C}_\text{fifths}]
        \end{aligned}
    \end{equation}
    \caption{Array of costs demonstrating the practicality of a lexicographical order solving.}
\end{figure}

A second example, which ties in particularly well with the first, is that Fux tells us that the harmonic triad must be used in every measure unless a rule forbids it. In saying this, he places the preference for the harmonic triad above all other preferences, because the only reason that can prevent the use of a harmonic triad is a fixed constraint (and not a preference). You'll notice that the harmonic triad consists of a fifth (which is a perfect consonant), so Fux is telling us that we'd rather use a fifth in a harmonic triad than an imperfect consonant outside a harmonic triad. The lexicographic order search is the only one that allows this kind of concept to be taken into account, because in a linear combination these two preferences would be mutually "exclusive"\footnote{in the sense that their effects would work against each other.}: the first preference would add a cost where the second preference would not, and the second preference would add a cost where the first would not.

\begin{figure}[h]
    \begin{equation}
        \begin{aligned}
            \tau = [& \underset{\text{\fontsize{7}{11}\selectfont{minimize this first}}}{\underbrace{\mathcal{C}_\text{harmonic\_triad}}}, \underset{\text{\fontsize{7}{11}\selectfont\parbox{4cm}{and start minimizing this only if it is not possible anymore to minimize the harmonic triad cost}}}{\underbrace{\mathcal{C}_\text{octaves}}},\quad  \mathcal{C}_\text{fifths}]
        \end{aligned}
    \end{equation}
    \caption{Array of costs demonstrating the practicality of a lexicographical order solving.}
\end{figure}

And in this way we can keep integrating the different costs until we get a full array $\tau$ with all the costs ordered in a lexicographical way.

Taking into account all costs, as defined by Fux in \gap, this is the order we get $\tau =$
\begin{multicols}{3}
    \begin{enumerate}
        \item $\mathcal{C}_\text{not-cambiata}$
        \item $\mathcal{C}_\text{penult-thesis}$
        \item $\mathcal{C}_\text{off-key}$
        \item $\mathcal{C}_\text{successive-p-cons}$
        \item $\mathcal{C}_\text{no-syncope}$
        \item $\mathcal{C}_\text{harmonic\_triad}$
        \item $\mathcal{C}_\text{harmonic\_triad\_3rd\_species}$
        \item $\mathcal{C}_\text{octaves}$
        \item $\mathcal{C}_\text{fifths}$
        \item $\mathcal{C}_\text{variety}$
        \item $\mathcal{C}_\text{motions}$
        \item $\mathcal{C}_\text{m-degrees}$
        \item $\mathcal{C}_\text{m2-eq-zero}$
        \item $\mathcal{C}_\text{direct-move-to-p-cons}$
    \end{enumerate}
\end{multicols}
    

\subsection*{Comparison between the three types of costs.}


\begin{center}
    \centering
    \captionof{table}{Comparison of Three Methods According to Criteria}
    \label{tab:comparison}
    \begin{tabularx}{\textwidth}{|>{\centering\arraybackslash}p{4cm}|>{\centering\arraybackslash}X|>{\centering\arraybackslash}X|>{\centering\arraybackslash}X|}
        \hline
        \textbf{Criteria} & \textbf{Linear Combination} & \textbf{Capping with minima} & \textbf{Lexicographic Search} \\
        \hline
        Outsanding costs & \cellcolor{red!25}Yes & \cellcolor{green!25}No & \cellcolor{green!25}Only for minor costs \\
        \hline
        One cost might be a bottleneck & \cellcolor{green!25}No & \cellcolor{red!25}Yes & \cellcolor{green!25}No \\
        \hline
        Possibility to ensure a preference of one cost over another  & \cellcolor{red!25}No & \cellcolor{red!25}No & \cellcolor{green!25}Yes \\
        \hline
        Need for hierarchisation of costs & \cellcolor{green!25}No & \cellcolor{green!25}No & \cellcolor{red!25}Yes \\
        \hline
        Need to normalise costs& \cellcolor{green!25}No & \cellcolor{red!25}Yes & \cellcolor{green!25}No \\
        \hline
    \end{tabularx}
\end{center}
blabla
\chapter{Musicality of the solutions}
This last chapter is concerned with the musicality of the solutions. Its objective is not anymore to discuss the best way to translate Fux's preference into costs, but to play a little bit with the possibilities offered by the tool. We try to change the costs in order to leave Fux's initial preferences and to see what it can yield as a result.

\section{Trying custom costs}
Below is an example of the difference between a counterpoint of the first species, using Fux's preferences (more precisely, using the default preferences defined in section 5), and a counterpoint in which personal preferences were expressed. These preferences were: to use as few contrary motions as possible, and as many oblique and direct motions as possible. Prioritising this preference (placing it first in the lexicographic order), then prioritising melodic intervals (maintaining a preference for small melodic intervals, as in \gap), then prioritising variety, and then placing all the other preferences at the penultimate level of the lexicographic order, with the exception of the preference for no successive perfect consonance, which was placed at the very last level.

The search has not been stopped manually; we leave it running until it finds the best solution according to the defined preferences.

The aim of this experiment is twofold: to show that changing the order of the preferences has an effect on the solution, which was already partially demonstrated in the previous chapter, and to obtain a much more monotonous solution than the Fux-like solutions (for example, for transitions between two parts of a composition, for more quiet moments, ...).

The results of this experiment can be found in figures \ref{fig:musicality-1sp-fux} and \ref{fig:musicality-1sp-custom}.
\begin{figure}[h]
    \centering
    \includegraphics[width=1\textwidth]{Images/Musicality/musicality-1sp-fux-pref.png}
    \caption{Example of a fifth species counterpoint in three-part composition with Fux's preferences}
    \label{fig:musicality-1sp-fux}
\end{figure}

\begin{figure}[h]
    \centering
    \includegraphics[width=1\textwidth]{Images/Musicality/musicality-1sp-custom-pref.png}
    \caption{Example of a fifth species counterpoint in three-part composition with custom preferences}
    \label{fig:musicality-1sp-custom}
\end{figure}

The result is striking, as the Fux-like solution simply sounds like... Fux, and the custom solution is completely in line with its aim, which was to create a composition that is more monotonous and where there is less sense of things happening.
\chapter{Conclusion - Chapter is Work In Progress}
It is time to look back at the work that has been done, to highlight the progress that has been made, but also the shortcomings and gaps that need to be filled by future improvements. We will therefore use this chapter to discuss some of the key points that emerge from this thesis.
 
\section{Intended use of the FuxCP tool}
Throughout this work, it is clear that all the examples provided are fairly short (fourteen bars at most). This is primarily due to Fux himself, as the examples he gives are always of the same shortness, probably for pedagogical purposes. He does not mention this explicitly, but there may also be a practical reason for considering only such short compositions. Indeed, these small compositions can be considered as 'blocks', which can then be arranged to form a whole. The great advantage of this approach is that the countrepoints can be given different species between the blocks, allowing the composition to be constantly renewed.

\begin{figure}[h]
  \centering
  \includegraphics[width=1\textwidth]{Images/composition-in-blocks.png}
  \caption{Example of what of a composition in blocks could look like}
  \label{fig:composition-in-blocks}
\end{figure}

\section{Known issues about the current state of the work} \label{section:known-issues-about-the-current-state-of-the-work}
\begin{itemize}
  \item As mentioned in section \ref{section:time-to-find-a-solution}, some few combinations of species, voice ranges and \textit{cantus firmi} cause the solver to fail to find a solution. The current roundabout way to "solve" this is to... change the voice ranges or some other parameter until a structural solution is found. These cases are relatively rare and do not prevent the use of FuxCP.
  \item If a counterpoint of the fourth species is the lowest stratum, the solver needs more time to find a solution in which all notes are ligated. This is not a problem when combining a fourth species counterpoint with a simple species counterpoint (first or second species), but becomes difficult to handle with more complex species (third or fifth species), as the search time before the solver finds a suitable solution (i.e. with all notes tied) can become very long.
  \item The current heuristics make the solver branch on the lowest stratum array with a "select the values greater than (min+max)/2" policy. For some reason, it always seems to take the lowest value (or close to it), which means that the melody of the voice that is the lowest stratum is really redundant, something that even variety preference doesn't compensate for on complex species, since the time needed to find a good solution (i.e. one where the variety cost is low, which again means that many notes are different) is too high compared to the time needed to find a passable solution.
  
  This leads to a second problem, which is that since the downbeat of the part, which is the lowest layer, always consists of the lowest possible note, the following notes (e.g. in the upbeat) will necessarily be higher. This is because the note in the 2nd, 3rd and 4th bars cannot be the same as the note in the 1st bar, but because the 1st bar is already the lowest possible, the notes in the 2nd, 3rd and 4th bars must be higher. 
  
  As this may be difficult to understand, see Figure \ref{fig:limitation-lowest-array}, which shows that the second species of counterpoint, in the bass, is composed exclusively of upbeat notes that are higher than the downbeat notes. This is a side effect of the heuristic, and a way should be found to resolve it. Note that in this example the problem is solved very quickly (a matter of seconds) because in this setting the other counterpoint is a first species counterpoint (i.e. the setting is not complex, as explained in \ref{section:time-to-find-a-solution}) and the and the solver finds solutions where the upbeat is not always higher than the downbeat very quickly.
\end{itemize}
\begin{figure}[h]
  \centering
  \includegraphics[width=.6\textwidth]{Images/limitation-with-lowest-array.png}
  \caption{Illustration showing the limitation of current heuristics, as the counterpoint of the second species always has a higher upbeat than its downbeat. Note that the composition has been cropped.}
  \label{fig:limitation-lowest-array}
\end{figure}




\section{Future work}
The work towards fully automated counterpoint composition is progressing, but there is still a long way to go before we can claim to have finished the job. There are several ways to improve the current situation for those who want to improve FuxCP. Here are some ideas for improvement:
\begin{enumerate}
  \item \textbf{Solution quality improvements}
  \begin{itemize}
    \item \textit{Finalise the formalisation of \gap} -- The first obvious way to improve the tool is simply to complete the formalisation of Fux's work, which would make it possible to compose in four voices, but also to integrate the additional rules he mentions in his last chapter. This would make it possible to have a complete FuxCP tool, in terms of Fux's rules, and then to supplement Fux's rules with additional rules that could come from any influence. This idea of adding external rules to the Fux rules has already been tested to some extent in the work of T. Wafflard, and the results were more than promising. It is therefore clearly a direction to take in the improvement of the FuxCP tool.
    \item \textit{Relate the notes of the 2nd, 3rd and 4th beats to each other} -- As mentioned in one of the preference ordering experiments (see \ref{subsection:third-experiment-with-costs}), there are no direct constraints between the 2nd, 3rd and 4th beats of each counterpoint. The only way they influence each other is through the transitivity of the constraints: A and B are constrained together, and so are B and C, so A and C are connected in some way. The reason there are no direct constraints is that Fux didn't mention any. Anyway, this leads to some unmelodicity, and it is definitely a good idea to find some rules (e.g. from other authors) that could deal with this unmelodicity. 
    \item \textit{Address any of the current limitations} -- The section \ref{section:known-issues-about-the-current-state-of-the-work} discusses some limitations and issues with the current state of the work. Solving any of them would be an improvement for FuxCP.
  \end{itemize}
  
  
  \item \textbf{Software architecture}
  \begin{itemize}
    \item \textit{Migrate the project to C++} -- Gecode is written in C++, and C++ is a language much better suited to managing implementations like FuxCP. GiL works really well, but has shown its limitations more than once: way too verbose, hard to manage objects (which are useful for designing FuxCP) since using Lisp, and lacking some of Gecode's features. These reasons alone are a huge incentive to migrate the whole implementation to C++. This would make it possible to further improve the implementation with more convenience and efficiency.
    
    Here we repeat the words of T. Wafflard, who had already reached the same conclusion:
    \begin{quotation}
      ``Currently, constraints are added to a species via a long function that dispatches the constraints, rather than via class inheritance. Ideally, object-oriented inheritance
      should be used to represent the different variable arrays and species. All variable arrays (H, M , P , etc.) have something in common, whether in terms of their size rel-
      ative to the cantus firmus, or in terms of the way certain rules are applied. A relatively abstract class should represent this type of array to enable these commonalities to be brought together.

      The same applies to species that share common rules and should have been represented in a class system of their own. It would be logical for species to be children of the first species. Unfortunately, the scope of this work does not allow for a complete overhaul of the architecture. Moreover, in the near future, the entire code may have to be redone in C++ for reasons of performance, features, maintainability, and so on.
      Also, GiL has reached its limits, both in terms of ease of programming and in terms of possibilities. The Lisp language is not designed for writing mathematics, since each operation requires a different function call. Code readability can become complicated because these calls are all represented by parentheses. At the same time, it is not possible with GiL to combine basic mathematical operations to form a larger one. One has to break down each complex operation into simple intermediate basic operations a bit like writing assembly, which is undesirable for larger projects. Not to mention that branch-and-bound, heuristics, and multithreading seem complicated to implement in
      GiL.'' \cite[p.67]{wafflard2023}
      \end{quotation}
  \end{itemize}

  \item \textbf{Solver performance}
  \begin{itemize}
    \item \textit{Bettering the heuristics and reorganising the constraints} -- Obviously, increasing the speed at which the solver finds solutions increases the speed at which the solver finds good solutions. It is therefore crucial to continue working on the heuristics to find better and better solutions. At the same time, once the globality of \gap has been formalised, it might be interesting to rethink the constraints that apply to the composition in an intelligent way, to make the solver's work easier and to have a set of constraints that hold together better.
  \end{itemize}
\end{enumerate}
\addcontentsline{toc}{chapter}{Bibliography}
\printbibliography

%\bibliographystyle{plain}
%\bibliography{Bibliography/cite}
\appendix
\chapter{Software Architecture}\label{chapter:architecture}
\begin{figure}[h]
    \centering
    \includegraphics[width=1\textwidth]{Images/structure_memoire.png}
    \caption{Software architecture of FuxCP}
    \label{fig:softwarearchitecure}
\end{figure}
\chapter{User Guide}\label{chapter:user-guide}
This user guide provides a overview of FuxCP, covering its installation process, usage within OpenMusic, and a description of the costs displayed in the interface. While FuxCP is designed to be compatible with all platforms, it relies on GiL, which currently works only on MacOS and Linux. Unfortunately, GiL does not support Windows due to compatibility issues between the 32-bit Lisp license used by OpenMusic and the 64-bit Gecode Windows version. Although it is technically possible to obtain a 32-bit version of Gecode for Windows, it is not recommended. 

\section{Installing FuxCP}
\subsection{Prerequisites}
To use FuxCP, it is necessary to download and install the following tools:
\begin{itemize}
    \item Gecode : \url{https://www.gecode.org/download.html/}
    \item OpenMusic : \url{https://openmusic-project.github.io/openmusic/}
\end{itemize}

And download the following libraries:
\begin{itemize}
    \item GiL : \url{https://github.com/sprockeelsd/GiLv2.0/}
    \item FuxCP : \url{https://github.com/sprockeelsd/Melodizer/}
\end{itemize}
On the latest GitHub there are other tools available such as Melodizer and Melodizer2.0. For the purposes of this guide, only the FuxCP folder will be needed.

\subsection{Loading FuxCP in OpenMusic}
In order to use the above libraries, OpenMusic must be running. When opening any workspace, locate the toolbar at the top of the interface. Click on the "Windows" button, highlighted in the figure \ref{fig:library}, and select "Library" from the drop-down menu. This will bring up a new window. From the toolbar of this window, select 'File' and then 'Add Remote Library'. Navigate through your file system to find the path where the previously downloaded FuxCP and GiL libraries are stored. Once located, the libraries should appear under the "Libraries" folder in the "Library" window, as shown in Figure \ref{fig:load}. Right click on "fuxcp" and select "Load Library". If no errors occur, the setup is complete.

However, if an error arises, it may be a linking issue with the Gecode library. For MacOS users, a script can be used from the \texttt{c++} folder of the GiL library. Edit the path to Gecode inside the script to match your system's configuration. Linux users should add the Gecode library to the \texttt{LD\_LIBRARY\_PATH} variable. Go to the \texttt{/etc/ld.so.conf.d} folder and create a new \texttt{.conf} file if one does not already exist. In this file, paste the full path to the Gecode library, save it, and run \texttt{sudo ldconfig} to update the system with the new library. Don't forget to restart OpenMusic and don't stop believing. Following these steps should ensure the proper utilization of FuxCP.

\begin{figure}[h]
    \centering
    \includegraphics[height=2.6in]{Images/openmusic_library.png}
    \caption{Opening the "Library" window in OpenMusic.}
    \label{fig:library}
\end{figure}
\begin{figure}[h]
    \centering
    \includegraphics[height=2.6in]{Images/openmusic_load.png}
    \caption{Loading the "fuxcp" library in OpenMusic.}
    \label{fig:load}
\end{figure}

\section{Using FuxCP in OpenMusic}
It is straightforward to use FuxCP in OpenMusic. There is a single block comprising the entire graphical interface of the tool. This block or class is called \texttt{cp-params}. To load it, it is possible to type \texttt{fuxcp::cp-params} in a new patch entry; or load the block of the class by loading "cp-params" from the drop-down menu by right-clicking in the patch ($Classes\to Libraries\to FuxCP\to Solver\to CP-PARAMS$). Alternatively, you can just double-click anywhere in the patch, type "fuxcp::cp-params" and press Enter. This also works for "poly", "voice" and "x-append".

Once this block has appeared, all you have to do is bind an OM voice object, representing the \cfcomma to the second argument of \texttt{cp-params} as shown in figure \ref{fig:om_ext_interface_mod}. Don't forget to block the input voice object and evaluate \texttt{cp-params} so it can detect the new input. Now \texttt{cp-params} can be blocked too. From now on, you could directly use the interface and generate counterpoints using the tool. If you want to retrieve the voice object containing the counterpoint generated by the tool, just bind the third argument on the output side to a voice object. Once bound, it is then possible to evaluate the voice object so that it updates.

If you want to get the whole composition in one object, you have to do some fiddling with OpenMusic. The simplest way to do this is shown in Figure \ref{fig:om_ext_interface_mod}, and works as follows: get the POLY object returned by CP-PARAMS on its third output, split this object in two (the two voices), then get the \cfs, which is the second output of CP-PARAMS, and put all the voices back together in the desired order using x-append functions. 

\begin{figure}[h]
    \centering
    \includegraphics[width=5.2in]{Images/om_ext_interface_mod.png}
    \caption{View of a patch using \texttt{fuxcp::cp-params} in OpenMusic.}
    \label{fig:om_ext_interface_mod}
\end{figure}

But how do you use the interface? Simply double-click on the block to bring it up. The interface is sorted from left to right, so the preferences are divided into three different categories: "Preferences for Melodic Intervals of\dots", "General Preferences", "Species-Specific Preferences", "Solver Configuration", "Cost importance order of\dots", and, in the bottom right corner, "Solver Launcher" (see figure \ref{fig:om_int_interface}). Once you have chosen the preferences, the default ones representing the Fux style, you have to save the parameters ("Save Config") in order to start the search for a solution ("Next Solution"). This search may take a fraction of a second, or it may take tens of minutes if the parameters chosen make the search difficult. If the search takes too long, you can always stop it by clicking on "Stop". You can then either change the settings in some way (often by changing the voice range).

The parameters are described in the next section. As far as the "Cost importance order" panel is concerned, it works by setting an order of importance for the costs. This means that it will give priority to minimising the cost that has a high importance. The importance ranges from one to thirteen, with one being the most important. The option between Linear combination and Maximum minimisation is to know, in case two costs are set to have the same importance, how to combine the equally important costs at one level of the lexicographic search;

\paragraph{}
Pressing the "Next Solution" button will display the solution as a pop-up. What appears on the screen are the two counterpoints. The \cfs must be added manually as described at the beginning of this section. 


The other option is to press the "Best Solution" button. This will start an infinite search that will only stop when the best solution has been found (which can take hours). You can evaluate the output object at any time to see what the best result is so far, and this will not stop the search, so you can see how the solution improves step by step, and stop the search when it has produced something you are happy with.


Please note that the preferences do not affect the speed of finding the first solution. The first solution is the first valid solution and is not affected by the cost. Only the subsequent solutions can be affected by the preferences.

\begin{figure}[h]
    \includegraphics[width=1.2\textwidth, center]{Images/om_int_interface.png}
    \caption{User interface of the \texttt{fuxcp::cp-params} class in OpenMusic.}
    \label{fig:om_int_interface}
\end{figure}

\section{Interface Parameters Description} \label{appendix:interface-parameters-description}
Table \ref{tab:cp-params} describes all the parameters available in the interface. A low cost represents a high preference while a high cost represents a low preference.

\begin{table}[!h]
    \footnotesize
    \begin{adjustbox}{center}
        \begin{tabular}{|m{0.22\textwidth}|m{0.82\textwidth}|m{0.15\textwidth}<{\centering}|}
        \hline
        \multicolumn{1}{|c|}{\textbf{Name}} &
          \multicolumn{1}{c|}{\textbf{Description}} &
          \textbf{Default value} \\ \hline
        \cellcolor[HTML]{BCE08D}Step &
          Preference for melodic intervals of one step or less. &
          No cost \\ \hline
        \cellcolor[HTML]{BCE08D}Third &
          Preference for melodic third skips. &
          Low cost \\ \hline
        \cellcolor[HTML]{BCE08D}Fourth &
          Preference for melodic fourth leaps. &
          Low cost \\ \hline
        \cellcolor[HTML]{BCE08D}Tritone &
          Preference for melodic tritone leaps. &
          Forbidden \\ \hline
        \cellcolor[HTML]{BCE08D}Fifth &
          Preference for melodic fifth leaps. &
          Medium cost \\ \hline
        \cellcolor[HTML]{BCE08D}Sixth &
          Preference for melodic sixth leaps. &
          Medium cost \\ \hline
        \cellcolor[HTML]{BCE08D}Seventh &
          Preference for melodic seventh leaps. &
          Medium cost \\ \hline
        \cellcolor[HTML]{BCE08D}Octave &
          Preference for melodic octave leaps. &
          Low cost \\ \hline
        \hline
        \cellcolor[HTML]{C8D6FF}Apply specific penultimate note rules &
          Force all rules on the notes of the penultimate measure. This mainly refers to the penultimate note that must harmonically be either a major sixth or a minor third depending on whether the counterpoint is above or below. &
          Checked \\ \hline
        \cellcolor[HTML]{C8D6FF}Borrowing mode &
          Type of scale from which notes can be borrowed to generate counterpoint. The first note of the \cfs determines the tonic of this scale. Applies everywhere except the penultimate bar. &
          Major \\ \hline
        \cellcolor[HTML]{C8D6FF}Borrowed notes &
          Preference for borrowed notes outside the diatonic scale. These notes are defined by the "Borrowing mode" parameter. &
          High cost \\ \hline
        \cellcolor[HTML]{C8D6FF}Fifths in down beats &
          Preference to have harmonic fifths on the first beat of a bar. &
          Low cost \\ \hline
        \cellcolor[HTML]{C8D6FF}Octaves in down beats &
          Preference to have harmonic octaves on the first beat of a bar. &
          Low cost \\ \hline
        \cellcolor[HTML]{C8D6FF}Contrary motions &
          Preference to have, between two bars, one voice rising while the other is falling. &
          No cost \\ \hline
        \cellcolor[HTML]{C8D6FF}Oblique motions &
          Preference to have, between two bars, one static voice while the other is moving. &
          Low cost \\ \hline
        \cellcolor[HTML]{C8D6FF}Direct motions &
          Preference to have, between two bars, the two voices going in the same direction. &
          Medium cost \\ \hline
        \cellcolor[HTML]{C8D6FF}Successive perfect consonances &
          Preference to have as few successive perfect consonances as possible. &
          Medium cost \\ \hline
        \hline
        \cellcolor[HTML]{FFCE93}2nd: Penultimate thesis note is not a fifth &
          Preference for the first note of the penultimate bar to be something other than a harmonic fifth &
          Last resort \\ \hline
        \cellcolor[HTML]{FFCE93}3rd: Non-cambiata notes &
          Preference for the second quarter note of a bar to be a consonance already surrounded by two consonances. &
          High cost \\ \hline
        \cellcolor[HTML]{FFCE93}3rd: Same notes two beats apart &
          Preference to have the same quarter notes two beats apart. A high cost allows to avoid a certain monotony. &
          Low cost \\ \hline
        \cellcolor[HTML]{FFCE93}3rd: Force joint contrary melody after skip &
          Force that a melodic skip or leap is followed by a melodic step in the opposite direction. &
          Unchecked \\ \hline
        \cellcolor[HTML]{FFCE93}4th: Same syncopations two bars apart &
          Preference to have the same half notes two bars apart. A high cost allows to avoid a certain monotony. &
          High cost \\ \hline
        \cellcolor[HTML]{FFCE93}4th: No syncopation &
          Preference to have distinct half notes instead of syncopations. &
          Last resort \\ \hline
        \cellcolor[HTML]{FFCE93}5th: Preferences to a lot of quarters or a lot of syncopations &
          Determines the minimum percentage of quarter notes (to the left) and syncopations (to the right) in the fifth species. Pushing the slider all the way to one side is not recommended. &
          <center> \\ \hline
        \hline
        \cellcolor[HTML]{EFEFEF}Chosen species &
          Determines the type of counterpoint that the tool will generate. From whole notes to syncopations, passing through quarter notes. The fifth species uses the rules and preferences of all other species. &
          1st \\ \hline
        \cellcolor[HTML]{EFEFEF}Voice range &
          Determines around which pitch the counterpoint will be generated depending on the pitch of the first note of the \cfdot &
          Above and very far above \\ \hline
        \cellcolor[HTML]{EFEFEF}Minimum \% of skips &
          Determines, depending on the counterpoint size, the percentage of melodic intervals larger than one step. &
          0\% \\ \hline
        \cellcolor[HTML]{D1D1D1}Save Config &
          Saves all established preferences and allows you to start a new search for this configuration later. &
          - \\ \hline
        \cellcolor[HTML]{D1D1D1}Next Solution &
          Starts or continues searching for the previously saved configuration. Displays a new window with the solution when it is found. Displays an error message if no other solution can be found. &
          - \\ \hline
        \cellcolor[HTML]{D1D1D1}Stop &
          Pause the search. It may take up to 5 seconds. &
          - \\ \hline
        \end{tabular}
    \end{adjustbox}
    \caption{Description of the parameters of \texttt{fuxcp::cp-params}.}
    \label{tab:cp-params}
\end{table}


\chapter{Complete set of rules for two and three part compositions}\label{appendix:complete-set-of-rule}
This appendix contains all the rules for composing counterpoint for two or three voices. All the rules apply to three-voice compositions, but only the rules for two voices apply to two-voice compositions. Rules for three-part compositions are indicated by '3V' at the beginning of the rule.


Some of the rules are different depending on whether the composition is for two or three voices. In these cases, the mathematical relationship to be followed for the rule in question, depending on the situation, is clearly indicated.
\section*{Constraints of the First Species}
\subsection*{Harmonic Constraints of the First Species}
\begin{enumerate}[wide, label=\bfseries 1.H\arabic*]
  \item\label{rule:allcons}{\textit{All harmonic intervals must be consonances.}} 
\begin{equation}
    \begin{gathered}
        \forall j \in [0, m)\quad 
        H[0, j] \in Cons
    \end{gathered}
\end{equation}

\item\label{rule:firstpcons}{\textit{The first harmonic interval must be a perfect consonance.}}
When dealing with two-part composition:
\begin{equation}
    \begin{gathered}
        H[0, 0] \in Cons_{p}
    \end{gathered}
\end{equation}

\item\label{rule:lastpcons}{\textit{The last harmonic intervals must be a perfect consonance.}}
When dealing with three-part composition:
\begin{equation}
  \begin{gathered}
      H[0, m-1] \in Cons_{p}
  \end{gathered}
\end{equation}

\item\label{rule:keytone}{\textit{The key tone is tuned according to the first note of the \cfdot}}

\begin{equation}
    \begin{gathered}
        \lnot IsCfB[0, 0] \implies H[0, 0] = 0\\
        \lnot IsCfB[0, m-1] \implies H[0, m-1] = 0
    \end{gathered}
\end{equation}

\item{\textit{The counterpoint and the \cfs cannot play the same note at the same time except in the first and last measure.}}

\begin{equation}
    \begin{gathered}
        \forall j \in [1, m-1)\quad
        Cp[0, j] \neq Cf[j]
    \end{gathered}
\end{equation}

\item{\textit{Imperfect consonances are preferred to perfect consonances.}}


\begin{equation}
    \begin{gathered}
        \forall j \in [0, m)\\
        Pcons_{costs}[j] = \begin{cases}
            cost_{Pcons} & \text{if } H[0, j] \in Cons_{p}\\
            0 & \text{otherwise}
        \end{cases}\\
        \text{moreover } \C = \C \cup \sum _{c \in Pcons_{costs}} c
    \end{gathered}
\end{equation}

\item{and \textbf{1.H8} \textit{The harmonic interval of the penultimate note must be a major sixth or a minor third depending on the \cfs pitch.}}\label{rule:penult-interval-2v}
\addtocounter{enumi}{1} 
\begin{equation}
    \begin{gathered}
        % \lnot IsCfB[0, m-2] \implies H[0, m-2] = 9
        \rho := \max (positions(m)) - 1\\
        H[\rho] = \begin{cases}
            9 & \text{if } IsCfB[\rho]\\
            3 & \text{otherwise}
        \end{cases}\\
        \text{where } \rho \text{ represents the penultimate index of any counterpoint.}
    \end{gathered}
\end{equation}

\subsection*{Melodic Constraints of the First Species}
\end{enumerate}

\begin{enumerate}[wide, label=\bfseries 1.M\arabic*]
\item\label{rule:notritone}{\textit{Tritone melodic intervals are forbidden.} }

\begin{equation}
    \begin{gathered}
        \forpm\\
        M[\rho] = 6 \implies Mdeg_{costs}[\rho] = cost_{tritoneMdeg}\\
    \end{gathered}
\end{equation}

\item\label{rule:mlesixth}{\textit{Melodic intervals cannot exceed a minor sixth interval.}}

\begin{equation}
    \begin{gathered}
        \forj\quad
        M[0, j] \leq 8
    \end{gathered}
\end{equation}

\subsection*{Motion Constraints of the First Species}
\end{enumerate}
\begin{enumerate}[wide, label=\bfseries 1.P\arabic*]

\item\label{rule:nopconsbydm}{ \textit{Perfect consonances cannot be reached by direct motion.}}

When dealing with two-part composition:
\begin{equation}
    \begin{gathered}
        \forj\quad
        H[0, j+1] \in Cons_{p} \implies P[0, j] \neq 2
    \end{gathered}
\end{equation}

When dealing with three-part composition:
\begin{equation} \begin{aligned}
  &\forall j \in [0, m-2) :\\
  &P[0, j] = 2 \land H[0, j+1] \in Cons_{p} \\
  &\iff cost_{\text{{direct\_move\_to\_p\_cons}}}[j] = 8
\end{aligned} \end{equation}

\item\label{rule:codmotions} {\textit{Contrary motions are preferred to oblique motions which are preferred to direct motions.}}

\begin{multicols}{3}
    \begin{itemize}
        \item $cost_{con}$\\ \dft{no cost}
        \item $cost_{obl}$\\ \dft{low cost}
        \item $cost_{dir}$\\ \dft{medium cost}
    \end{itemize}
\end{multicols}

\begin{equation}
    \begin{gathered}
        \forj\\
        P_{costs}[j] = \begin{cases}
            cost_{con} & \text{if } P[0, j] = 0\\
            cost_{obl} & \text{if } P[0, j] = 1\\
            cost_{dir} & \text{if } P[0, j] = 2
        \end{cases}\\
        \text{moreover } \C = \C \cup \sum _{c \in P_{costs}} c
    \end{gathered}
\end{equation}

\item\label{rule:battuta}{ \textit{At the start of any measure, an octave cannot be reached by the lower voice going up and the upper voice going down more than a third skip.}}


\begin{equation}
    \begin{gathered}
        i := \max (\B), \forj\\
        H[0, j+1] = 0 \land P[i, j] = 0 \land \begin{cases}
            M_{brut}[i, j] < -4 \land IsCfB[i, j] \iff \bot\\
            M_{cf}[i, j] < -4 \land \lnot IsCfB[i, j] \iff \bot
        \end{cases}\\
        \text{where } i \text{ stands for the last beat index in a measure.}
    \end{gathered}
    \label{eq:battuta}
\end{equation}
\end{enumerate}

\section*{Constraints of the Second Species}
\subsection*{Harmonic Constraints of the Second Species}
\begin{enumerate}[wide, label=\bfseries 2.H\arabic*]
\item\label{rule:consthesis}{ \textit{Thesis harmonies cannot be dissonant.}}

As explained above, there is no constraint to add because it would be a duplicate of rule \ref{rule:allcons}.

\item\label{rule:arsisdim}{\textit{Arsis harmonies cannot be dissonant except if there is a diminution.}}

\begin{equation}
    \begin{gathered}
        \forj\\
        IsDim[j] = \begin{cases}
            \top & \text{if } M^2[0, j] \in \{3, 4\} \land M^1[0, j] \in \{1, 2\} \land M^1[2, j] \in \{1, 2\}\\
            \bot & \text{otherwise}
        \end{cases}
    \end{gathered}
\end{equation}

\begin{equation}
    \begin{gathered}
        \forj \quad
        \lnot IsCons[2, j] \implies IsDim[j]
    \end{gathered}
\end{equation}

\item\label{rule:penult2nd} \label{rule:penultexception}{and \textbf{2.H4} \textit{In the penultimate measure the harmonic interval of perfect fifth must be used for the thesis note if possible. Otherwise, a sixth interval should be used instead.}}
\addtocounter{enumi}{1}

\begin{equation}
    \begin{gathered}
        H[0, m-2] \in \{7, 8, 9\}\\
        \therefore penulthesis_{cost} = \begin{cases}
            cost_{penulthesis} & \text{if } H[0, m-2] \neq 7\\
            0 & \text{otherwise}
        \end{cases}\\
        \text{moreover } \C = \C \cup penulthesis_{cost}
    \end{gathered}
\end{equation}

\end{enumerate}
\subsection*{Melodic Constraints of the Second Species}
\begin{enumerate}[wide, label=\bfseries 2.M\arabic*]

\item\label{rule:octaveleap}{ \textit{If the two voices are getting so close that there is no contrary motion possible without crossing each other, then the melodic interval of the counterpoint can be an octave leap.}}

\begin{equation}
    \begin{gathered}
        \forj, \forall M_{cf}[j] \neq 0\\
        M[0, j] = 12 \implies (H_{abs}[0, j] \leq 4) \land (IsCfB[j] \iff M_{cf}[j]>0)
    \end{gathered}
\end{equation}

\item\label{rule:notsamecons}{ \textit{Two consecutive notes cannot be the same.}}
When dealing with two-part composition:
\begin{equation}
    \begin{gathered}
        \forp \quad
        Cp[\rho] \neq Cp[\rho+1]
    \end{gathered}
\end{equation}

When dealing with three-part composition:
\begin{equation}
  \begin{aligned}
      &\forall j \in [1, m-1), \quad j \neq m-2:\\
      &((N[2, j-1] \neq N[0, j]) \land (N[0, j] \neq \land N[2, j])) \\
      &\land \\
      & ((N[2, m-3] \neq N[0, m-2]) \lor (N[0, m-2] \neq N[2, m-2]) )
  \end{aligned}
\end{equation}


\end{enumerate}
\subsection*{Motion Constraints of the Second Species}
\begin{enumerate}[wide, label=\bfseries 2.P\arabic*]

\item\label{rule:motion2nd}{\textit{If the melodic interval of the counterpoint between the thesis and the arsis is larger than a third, then the motion is perceived based on the arsis note.}}


\begin{equation}
    \begin{gathered}
        \forj \quad
        P_{real}[j] = \begin{cases}
            P[2, j] & \text{if } M[0, j] > 4\\
            P[0, j] & \text{otherwise}
        \end{cases}
    \end{gathered}
\end{equation}

\item\label{rule:battuta2}{ \textit{Rule \ref{rule:battuta} on the battuta octave is adapted such that it focuses on the motion from the note in arsis.}}
\item 
This constraint already had an adapted mathematical notation in the chapter of
the first species. Note that this constraint would indeed use P[2] and not P$_{real}$.


\end{enumerate}
%%%%%%%%%%%%%

\section*{Constraints of the Third Species}
\subsection*{Harmonic Constraints of the Third Species}
\begin{enumerate}[wide, label=\bfseries 3.H\arabic*]
  \item\label{rule:fivequarters}{ \textit{If five notes follow each other by joint degrees in the same direction, then the harmonic interval of the third note must be consonant.}}

\begin{equation}
    \begin{gathered}
        \forj\\
        % \{M[0, j]\land M[1, j]\land M[2, j]\land M[3, j]\} \leq 2\ \land\\
        % \left(
        %     \{M_{brut}[0, j]\land M_{brut}[1, j]\land M_{brut}[2, j]\land M_{brut}[3, j]\} > 0\ \lor \right. \\
        %     \left.
        %     \{M_{brut}[0, j]\land M_{brut}[1, j]\land M_{brut}[2, j]\land M_{brut}[3, j]\} < 0\
        % \right)\\
        % \implies IsCons[2, j]
        \left(
            \bigwedge_{i=0}^{3} M[i, j] \leq 2
        \right)
        \land
        \left(
            \bigwedge_{i=0}^{3} M_{brut}[i, j] > 0
            \lor
            \bigwedge_{i=0}^{3} M_{brut}[i, j] < 0
        \right)\\
        \implies IsCons[2, j]
    \end{gathered}
\end{equation}


\item\label{rule:thirddiss} {\textit{If the third harmonic interval of a measure is dissonant then the second and the fourth interval must be consonant and the third note must be a diminution.}}


\begin{equation}
    \begin{gathered}
        \forj\\
        IsCons[2, j] \lor \left( IsCons[1, j] \land IsCons[3, j] \land IsDim[j]\right)\\
        \text{where } IsDim[j]=\top \text{ when the \nth{3} note of the measure } j \text{ is a diminution.}
    \end{gathered}
\end{equation}

\item\label{rule:cambiata} {\textit{It is best to avoid the second and third harmonies of a measure to be consonant with a one-degree melodic interval between them.}}


\begin{equation}
    \begin{gathered}
        \forj\\
        Cambiata_{costs}[j] = \begin{cases}
            cost_{Cambiata} & \text{if } IsCons[1, j] \land IsCons[2, j] \land M[1, j] \leq 2\\
            0 & \text{otherwise}
        \end{cases}
    \end{gathered}
\end{equation}

\item\label{rule:penult3sp} {\textit{In the penultimate measure, if the \cfs is in the upper part, then the harmonic interval of the first note should be a minor third.}}

\begin{equation}
    \begin{gathered}
        \lnot IsCfB[m-2] \implies H[0, m-2] = 3
    \end{gathered}
\end{equation}
\end{enumerate}
\subsection*{Melodic Constraints of the Third Species}
\begin{enumerate}[wide, label=\bfseries 3.M\arabic*]
  \item\label{rule:twobeats} {\textit{Each note and its two beats further peer are preferred to be different.}}


\begin{equation}
    \begin{gathered}
        \forpmm \\
        MtwoSame_{costs}[i, j] = \begin{cases}
            cost_{MtwobSame} & \text{if } M^2[\rho] = 0\\
            0 & \text{otherwise}
        \end{cases}
    \end{gathered}
\end{equation}
\end{enumerate}
\subsection*{Motion Constraints of the Third Species}
\begin{enumerate}[wide, label=\bfseries 3.P\arabic*]
  \item\label{rule:motion3rd} {\textit{The motion is perceived based on the fourth note.}}

This implies that the costs of the motions and the first species constraints on the motions are deducted from $P[3]$.
\end{enumerate}
%%%%%%%%%%%%%%%%%%%%%


\section*{Constraints of the Fourth Species}

\subsection*{Motion Constraints of the Fourth Species}
\begin{enumerate}[wide, label=\bfseries 4.P\arabic*]
  \item\label{rule:dissolved} {\textit{Dissonant harmonies must be followed by the next lower consonant harmony.}}

\begin{equation}
    \begin{gathered}
        \forall j \in [1, m-1) \quad
        \lnot IsCons[0, j] \implies M_{brut}[0, j] \in \{-1, -2\}
    \end{gathered}
\end{equation}

\item\label{rule:nosecond} {\textit{If the \cfs is in the lower part then no second harmony can be preceded by a unison/octave harmony.}}

\begin{equation}
    \begin{gathered}
        \forall j \in [1, m-1)\\
        IsCfB[j+1] \implies H[2, j] \neq 0 \land H[0, j+1] \notin \{1, 2\}
    \end{gathered}
\end{equation}

\end{enumerate}
\subsection*{Harmonic Constraints of the Fourth Species}
\begin{enumerate}[wide, label=\bfseries 4.H\arabic*]
  \item\label{rule:arsiscons} {\textit{Arsis harmonies must be consonant.}}

\begin{equation}
    \begin{gathered}
        \forall j \in [0, m-1) \quad
        H[2, j] \in Cons
    \end{gathered}
    \label{eq:arsiscons}
\end{equation}

\item\label{rule:noseventh} {\textit{If the \cfs is in the upper part, then no harmonic seventh interval can occur.}}

\begin{equation}
    \begin{gathered}
        \forall j \in [1, m-1) \quad
        \lnot IsCfB[j] \implies H[0, j] \notin \{10, 11\}
    \end{gathered}
\end{equation}

\item\label{rule:lowpenult4th} \label{rule:uppenult4th} {and \textbf{4.H4} \textit{In the penultimate measure, the harmonic interval of the thesis note must be a major sixth or a minor third depending on the \cfs pitch.}}

\begin{equation}
    \begin{gathered}
        H[0, m-2] = \begin{cases}
            9 & \text{if } IsCfB[m-2]\\
            3 & \text{otherwise}
        \end{cases}
    \end{gathered}
\end{equation}
\end{enumerate}
\subsection*{Melodic Constraints of the Fourth Species}
\begin{enumerate}[wide, label=\bfseries 4.M\arabic*]
  \item\label{rule:fullsyncopations} {\textit{Arsis half notes should be the same as their next halves in thesis.}}


\begin{equation}
    \begin{gathered}
        \forall j \in [0, m-1) \quad
        NoSync_{costs} = \begin{cases}
            cost_{NoSync} & \text{if } M[2, j] \neq 0\\
            0 & \text{otherwise}
        \end{cases}
    \end{gathered}
\end{equation}

\item\label{rule:m2same} {\textit{Each arsis note and its two measures further peer are preferred to be different.}}


\begin{equation}
    \begin{gathered}
        \forall j \in [0, m-1)\\
        MtwomSame_{costs} = \begin{cases}
            cost_{MtwomSame} & \text{if } Cp[2, j] = Cp[2, j+2]\\
            0 & \text{otherwise}
        \end{cases}
    \end{gathered}
\end{equation}

\end{enumerate}
%%%%%%%%%%%%%%
\chapter{Code}\label{chapter:whole-code}
\section{FuxCP.lisp}
\lstinputlisting{../FuxCP/FuxCP.lisp}
\section{package.lisp}
\lstinputlisting{../FuxCP/sources/package.lisp}
\section{interface.lisp}
\lstinputlisting{../FuxCP/sources/interface.lisp}
\section{fuxcp-main.lisp}
\lstinputlisting{../FuxCP/sources/fuxcp-main.lisp}
\section{3v-ctp.lisp}
\lstinputlisting{../FuxCP/sources/3v-ctp.lisp}
\section{cf.lisp}
\lstinputlisting{../FuxCP/sources/cf.lisp}
\section{1sp-ctp.lisp}
\lstinputlisting{../FuxCP/sources/1sp-ctp.lisp}
\section{2sp-ctp.lisp}
\lstinputlisting{../FuxCP/sources/2sp-ctp.lisp}
\section{3sp-ctp.lisp}
\lstinputlisting{../FuxCP/sources/3sp-ctp.lisp}
\section{4sp-ctp.lisp}
\lstinputlisting{../FuxCP/sources/4sp-ctp.lisp}
\section{5sp-ctp.lisp}
\lstinputlisting{../FuxCP/sources/5sp-ctp.lisp}
\section{constraints.lisp}
\lstinputlisting{../FuxCP/sources/constraints.lisp}


%add all translations, gecode, gil, t.Wafflard's thesis AND article, Bitsch, 

\includepdf[pages=1]{BackPage.pdf}
\end{document}