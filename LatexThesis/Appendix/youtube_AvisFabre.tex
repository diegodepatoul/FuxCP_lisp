\begin{quotation}
    "Ça ce n'est pas bien, j'ai trois fois sol, même deux fois je m'en prive. Alors bon, exceptionnellement je peux permettre de temps en temps d'avoir deux fois la même note mais c'est vrai que dans les traités tels qu'on les utilise, ceux de par exemple: Marcel Bitsch, Marcel Dupré, les traités du XIXème siècle, on évite, enfin on proscrit même la répétition de la note. Bon et bien ça c'est une règle de bon sens en fait. Ce n'est pas une règle imposée comme ça de manière arbitraire. C'est que le contrepoint doit être une ligne en perpétuel mouvement [\dots]. Attention, chez Fux il le fait, donc c'est intéressant de voir que lui se permet ce genre de choses."
    \textcite[Jean-Louis Fabre's opinion on the repetition of the same note in counterpoint.][1min 11]{ReglesJLFabre}
    \captionof{trans}{French transcription of the video \citetitle{ReglesJLFabre} for rule \ref{rule:codmotions}.}
    \label{appen:JLFabreAvis}
\end{quotation}

Which can be translated as:
\begin{quotation}
    This is not good, I have three times G, even twice I do not use it. So, exceptionally, I can allow from time to time to have the same note twice, but it is true that in the treatises as we use them, those of for example: Marcel Bitsch, Marcel Dupré, the treatises of the XIXth century, we avoid, well we even proscribe the repetition of the note. Well, this is a rule of common sense in fact. It is not a rule imposed arbitrarily. It is that the counterpoint must be a line in perpetual movement [\dots]. Mind you, Fux does this, so it's interesting to see that he allows himself this kind of thing.
    \captionof{trans}{English translation of the above quotation \ref{appen:JLFabreAvis}.}
\end{quotation}